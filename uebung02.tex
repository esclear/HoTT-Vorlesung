\documentclass{uebung}

\begin{document}
\maketitle{2}

\begin{exercise}
  Zeige $1\cdot 1\equiv 1:\N$.
\end{exercise}

\begin{exercise}[Zwei Einführungsregeln für Nachfolger]
  Zeige, dass die folgenden beiden Einführungsregeln für den Nachfolger einer natürlichen Zahl äquivalent sind:
  \begin{mathpar}
    \inferrule{
      \Gamma\vdash n:\N
    }{
      \Gamma\vdash\sucN(n):\N
    }{
      \N\mathrm{I2}
    }
    \and
    \inferrule{
      \Gamma\text{ Kontext}
    }{
      \Gamma,x:\N\vdash\sucN(x):\N
    }{
      \N\mathrm{I2'}
    }
  \end{mathpar}
  {\tiny Die linke Regel ist aus dem Skript, die rechte aus der Vorlesung.}
\end{exercise}

\begin{exercise}[$\N$-rekursive Funktionen]
  Konstruiere folgende Funktionen auf $\N$ durch Pattern Matching und durch Rekursion.
  \begin{enumerate}
    \item Die Dreieckszahlen $n\mapsto n+\dots+1$ vom Typ $\N\to\N$.
    \item Die Fakultätsfunktion $n\mapsto n!$ vom Typ $\N\to\N$.
    \item Die Potenzfunktion $m,n\mapsto n^m$ vom Typ $\N\to\N\to\N$.
    \item Die Minimumsfunktion $m,n\mapsto\min(m,n)$ vom Typ $\N\to\N\to\N$.
    \item Die Maximumsfunktion $m,n\mapsto\max(m,n)$ vom Typ $\N\to\N\to\N$.
    \item Den Binomialkoeffizienten $m,n\mapsto\binom{m}{n}$ vom Typ $\N\to\N\to\N$.
  \end{enumerate}
\end{exercise}

\begin{exercise}[Die Ackermannfunktion]
  Die \emph{Ackermannfunktion} $\mathrm{ack}:\N\to\N\to\N$ ist durch folgende Rekursionsgleichungen gegeben:
  \begin{align*}
    \mathrm{ack}(0,n)&\equiv\sucN(n)\\
    \mathrm{ack}(\sucN(m),0)&\equiv\mathrm{ack}(m,1)\\
    \mathrm{ack}(\sucN(m),\sucN(n))&\equiv\mathrm{ack}(m,\mathrm{ack}(\sucN(m),n))
  \end{align*}
  Konstruiere $\mathrm{ack}$ unter Verwendung von $\rec \N$.
\end{exercise}

\begin{exercise}[Kartesische Produkte]
  Seien $A$, $B$ Typen.
  Konstruiere einen Typ $A\times B$, welcher folgende Voraussetzungen erfüllt:
  \begin{enumerate}
    \item Für je zwei Terme $a:A$ und $b:B$ lässt sich ein Term $t_{a,b}:A\times B$ konstruieren.
    \item Es lassen sich Funktionen $\mathrm{pr}_1:A \times B\to A$ und $\mathrm{pr}_2:A\times B\to B$ konstruieren.
    \item Es ist $\mathrm{pr}_1(t_{a,b})\equiv a:A$ und $\mathrm{pr}_2(t_{a,b})\equiv b:B$.
  \end{enumerate}
\end{exercise}

\begin{exercise}[Fibonacci]
  Konstruiere die Fibonaccifolge $\N\to\N$.
  {\tiny Tipp: Verwende das kartesische Produkt $\N\times\N$.}
\end{exercise}

\begin{exercise}
  Warum lässt sich mittels Pfadinduktion kein Term vom Typ $f:\prod_{x:A}\prod_{p:x=x}p=\refl_x$ konstruieren, welcher die Gleichung $f(x,\refl_x)\equiv \refl_{\refl_x}$ erfüllt?
\end{exercise} 

\begin{exercise}
  Sei $C$ eine Typfamilie über einem Typen $A$, also ein Typ $C$ im Kontext $x:A$.
  Verwende Pfadinduktion um eine abhängige Funktion $f:\prod_{x,y:A}\prod_{x=y:A} C(x)\to C(y)$ zu konstruieren, welche $f(x,x,\refl_x)=\id_{C(x)}$ erfüllt.
\end{exercise}

\end{document}
