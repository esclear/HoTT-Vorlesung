\documentclass{uebung}

\begin{document}
\maketitle{2}

\begin{exercise}
  Zeige $1\cdot 1\equiv 1:\N$. Orientiere Dich dabei am Beispiel $1 + 1\equiv 2$ aus der Vorlesung.
\end{exercise}

\begin{exercise}[$\N$-rekursive Funktionen]
  Konstruiere folgende Funktionen auf $\N$ durch ihre definierenden Gleichhungen und durch Rekursion.
  \begin{enumerate}
    \item Die Dreieckszahlen $n\mapsto n+\dots+1$ vom Typ $\N\to\N$.
    \item Die Fakultätsfunktion $n\mapsto n!$ vom Typ $\N\to\N$.
    \item Die Potenzfunktion $m,n\mapsto n^m$ vom Typ $\N\to\N\to\N$.
    \item Die Minimumsfunktion $m,n\mapsto\min(m,n)$ vom Typ $\N\to\N\to\N$.
    \item Die Maximumsfunktion $m,n\mapsto\max(m,n)$ vom Typ $\N\to\N\to\N$.
    \item Den Binomialkoeffizienten $m,n\mapsto\binom{m}{n}$ vom Typ $\N\to\N\to\N$.
  \end{enumerate}
\end{exercise}

\begin{exercise}[Die Ackermannfunktion]
  Die \emph{Ackermannfunktion} $\mathrm{ack}:\N\to\N\to\N$ ist durch folgende Rekursionsgleichungen gegeben:
  \begin{align*}
    \mathrm{ack}(0,n)&\equiv\sucN(n)\\
    \mathrm{ack}(\sucN(m),0)&\equiv\mathrm{ack}(m,1)\\
    \mathrm{ack}(\sucN(m),\sucN(n))&\equiv\mathrm{ack}(m,\mathrm{ack}(\sucN(m),n))
  \end{align*}
  Konstruiere $\mathrm{ack}$ unter Verwendung von $\rec \N$.
\end{exercise}


\begin{exercise}
  Zeige, dass für die in der Vorlesung definierten Funktionen $f:1\amalg 1\to \zwei$ und $g:\zwei\to 1\amalg 1$ in zueinander invers sind, indem Du Terme der folgenden Typen konstruierst:
  \begin{mathpar}
    \prod_{x:1\amalg 1} g(f(x))=x\quad\quad \prod_{y:\zwei} f(g(y))=y.
  \end{mathpar}
\end{exercise}

\end{document}
