\documentclass{uebung}

\begin{document}
\maketitle{2}

\begin{exercise}
  Zeige $1\cdot 1\equiv 1:\N$.
\end{exercise}


\begin{exercise}[Zwei Einführungsregeln für Nachfolger]
  Zeige, dass die folgenden beiden Einführungsregeln für den Nachfolger einer natürlichen Zahl äquivalent sind:
  \begin{mathpar}
    \inferrule{
      \Gamma\vdash n:\N
    }{
      \Gamma\vdash\sucN(n):\N
    }{
      \N\mathrm{I2}
    }
    \and
    \inferrule{
      \Gamma\text{ Kontext}
    }{
      \Gamma,x:\N\vdash\sucN(x):\N
    }{
      \N\mathrm{I2'}
    }
  \end{mathpar}
  {\tiny Die linke Regel ist aus dem Skript, die rechte aus der Vorlesung.}
\end{exercise}

\begin{exercise}[$\N$-rekursive Funktionen]
  Definiere folgende Funktionen auf $\N$ durch Pattern Matching und durch Rekursion.
  \begin{enumerate}
    \item Die Dreieckszahlen $n\mapsto n+\dots+1$ vom Typ $\N\to\N$.
    \item Die Fakultätsfunktion $n\mapsto n!$ vom Typ $\N\to\N$.
    \item Die Potenzfunktion $m,n\mapsto n^m$ vom Typ $\N\to\N\to\N$.
    \item Die Minimumsfunktion $m,n\mapsto\min(m,n)$ vom Typ $\N\to\N\to\N$.
    \item Die Maximumsfunktion $m,n\mapsto\max(m,n)$ vom Typ $\N\to\N\to\N$.
    \item Den Binomialkoeffizienten $m,n\mapsto\binom{m}{n}$ vom Typ $\N\to\N\to\N$.
    \item Die Fibonaccifolge vom Typ $\N\to\N$.
  \end{enumerate}
\end{exercise}

\end{document}
