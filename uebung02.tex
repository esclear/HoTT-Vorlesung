\documentclass{uebung}

\begin{document}
\maketitle{2}

\begin{exercise}[Das sehr kleine Einmaleins]
  Zeige $1\cdot 1\equiv 1:\N$.
  Gehe dabei ähnlich detailliert wie beim Beispiel $1+1\equiv 2:\N$ im Skript vor.
\end{exercise}

\begin{exercise}[$\N$-rekursive Funktionen]
  Konstruiere folgende Funktionen auf $\N$ durch ihre definierenden Gleichhungen und durch Rekursion.
  \begin{enumerate}
    \item Die Fakultätsfunktion $n\mapsto n!$ vom Typ $\N\to\N$.
    \item Die Potenzfunktion $m,n\mapsto n^m$ vom Typ $\N\to\N\to\N$.
    \item Die Maximumsfunktion $m,n\mapsto\max(m,n)$ vom Typ $\N\to\N\to\N$.
    \item Den Binomialkoeffizienten $m,n\mapsto\binom{m}{n}$ vom Typ $\N\to\N\to\N$.
  \end{enumerate}
\end{exercise}

\begin{exercise}[Die Ackermannfunktion]
  Die \emph{Ackermannfunktion} $\mathrm{ack}:\N\to\N\to\N$ ist durch folgende Rekursionsgleichungen gegeben:
  \begin{align*}
    \mathrm{ack}(0,n)&\equiv\sucN(n)\\
    \mathrm{ack}(\sucN(m),0)&\equiv\mathrm{ack}(m,1)\\
    \mathrm{ack}(\sucN(m),\sucN(n))&\equiv\mathrm{ack}(m,\mathrm{ack}(\sucN(m),n))
  \end{align*}
  Konstruiere $\mathrm{ack}$ unter Verwendung von $\rec \N$.
\end{exercise}


\begin{exercise}[Zwei Boolesche Typen]
  Seien $f:\zwei\to 1\amalg 1$ und $g:1\amalg 1\to\zwei$ die Funktionen aus der Vorlesung mit definierenden Gleichungen
  \begin{align*}
    f(0_{\zwei})&:\equiv\iota_1(\ast) &g(\iota_1(\ast)):\equiv 0_{\zwei}\\
    f(1_{\zwei})&:\equiv\iota_2(\ast)&g(\iota_2(\ast)):\equiv 1_{\zwei}.
  \end{align*}
  Zeige, dass $f$ und $g$ zueinander invers sind, indem du Terme für die folgenden Typen konstruierst:
  \begin{enumerate}
    \item $\prod_{y:\zwei} g(f(y))=_\zwei y$
    \item $\prod_{x:1\amalg 1} f(g(x))=_{1\amalg 1} x$
  \end{enumerate}
\end{exercise}

\end{document}
