\documentclass{uebung}

\begin{document}
\maketitle{4}

\begin{exercise}[Speziellere Rechengesetze für Gleichheiten]
  Seien $A$ ein Typ und $x,y,z:A$. Zeige:
  \begin{enumerate}
  \item Für Gleichheiten $p:x=y$ gilt $\left(p^{-1}\right)^{-1}=p$.
  \item Für Gleichheiten $p:x=y$ und $q:y=z$ gilt: $(p\kon q)^{-1}=q^{-1}\kon p^{-1}$.
  \end{enumerate}
\end{exercise}

\begin{exercise}[Kontrahierbare Typen sind äquivalent zu $\einheit$]
  Sei $A$ ein kontrahierbarer Typ.
  Finde Funktionen $f:A\to\einheit$ und $g:\einheit\to A$ welche invers zueinander sind.
\end{exercise}

\begin{exercise}[Inverse sind eindeutig]
  Seien $A,B,C$ Typen.
  \begin{enumerate}
  \item Seien $f,f':A\to B$ Funktionen und $f\sim f'$, also $f$ homotop zu $f'$.
    Dann ist für $h:B\to C$, auch $h\circ f$ homotop zu $h\circ f'$.
    Analog ist auch für $i:Z\to A$ die Funktion $f\circ i$ homotop zu $f'\circ i$.
  \item Seien $f:A\to B$ und $g:B\to A$, $g':B\to A$ jeweils eine Invers von $f$.
    Dann sind $g$ und $g'$ homotop.
  \item Seien $f:A\to B$ und $g:B\to A$ invers zueinander und $f':B\to C$, $g':C\to B$ invers zueinander.
    Dann sind auch $f'\circ f$ und $g\circ g'$ invers zueinander.
  \end{enumerate}
\end{exercise}

\begin{exercise}[Verallgemeinerte de Morgansche Regeln]
  Seien $A,B,C$ Typen.
  Konstruiere einen Term 
  \begin{equation*}
    \mathrm{deMorgan}:((A+B)\to C) \leftrightarrow ((A\to C)\times (B\to C)).
  \end{equation*}
  Sind $\pi_1(\mathrm{deMorgan})$ und $\pi_2(\mathrm{deMorgan})$ invers zueinander?
\end{exercise}

\begin{exercise}[Spaß mit Negationen]
  Sei $A$ ein Typ.
  Wir schreiben $\neg A:\equiv A\to\leer$.
  Zeige:
  \begin{enumerate}
    \item $A\to\neg\neg A$
    \item $\neg\neg\neg A\to\neg A$
    \item $\neg\neg(A+\neg A)$
    \item $\neg\neg(\neg\neg A\to A)$
    \item $(A+\neg A)\leftrightarrow (\neg\neg A\to A)$
  \end{enumerate}
\end{exercise}

\begin{exercise}[Alternative Charakterisierung von Aussagen]
  Sei $A$ ein Typ.
  Zeige, dass $A$ genau dann eine Aussage ist, wenn alle Gleichheitstypen von $A$ kontrahierbar sind, das heißt, konstruiere Funktionen
  \begin{enumerate}
    \item $\prod_{x,y:A}\isContr(x=_Ay)\to\isProp(A)$,
    \item \textbf{Bonus.} $\isProp(A)\to\prod_{x,y:A}\isContr(x=_Ay)$.
  \end{enumerate}
  Verwende dies, um zu zeigen, dass
  \begin{enumerate}[start=3]
    \item $\isContr(A)\to\isProp(A)$, {\tiny Tipp:Blatt 3}
    \item $\isProp(A)\to\isSet(A)$.
  \end{enumerate}
\end{exercise}

\begin{bonus}[Kontrahierbarkeit ist eine Aussage]
  Zeige, dass für jeden Typ $A$ der Typ
  \begin{mathpar}
    \isContr'(A)\equiv\sum_{c:A}\prod_{x:A}c=x
  \end{mathpar}
  ein Aussage ist. Folgere daraus, dass auch $\isContr(A)$ eine Aussage ist.
\end{bonus}

\end{document}
