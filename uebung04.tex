\documentclass{uebung}

\begin{document}
\maketitle{4}

\begin{exercise}[Zwei Isomorphismen]
  Seien $A,B,C$ Typen.
  Konstruiere zueinander inverse Funktionen zwischen den folgenden Typen:
  \begin{enumerate}
    \item $A$ und $\einheit$, falls $A$ kontrahierbar ist.
    \item $(A \amalg B) \to C$ und $(A\to C) \times (B\to C)$.
      {\tiny Eine Richtung benötigt Funktionsextensionalität.}
  \end{enumerate}
\end{exercise}

\begin{exercise}[Inverse sind eindeutig]
  Sei $f:A\to B$ eine Funktion mit Inversen $g,g':B\to A$.
  Zeige, dass $\prod_{y:B}g(b)=g'(b)$.
\end{exercise}

\begin{exercise}[Speziellere Rechengesetze für Gleichheiten]
  Seien $A$ ein Typ und $x,y,z:A$. Zeige:
  \begin{enumerate}
  \item Für Gleichheiten $p:x=y$ gilt $\left(p^{-1}\right)^{-1}=p$.
  \item Für Gleichheiten $p:x=y$ und $q:y=z$ gilt: $(p\kon q)^{-1}=q^{-1}\kon p^{-1}$.
  \end{enumerate}
\end{exercise}

\begin{bonus}[Kontrahierbarkeit ist eine Aussage]
  Zeige, dass für jeden Typ $A$ der Typ
  \begin{mathpar}
    \isContr(A)\equiv\sum_{c:A}\prod_{x:A}x=c
  \end{mathpar}
  ein Aussage ist.
\end{bonus}

\end{document}
