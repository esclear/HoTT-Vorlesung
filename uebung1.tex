\documentclass{uebung}

\begin{document}
\maketitle{1}

\begin{exercise}[Urteilsgleichheit ist eine Äquivalenzrelation]
  Formuliere Regeln, welche besagen, dass Urteilsgleichheit $s\equiv t:A$ eine
  \begin{enumerate}
    \item reflexive,
    \item symmetrische,
    \item transitive
  \end{enumerate}
  Relation zwischen Termen desselben Typs ist.
\end{exercise}

\begin{exercise}[Assoziativität der Komposition]
  Seien $f:A\to B$, $g:B\to C$ und $h:C\to D$ Funktionen.
  \begin{enumerate}
    \item Stelle einen Herleitungsbaum für $x\mapsto h(g(f(x))):A\to D$ auf.
    \item Leite aus $x\mapsto h(g(f(x))):A\to D$ folgendes ab:
      \begin{enumerate}
        \item $h\circ (g\circ f):A\to D$.
        \item $(h\circ g)\circ f:A\to D$.
      \end{enumerate}
    \item Kombiniere dies zu einer Herleitung von $h\circ(g\circ f)\equiv (h\circ g)\circ f:A\to D$.
  \end{enumerate}
\end{exercise}

\end{document}
