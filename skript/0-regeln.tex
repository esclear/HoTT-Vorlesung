\section*{Regeln}
Wir werden zunächst nur Regeln kennenlernen, die Teil einer abhängigen Typentheorie sind.
Regeln einer (abhängigen) Typentheorie können etwa so aussehen:
\begin{mathpar}
\inferrule*{\Gamma \yields f : A\to B \and \Gamma \yields t : A}{\Gamma \yields f(t) : B}
\end{mathpar}
Über dem waagrechten Strich stehen die Voraussetzungen und darunter, was aus den Voraussetzungen geschlossen werden darf.
Dabei kann man zum Beispiel ``$\Gamma \yields t : A$'' lesen als ``Im Kontext $\Gamma$ gibt es einen Term $t$ des Typs $A$''.
Wir werden nur anfangs mit Regeln arbeiten, um ein Verständnis für Typentheorie zu erlangen.
Irgendwann werden wir Regeln wie die obige wieder, wie in der Mathematik üblich, sprachlich formulieren,
etwa so
\begin{center}
  Für $f : A\to B$ und $t : A$ gibt es ein $f(t) : B$.
\end{center}
Wie wir später sehen werden, können diese Regeln zu Herleitungsbäumen kombiniert werden.
Dieses Kombinieren ist der vollständig formale Weg Beweise in der Homotopietypentheorie zu führen.

Ein Kontext wie ``$\Gamma$'' darf man sich als Liste von Variablen zusammen mit ihren Typen vorstellen. Also etwa so:
\begin{mathpar}
  x_1 : A_1, \cdots ,x_n : A_n
\end{mathpar}
Dabei dürfen die Variablen nach ihrer Einführung verwendet werden.
So könnte etwa in der Konstruktion des Typs $A_2$ die Variable $x_1$ verwendet werden.
Dass der Kontext eine solche Liste ist, wird üblicherweise durch die strukturellen Regeln einer abhängigen Typentheorie festgelegt.
\begin{table}
  \centering
  \begin{tabular}{ll}
    Urteil                        & Bedeutung (evtl. im Kontext $\Gamma$) \\
    \hline
    $\Gamma\yields t : A$         & $t$ ist ein Term vom Typ $A$ \\
    $\Gamma\yields A$ Typ         & $A$ ist ein Typ \\
    $\Gamma\yields A\equiv B$     & $A$ und $B$ sind (urteils-)gleiche Typen \\
    $\Gamma\yields t\equiv s : A$ & $t$ und $s$ sind (urteils-)gleiche Terme des Typs $A$ \\
    $\Gamma$ Kontext              & $\Gamma$ ist ein Kontext
  \end{tabular}
  \caption{Urteile}
  \label{tab:urteile}
\end{table}

Ein Block der Form ``$\Gamma \yields t : A$''  ist ein spezielles \begriff{Urteil}.
Insgesamt gibt es die Urteile in \cref{tab:urteile}.

Eine Regel ist im Allgemeinen Fall nun von dieser Form:
\begin{mathpar}
\inferrule{\mathcal U_1 \and \dots \and \mathcal U_n}{\mathcal U_0}{\mathrm{Name}}
\end{mathpar}
Für $n\in\mathbb N=\{0,1,\dots\}$ und Urteile $\mathcal U_0,\dots,\mathcal U_n$.

\subsection*{Strukturregeln}

Neben den Regeln für einzelne Typen, die wir in den folgenden Abschnitten kennenlernen, gibt es sogenannte \begriff{strukturelle Regeln} oder \begriff{Strukturregeln}.
Diese legen fest, wie Grundsätzliches funktioniert, etwa wie Kontexte geformt werden dürfen und was man mit Gleichheitsurteilen anfangen darf.
Hier ist ein Beispiel, die sogenannte ``Weakening''-Regel oder \begriff{Abschwächungsregel}:
\begin{mathpar}
\inferrule{\Gamma \yields \mathcal U \and \Gamma \yields A \text{ Typ}}{\Gamma, x : A \yields \mathcal U}{\Weak}
\end{mathpar}
Das gilt für alles was man durch andere Regeln an der Stelle von $\mathcal U$ bekommen könnte.
Die folgende \begriff{Variablenregel} erlaubt es Variablen aus dem Kontext zu benutzen:
\begin{mathpar}
  \inferrule{\Gamma, x:A\text{ Kontext}}{\Gamma, x:A\yields x:A }{\mathrm{Var}}
\end{mathpar}
Die Regeln für die Urteilsgleichheit legen fest, dass diese eine Äquivalenzrelation auf Termen und Typen ist.
Außerdem gibt es Strukturregeln und sogenannte \begriff{Kongruenzregeln}, die es letztendlich erlauben, Terme und Typen durch Urteilsgleiches zu ersetzen.
Wir erlauben es uns einfach zu Ersetzen und führen diese Regeln nicht aus.

Weiter sind Typentheorien typischerweise so aufgebaut, dass wenn es möglich ist $t:A$ herzuleiten, es auch immer möglich ist, $A\text{ Typ}$ herzuleiten,
in welchem Fall es wieder möglich ist, herzuleiten, dass der Kontext in dem das gilt ein Kontext ist.
Allgemeiner erlauben wir uns stets notwendige Voraussetzungen für vorliegende Urteile zu verwenden.
Also etwa der Schluss von einem Urteil, in dem der Typ ``$A\to B$'' vorkommt, zum Urteil ``$A$ Typ''.
Wir wollen diese Tatsachen frei verwenden und wie Regeln einsetzen, die wir zusammenfassend mit ``$\mathrm{Str}$'' bezeichnen.

Diese Regeln führen wir hier nicht auf, eine gute Quelle, um die genauen Regeln im Bedarfsfall anzuschauen,
ist das (noch nicht erschienene) Lehrbuch von Egbert Rijke oder Anhang A.2 des HoTT-Books.
