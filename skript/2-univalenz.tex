\subsection{Äquivalenzen}
\label{sub:aequivalenzen}

In diesem Abschnitt werden wir drei verschiedene Definitionen von Äquivalenzen einführen und zeigen,
dass es sich um den selben Begriff handelt. Einen haben wir bereits eingeführt - Funktionen mit kontrahierbaren Fasern heißen Äquivalenz.
Von einem Äquivalenzbegriff erwarten wir, dass es sich beim entsprechenden Typ $\isEquiv(f)$ um eine Aussage handelt.
Das hat zur Folge, dass wir es als eine Eigenschaft von Funktionen anssehen können eine Äquivalenz zu sein.
Damit ist dann auch der Typ der Äquivalenzen zwischen $A$ und $B$ eine Untertyp von $A\to B$, d.h. die Vergiß-Abbildung
\begin{mathpar}
  \pi_1:\left(\sum_{f:A\to B}\isEquiv(f)\right)\to (A\to B)
\end{mathpar}
ist injektiv.

Zur Vorbereitung beschäftigen wir uns noch etwas mit Homotopien.
\begin{bemerkung}
  Seien $A,B:\mU$ und $f,g:A\to B$.
  \begin{enumerate}
  \item Sei $H:f\sim g$. Für $\varphi:A'\to A$ gibt es eine Homotopie $H_{\varphi(\_)}:f\circ \varphi\sim g\circ \varphi$ und für $\psi:B\to B'$ gibt es eine Homotopie $\psi(H):\psi\circ f\sim \psi\circ g$. Diese Operationen nennt man \begriff{whiskering}.
  \item Sei $H:f\sim g$. Homotopien sind natürlich in folgendem Sinn: Für $p:x=_A y$ gilt:
    \begin{mathpar}
      H_x\kon g(p) = f(p)\kon H_{y}
    \end{mathpar}
  \item Für eine Homotopie $H:f\sim\id$ gilt $f(H_x)=H_{f(x)}$.
  \end{enumerate}
\end{bemerkung}


Wir beginnen mit einer scheinbar unbedeutenden Variation des Begriffs Quasi-Inverse und legen dabei mehr Notation für Äquivalenzen fest:

\begin{definition}
  Seien $A,B:\mU$ und $f:A\to B$.
  \begin{enumerate}
  \item $f$ hat eine \begriff{Linksinverse}, wenn es ein $g:B\to A$ gibt und $g\circ f\sim\id_A$.
  \item $f$ hat eine \begriff{Rechtsinverse} oder einen \begriff{Schnitt}, wenn es ein $h:B\to A$ gibt und $f\circ h\sim \id_B$.
  \item $f$ ist eine \begriff{Äquivalenz} oder eine Funktion mit \begriff{Links- und Rechtsinversen}, wenn
    \begin{mathpar}
      \isEquiv(f)\colonequiv\mathrm{LRInv}(f)\colonequiv  \left(\sum_{g:B\to A}g\circ f\sim\id_A\right)\times\left(\sum_{h:B\to A}f\circ h\sim\id_B\right)
    \end{mathpar}
  \item Wenn es eine Äquivanz zwischen $A$ und $B$ gibt, dann sagen wir $A$ ist \begriff{äquivalent} zu $B$.
  \item Für den Typ der Äquivalenzen zwischen $A$ und $B$ schreiben wir:
    \begin{mathpar}
      A\simeq B\colonequiv\sum_{f:A\to B}\isEquiv(f)
    \end{mathpar}
  \end{enumerate}
\end{definition}

Es wird sich herausstellen, dass $\mathrm{LRInv}(f)$ und $\mathrm{qinv}(f)$ für manche Funktionen keine äquivalenten Typen sind.
Insbesondere werden wir noch sehen, dass $\qinv(f)$ immer eine Aussage ist, $\mathrm{LRInv}(f)$ aber keine Aussage mehr sein muss, wenn $A$ und $B$ komplizierte Gleichheitstypen haben.
Die beiden Begriffe sind aber logisch äquivalent, d.h. es gibt für jedes $f:A\to B$ Abbildungen $\mathrm{LRInv}(f)\to \qinv(f)$ und $\qinv(f)\to\mathrm{LRInv}(f)$.
Zunächst halten wir die Vokabel ``logisch äquivalent'' fest:

\begin{definition}
  Zwei Typen $A$ und $B$ heißen \begriff{logisch äquivalent}, wenn es Funktionen $f:A\to B$ und $g:B\to A$ gibt.
\end{definition}

Bei Aussagen reicht die logische Äquivalenz bereits, damit die Typen äquivalent sind.
Damit werden die verschiedenen Varianten des Typs ``$\isEquiv(f)$'', die wir in diesem Abschnitt kennenlernen, äquivalente Typen sein,
wenn wir logische Äquivalenz gezeigt haben und dass es sich jeweils um Aussagen handelt. Letzteres werden wir allerdings noch Aufschieben.

\begin{bemerkung}
  \label{bem:lrinv-qinv}
  Seien $A,B:\mU$ und $f:A\to B$. Die Typen $\mathrm{LRInv}(f)$ und $\qinv(f)$ sind logisch äquivalent.
\end{bemerkung}
\begin{beweis}
  Nehmen wir zünachst an, es gibt eine Quasi-Inverse $g:B\to A$ mit $H:g\circ f\sim\id_A$ und $K:f\circ g\sim\id_B$. Dann ist
  \begin{mathpar}
    ((g,H),(g,K)):\mathrm{LRInv}(f)
  \end{mathpar}
  Seien nun andererseits $g,h:B\to A$ links- und rechtsinvers zu $f$. Wir zeigen, dass $g$ und $h$ homotop sind. Damit können wir dann zeigen, dass $g$ auch rechtsinvers ist.
  Durch whiskering erhalten wir:
  \begin{mathpar}
    g\sim g\circ (f\circ h) \sim (g\circ f)\circ h  \sim h
  \end{mathpar}
  Also ist $g$ rechtsinvers:
  \begin{mathpar}
    f\circ g\sim f\circ h\sim \id
  \end{mathpar}
\end{beweis}

Wenn wir also zu einer Funktion $f:A\to B$ eine beidseitige Inverse konstruieren, dann ist $f$ eine Äquivalenz (im Sinn einer Funktion mit Links- und Rechtsinversen).
Äquivalenzen haben die Operationen, die wir bereits von Gleichheiten kennen:

\begin{bemerkung}
  \begin{enumerate}
  \item Die Identität ist eine Äquivalenz.
  \item Eine Inverse einer Äquivalenz ist eine Äquivalenz.
  \item Seien $A,B,C:\mU$. Wenn $f:A\to B$ und $g:B\to C$ Äquivalenzen sind, dann ist $g\circ f:A\to C$ eine Äquivalenz.
  \end{enumerate}
\end{bemerkung}
\begin{beweis}
  Wir zeigen die Aussagen für den Äquivalenzbegriff der Funktion mit Links- und Rechtsinversen.
  Mit \cref{thm:aequivalenzen} gilt damit das gleiche für die anderen Äquivalenzbegriffe.
  \begin{enumerate}
  \item Die Identität ist ihre eigene Links- und Rechtsinverse. Die nötigen Homotopien gelten schon als urteilsmäßige von Funktionen.
  \item Wenn $f:A\to B$ eine Linksinverse $g:B\to A$ und eine Rechtsinverse $h:B\to A$ hat, dann ist bekommen wir mit Bemerkung \labelcref{bem:lrinv-qinv} eine Quasi-Inverse $f^{-1}:B\to A$.
    Und $f^{-1}$ ist eine Äquivalenz mit Linksinverser $f$ und Rechtsinverser $f$.
  \item Auf Übungsblatt 4 wird das für Quasi-Inversen gezeigt. Zusammen mit Bemerkung \labelcref{bem:lrinv-qinv} gilt das also auch für Äquivalenzen.
  \end{enumerate}
\end{beweis}

\begin{beispiel}[Transport ist Äquivalenz]
  Für $A:\mU$, $B:A\to\mU$, $x,y:A$ und jedes $p:x=y$ ist $\transp_B(p):B(x)\to B(y)$ eine Äquivalenz mit Inverser $\transp_B(p^{-1}):B(y)\to B(x)$.
\end{beispiel}

Wir wollen nun den Äquivalenzbegriff der Links- und Rechtsinvertierbarkeit mit dem der kontrahierbaren Fasern in Zusammenhang bringen.
Wir werden allerdings erstmal eine Implikation klären und für die zweite einen dritten Äquivalenzbegriff einführen.

\begin{bemerkung}
  \label{bem:isContr-lrinv}
  Seien $A,B:\mU$ und $f:A\to B$. Wenn alle Fasern von $f$ kontrahierbar sind, dann hat $f$ Links- und Rechtsinverse, es gibt also eine Funktion:
  \begin{mathpar}
    \left(\prod_{y:B}\isContr(f^{-1}(y))\right) \to \mathrm{LRInv}(f)
  \end{mathpar}
\end{bemerkung}

\begin{beweis}
  Sei $k:\prod_{y:B}\isContr(f^{-1}(y))$.
  Wir konstruieren eine beidseitige Inverse $g:B\to A$ indem wir $y:B$ auf das Kontraktionszentrum der Faser $f^{-1}(y)$ abbilden:
  Für $\pi_1(k_y)$ ist ein Element von $f^{-1}(y)$, besteht also aus $x:A$ und $p:f(x)=y$, also können wir festlegen
  \begin{mathpar}
    g(y)\colonequiv x
  \end{mathpar}
  Dann ist $f(g(y))=f(x)=y$, also $g$ Rechtsinverse von $f$. Für $x:A$ sei $x'\colonequiv g(f(x))$.
  Es müssen $x:A$ und $x':A$ beide in $f^{-1}(f(x))$ liegen und da diese Faser kontrahierbar ist, gibt es eine Gleichheit $q:x=x'=g(f(x))$.
  Genauer haben wir durch die Kontrahierbarkeit eine Gleichheit $q':(x,\refl_{f(x)})=(x',p')$ in der Faser $f^{-1}(f(x))$.
  Diese Gleichheit können wir mit $\pi_1:f^{-1}(f(x))\to A$ abbilden, also $q\colonequiv \pi_1(q')$ setzen.
\end{beweis}

Die Umkehrung dieser Aussage ist komplizierter zu zeigen und wir werden zunächst zeigen, dass sich Links- Rechtsinverse in eine sogenannte kohärente Inverse überstzen lassen.
Kohärente Inverse $g:B\to A$ einer Funktion $f:A\to B$ haben eine Kompatiblität zwischen den beiden Homotopien $f\circ g\sim \id$ und $g\circ f\sim \id$.
Für eine Funktion $f:A\to B$ mit Inverser $g:B\to A$ besagt diese Kohärenz, dass die beiden Möglichkeiten eine Homotopie $f\circ g\circ f\sim f$ zu konstruieren punktweise gleich sind.

\begin{definition}
  Seien $A,B:\mU$ und $f:A\to B$.
  Eine Funktion $g:B\to A$ heißt \begriff{kohärente Inverse} von $f$, wenn $f$ und $g$ es Homotopien $H:g\circ f\sim \id$ und $K:f\circ g \sim \id$ gibt und
  \begin{mathpar}
    \mathrm{koh} : \prod_{x:A}f(H_x)=K_{f(x)}
  \end{mathpar}
  Wir schreiben $\mathrm{CohInv}(f)$ für den Typ der \begriff{kohärenten Inversen} von $f$.
\end{definition}

Wir wollen direkt einsehen, dass wir von dieser Sorte Äquivalenz die Kontrahierbarkeit der Fasern beweisen können:

\begin{bemerkung}
  \label{bem:qinv-equiv}
  Seien $A,B:\mU$ und $f:A\to B$. Wenn $f$ eine kohärente Inverse hat, dann sind die Fasern von $f$ kontrahierbar.
\end{bemerkung}
Vor dem Beweis noch ein Hilfslemma, das wir auch sonst noch wiederverwenden können:
\begin{lemma}
  \label{lem:gleichheit-in-faser}
  Seien $A,B$ Typen, $f:A\to B$ und $y:B$. Für $(x,q),(x',q'):f^{-1}(y)$ gibt es eine Funktion:
  \begin{mathpar}
    \prod_{p:x=_A x'}f(p)^{-1}\kon q=q'\to (x,q)=(x',q')
  \end{mathpar}
\end{lemma}
\begin{beweis}[von \cref{lem:gleichheit-in-faser}]
  Induktion über $p$ und $\sum_=$.
\end{beweis}

\begin{beweis}[von \labelcref{bem:qinv-equiv}]
  Habe also $f:A\to B$ eine kohärente Inverse, gebe es also $g:B\to A$, $H:g\circ f\sim \id$, $K:f\circ g \sim \id$ und $\mathrm{koh} : \prod_{x:A}f(H_x)=K_{f(x)}$.
  Wir müssen nun für jede Faser von $f$ eine Kontraktion angeben.
  Als Kontraktionszentrum für die Faser über $y:B$ wählen wir
  \begin{mathpar}
    k_y\colonequiv (g(y), K_y) : f^{-1}(y)
  \end{mathpar}
  Seien nun also $(x,q),(x',q'):f^{-1}(y)$. Damit haben wir auch $q\kon q'^{-1}:f(x)=f(x')$ und damit:
  \begin{mathpar}
    H_x^{-1}\kon g(q\kon q'^{-1}) \kon H_x':x=x'
  \end{mathpar}
  Darauf wenden wir jetzt $f$ an:
  \begin{mathpar}
    f(H_x^{-1}\kon g(q\kon q'^{-1}) \kon H_x'):f(x)=f(x')
  \end{mathpar}
  und berechnen mit natürlichkeit von Homotopien, der Kohärenz und Gruppoidgesetzen:
  \begin{align*}
    f(H_x^{-1}\kon g(q\kon q'^{-1}) \kon H_x')&=f(H_x^{-1})\kon f(g(q\kon q'^{-1})) \kon f(H_x') \\
                                              &=f(H_x^{-1})\kon K_{f(x)}\kon q\kon q'^{-1}\kon K_{f(x')}^{-1} \kon f(H_x') \\
                                              &=q\kon q'^{-1}
  \end{align*}
  Mit dem Lemma haben wir also die gewünschte Gleichheit.
\end{beweis}

\begin{bemerkung}
  Seien $A,B:\mU$ und $f:A\to B$ habe eine beidseitige Inverse $g:B\to A$.
  Dann ist $g$ auch eine kohärente Inverse von $f$.
\end{bemerkung}
\begin{beweis}
  Seien $g,h:B\to A$ und $H:g\circ f\sim\id$, $K:f\circ g\sim \id$. Setzte für $y:B$:
  \begin{mathpar}
    K'\colonequiv K_{f(g(y))}^{-1}\kon f(H_{g(y)})\kon K_{y}
  \end{mathpar}
  und rechne nach dass für $x:A$ gilt: $f(H_x)=K'_{f(x)}$.
\end{beweis}

\begin{theorem}
  \label{thm:aequivalenzen}
  Seien $A,B:\mU$ und $f:A\to B$, dann sind die folgenden Äquivalenzbegriffe logisch äquivalent:
  \begin{enumerate}
  \item Alle Fasern von $f$ sind kontrahierbar:
    \begin{mathpar}
      \isEquiv(f)\colonequiv\prod_{y:B}\isContr(f^{-1}(y))
    \end{mathpar}
  \item $f$ hat eine Linksinverse und eine Rechtsinverse:
    \begin{mathpar}
      \isEquiv(f)\colonequiv\mathrm{LRInv}\equiv  \left(\sum_{g:B\to A}g\circ f\sim\id_A\right)\times\left(\sum_{h:B\to A}f\circ h\sim\id_B\right)
    \end{mathpar}
  \item $f$ hat eine kohärente Inverse:
    \begin{mathpar}
      \isEquiv(f)\colonequiv\mathrm{CohInv}(f)\colonequiv \sum_{g:B\to A} \sum_{H:g\circ f\sim\id}\sum_{K:f\circ g\sim\id} \prod_{x:A}f(H_x)=K_{f(x)}
    \end{mathpar}
  \end{enumerate}
  Wir schreiben für alle drei Begriffe $\isEquiv(f)$, weil sich noch rausstellen wird, dass alle diese Typen Aussagen sind und damit alle drei Typen äquivalent sind.
\end{theorem}
\begin{beweis}
  Die Bemerkungen dieses Abschnitts beweisen per Ringschluss die logische Äquivalenz.
\end{beweis}

\subsection{Univalenz}
In diesem Abschnitt werden wir uns mit den Universen und ihren Gleichheitstypen beschäftigen.
Eine historisch wichtige Idee von Vladimir Voevodsky ist das \begriff{Univalenzaxiom},
das die Gleichheitstypen im Universum mit den Äquivalenzen von Typen identifiziert.
Dieses Axiom werden wir kennenlernen und Konsequenzen daraus ziehen.
Davor wollen wir aber noch allgemeine Konsequenzen aus der Existenz von Universen ziehen.

Mit Universen sind wir in der Lage Ungleichheit von Elementen eines Typs zu beweisen.
Diese definieren wir zunächst als Negation der Gleichheit:
\begin{definition}
  Seien $A:\mU$ und $x,y:A$. Dann ist $x$ \begriff{ungleich} $y$, wenn
  \begin{mathpar}
    x\not= y\colonequiv (x=y\to \leer)
  \end{mathpar}
\end{definition}
Man beachte, dass es sich bei diesem Typ stets um eine Aussage handelt.
Es sind also in $x\not= y$ keine so interessanten Dinge zu finden, wie im Gleichheitstyp und wir werden uns auch wenig mit Ungleichheiten beschäftigen.
\begin{bemerkung}
  Es gilt $1_\zwei \not= 0_\zwei$.
\end{bemerkung}
\begin{beweis}
  Per Rekursion können wir uns den folgenden abhängigen Typen definieren:
  \begin{align*}
    B&:\zwei\to\mU \\
    B(0_\zwei)&\colonequiv \leer \\
    B(1_\zwei)&\colonequiv \eins
  \end{align*}
  Um die Ungleichheit zu zeigen, dürfen wir $p:1_\zwei = 0_\zwei$ annehmen und müssen ein Element in $\leer$ konstruieren.
  Sei also $p:1_\zwei = 0_\zwei$. Damit haben wir auch eine Abbildung:
  \begin{mathpar}
    \transp_B(p):B(1_\zwei)\to B(0_\zwei)
  \end{mathpar}
  Und wegen $\ast:B(1_\zwei)\equiv \eins$ haben wir auch ein Element, das wir in diese Abbildung einsetzen können.
  Es ist also $\transp_B(p)(\ast):B(0_\zwei)\equiv \leer$.
\end{beweis}
Dieser Trick lässt sich auf alle Elemente induktiver Typen übertragen, für die wir ``verschiedene'' Werte vorgeben können.
Für einzelne natürliche Zahlen etwa:
\begin{bemerkung}
  Es gilt $0_{\N}\not= 1_{\N}\colonequiv \sucN(0_{\N})$.
\end{bemerkung}
\begin{beweis}
  Sei
  \begin{align*}
    B&:\N\to \mU \\
    B(0_{\N})&\colonequiv \leer \\
    B(1_{\N})&\colonequiv \eins
  \end{align*}
  womit für $p:0_{\N}=1_{\N}$ ein Element $\transp_B(p^{-1})(\ast):\leer$ gegeben ist.
\end{beweis}
Typischerweise interessiert man sich für eine vollständige Charakterisierung der Gleichheitstypen eines Typs.
Für die natürlichen Zahlen wäre das ein in $n,k:\N$ abhängiger Typ $B(n,k)$ sodass $B(n,k)\simeq (n=_\N k)$.
Das wird unser Ziel in \cref{sec:zahlen-gleichheit} sein.

Ohne Weiteres können wir bereits für $A,B:\mU_i$ den Typ $A=_{\mU_i}B$ der Gleichheiten zwischen $A$ und $B$ formen.
Dieser liegt allerdings im nächsthöheren Universum $\mU_{i+1}$, da wir in der Formierungsregel für Gleichheitstypen ``X ist ein Typ'' durch ``$X : \mU_i$'' ersetzen und dafür nur ``$\mU_i : \mU_{i+1}$'' in Frage kommt:
\begin{mathpar}
  \inferrule{\Gamma\yields \mU_i:\mU_{i+1}\and \Gamma\yields A,B:\mU_i}{\Gamma\yields A=_{\mU_i}B:\mU_{i+1}}{=\mathrm{F}}
\end{mathpar}

Damit lässt sich zeigen, dass gleiche Typen äquivalent sind.
\begin{bemerkung}
  Für $A,B:\mU$ gibt es eine Funktion
  \begin{mathpar}
    \transp_\mU:A=_{\mU} B\to (A\simeq B)
  \end{mathpar}
  den \begriff{Universentransport}.
\end{bemerkung}
\begin{beweis}
  Seien $A,B:\mU_i$. Dann gibt es einen abhängigen Typ $X:\mU_i\to\mU_{i+1}$ und damit
  \begin{mathpar}
    \transp_\mU\colonequiv\transp_{X}:A=_{\mU_i}B\to A\simeq B
  \end{mathpar}
\end{beweis}

Es ist also möglich, aus einer Gleichheit von Typen eine Äquivalenz zu kontruieren.
Die Umkehrung, aus einer Äquivalenz auch eine Gleichheit konstruieren zu können, nennt man das Univalenzaxiom.
\begin{axiom}[Univalenzaxiom]
  Wir nehmen von nun an an, dass das \begriff{Univalenzaxiom} gilt: Für $A,B:\mU$ ist die Funktion
  \begin{mathpar}
    \transp_\mU:A=_{\mU}B\to A\simeq B
  \end{mathpar}
  eine Äquivalenz ist. Die Inverse von $\simeq_=$ bezeichnen wir mit $\ua$\index{$\ua$}:
  \begin{mathpar}
    \ua:A\simeq B\to A=_{\mU}B
  \end{mathpar}
  Wir nehmen an, dass das Univalenzaxion für jedes Universum gilt.
\end{axiom}

Mit Univalenz ergibt sich, dass es tatsächlich verschiedene Gleichheiten zwischen Elementen geben kann.
\begin{beispiel}
  Es gilt $(\zwei\simeq\zwei)\simeq\zwei$ und per Univalenz damit auch:
  \begin{mathpar}
    f:\zwei\simeq (\zwei=_{\mU}\zwei)
  \end{mathpar}
  Also: $f(0_\zwei)\not=f(1_\zwei)$.
\end{beispiel}

\begin{bemerkung}
  Für $A:\mU$ gilt $\ua(\id_A)=\refl_A$.
\end{bemerkung}
\begin{beweis}
  Es gilt $\transp_\mU(\refl_A)=\id_A$ und damit $\ua(\id_A)=\ua(\transp_\mU(\refl_A))=\refl_A$.
\end{beweis}

Da wir nun wissen, dass Gleichheiten und Äquivalenzen dasselbe sind, können wir auch die Gleichheitsinduktion auf Äquivalenzen übertragen:
\begin{lemma}[Äquivalenzinduktion]
  Für $C:\prod_{A,B:\mU}A\simeq B\to \mU$ gibt es:
  \begin{mathpar}
    \ind{\simeq}:\left(\prod_{A:\mU}C(A,A,\id_A)\right)\to \prod_{A,B:\mU}\prod_{f:A\simeq B}C(A,B,f)
  \end{mathpar}
\end{lemma}
\begin{beweis}
  Durch Gleichheitsinduktion ergibt sich:
  \begin{mathpar}
    \ind{=}:\left(\prod_{A:\mU}C(A,A,\ua(\refl_A))\right)\to\prod_{A,B:\mU}\prod_{p:A=_{\mU}B}C(A,B,\ua(p))
  \end{mathpar}
\end{beweis}

Wir importieren die folgende Aussage (findet man im HoTT-Book am Ende von Kapitel 4): 
\begin{fakt}
  Univalenz impliziert Funktionsextensionalität.
\end{fakt}

Wir verzichten daher im folgenden darauf, auf Verwendung von Funktionsextensionalität hinzuweisen.
Zunächst halten wir fest, dass abhängige Summen und abhängige Produkte mit äquivalenten Eingangsdaten äquivalente Typen produzieren:
\begin{bemerkung}
  Seien $A:\mU$ und $B,B':A\to\mU$. Wenn $B$ und $B'$ punktweise äquivalent sind, also gilt $\prod_{x:A}B(x)\simeq B'(x)$, dann gilt auch:
  \begin{mathpar}
    \left(\sum_{x:A}B(x)\right)\simeq \left(\sum_{x:A}B'(x)\right)\text{ und }\left(\prod_{x:A}B(x)\right)\simeq \left(\prod_{x:A}B'(x)\right)
  \end{mathpar}
\end{bemerkung}
\begin{beweis}
  Aus der punktweisen Äquivalenz $\prod_{x:A}B(x)\simeq B'(x)$ machen wir durch punktweises Anwenden von $\ua$ eine punktweise Gleichheit $\prod_{x:A}B(x)=B'(x)$ und daraus mit Funktionsextensionalität eine Gleichheit $p:B=B'$.
  Dann ist aber auch
  \begin{mathpar}
    \mathrm{ap}(\prod,p):\left(\prod_{x:A}B(x)\right)=\left(\prod_{x:A}B'(x)\right)
  \end{mathpar}
  und damit $\simeq_=(\mathrm{ap}(\prod,p)):\left(\prod_{x:A}B(x)\right)\simeq\left(\prod_{x:A}B'(x)\right)$.
  Mit dem gleichen Argument können wir die entsprechende Aussage für abhängige Summen zeigen.
\end{beweis}
Mit Univalenz lässt sich auch die Gleichheit von abhängigen Typen konkretisieren:
\begin{bemerkung}
  Für $A:\mU$ und $B,B':A\to \mU$ gibt es eine Äquivalenz:
  \begin{mathpar}
    (B=B') \simeq \left(\prod_{x:A}B(x)\simeq B'(x)\right)
  \end{mathpar}
\end{bemerkung}
\begin{beweis}
  
\end{beweis}
Im nächsten Kapitel ergibt sich ein einfacher Beweis einer stärkeren Version von Funktionsextensionalität.

Mit Univalenz können wir folgern, dass für jeden Typ $A$ die abhängige Typen dasselbse sind, wie Funktionen nach $A$.
\begin{lemma}
  Sei $A:\mU$. Dann gilt:
  \begin{mathpar}
    \left(\sum_{B:\mU}(B\to A)\right)\simeq (A\to \mU)
  \end{mathpar}
\end{lemma}

\begin{beweis}
  Einen abhängigen Typ $B:A\to\mU$ bilden wir auf die Funktion
  \begin{mathpar}
    \pi_1:\left(\sum_{x:A}B(x)\right)\to A
  \end{mathpar}
  ab. Und eine Funktion $f:B\to A$ bilden wir auf den abhängigen Typ ihrer Fasern ab:
  \begin{mathpar}
    \mathrm{fib}_f\equiv \left((x:A)\mapsto f^{-1}(x)\right):A\to \mU
  \end{mathpar}
\end{beweis}

\subsection{Gleichheit in den natürlichen Zahlen}
\label{sec:zahlen-gleichheit}
\begin{definition}
  $\Eq{\N}:\N\to\N\to\mU$ definieren wir durch doppelte Rekursion wie folgt:
  \begin{align*}
    \Eq{\N}(0_\N,    0_\N)     &\colonequiv\einheit \\
    \Eq{\N}(0_\N,    \sucN(k)) &\colonequiv\leer \\
    \Eq{\N}(\sucN(n),0_\N)     &\colonequiv\leer \\
    \Eq{\N}(\sucN(n),\sucN(k)) &\colonequiv\Eq{\N}(n,k) 
  \end{align*}
\end{definition}
