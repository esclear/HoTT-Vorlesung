\subsection{Höhere Induktive Typen}


\subsection{n-Typen}
In \cref{def:kontr-aussage-menge} hatten wir bereits kontrahierbare Typen, Aussagen und Mengen definiert.
Die Mengen haben den alternativen Namen $0$-Typen und zu Aussagen, kann man auch $-1$-Typen und zu kontrahierbaren Typen $-2$-Typen sagen.
Wir werden in diesem Abschnitt $n$-Typen definieren, für $n=-2,-1,0,1,\dots$.
Dazu sei zunächst:
\begin{definition}
  Es sei $\N_{-2}$ der induktive Typ mit Konstruktoren $-2:\N_{-2}$ und $\sucHL:\N_{-2}\to\N_{-2}$.
  Für $\sucHL(n)$ schreiben wir $n+1$.
  Die Elemente von $\N_{-2}$ nennen wir auch \begriff{Homotopielevel}.
\end{definition}
Damit können wir $n$-Typen für jeden Homotopielevel $n$ definieren:
\begin{definition}
  \begin{enumerate}
  \item Für jeden Typ $A$ sei der Typ $\isNType{n}$ wie folgt per Rekursion über $n:\N_{-2}$ definiert:
    \begin{align*}
      -2\mathrm{-Type}(A)&\colonequiv \isContr(A) \\
      (n+1)\mathrm{-Type}(A)&\colonequiv \prod_{x,y:A}n\mathrm{-Type}(x=_A y)
    \end{align*}
  \item Der \begriff{Typ der $n$-Typen} ist für $n:\N_{-2}$:
    \begin{mathpar}
      n\mathrm{-Type}\colonequiv \sum_{A:\mU} \isNType{n}
    \end{mathpar}
    Diese Typ gibt es also für jeden Universenlevel.
  \end{enumerate}
\end{definition}


Wir wollen jetzt Regeln für n-Abschneidungen einführen, die wir teilweise bereits für die -1-Abschneidung gesehen hatten.

\begin{regeln}
  Sei $n:\N_{-2}$ ein Homotopielevel und $A:\mU$.
  Es gibt einen Typ $\| A \|_{n}:\mU$, die \begriff{$n$-Abschneidung} von $A$.
  Es gilt $\isNType{n}(\| A\|_n)$ und für $a:A$ gibt es $|a|:\| A\|_n$.
  Weiter ist für jeden abhängigen Typ $B:\| A \|_n\to\mU$, die Abbildung
  \begin{mathpar}
    s\mapsto s\circ|\_|: \left(\prod_{x:\| A \|_n}\| B(x) \|_n\right)\to\left(\prod_{y:A}\| B(|y|)\|_n\right)
  \end{mathpar}
  eine Äquivalenz. Diese Eigenschaft nennen wir die \begriff{universelle Eigenschaft von $\|\_\|_n$}.
\end{regeln}

\begin{bemerkung}
  Für $n:\N_{-2}$, $A,B:\mU$ und jede Funktion $f:A\to B$ gibt es eine Funktion $\| f\|_n:\| A\|_n\to \| B\|_n$.
\end{bemerkung}
\begin{beweis}
  Universelle Eigenschaft anwenden auf den abhängigen Typ $\_\mapsto B$ und $x\mapsto |f(x)|:A\to \| B \|_n$, um ein Urbild $\| f \|_n:\| A \|_n\to \| B \|_n$ zu erhalten.
\end{beweis}

Erst später: Wie bereits angekündigt, ist $\isContr(A)$ stets eine Aussage, egal was $A$ ist.
Das stimmt auch allgemeiner für $\isNType{n}(A)$.

\begin{lemma}
  \label{lem:isprop-hlevel}
  Sei $A:\mU$.
  \begin{enumerate}
  \item Für jedes $B:A\to\mU$ haben wir
    \begin{mathpar}
      \left(\prod_{x:A}\isProp(B(x))\right)\to \isProp\left(\prod_{x:A}B(x)\right)
    \end{mathpar}
  \item Es gilt $\isProp(\isContr(A))$.
  \item Für jedes $n:\N_{-2}$ gilt $\isProp(\isNType{n}(A))$.
  \end{enumerate}
\end{lemma}
\begin{beweis}
  \begin{enumerate}
  \item Für $x,y:\prod_{x:A}B(x)$ müssen wir $x=y$ zeigen.
  \end{enumerate}
\end{beweis}

\subsection{Überlagerungen}
\subsection{Homotopiegruppen}
