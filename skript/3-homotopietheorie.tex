\subsection{n-Typen}
In \cref{def:kontr-aussage-menge} hatten wir bereits kontrahierbare Typen, Aussagen und Mengen definiert.
Die Mengen haben den alternativen Namen $0$-Typen und zu Aussagen, kann man auch $-1$-Typen und zu kontrahierbaren Typen $-2$-Typen sagen.
Wir werden in diesem Abschnitt $n$-Typen definieren, für $n=-2,-1,0,1,\dots$.
Dazu sei zunächst:
\begin{definition}
  Es sei $\N_{-2}$ der induktive Typ mit Konstruktoren $-2:\N_{-2}$ und $\sucHL:\N_{-2}\to\N_{-2}$.
  Für $\sucHL(n)$ schreiben wir $n+1$.
  Die Elemente von $\N_{-2}$ nennen wir auch \begriff{Homotopielevel}.
\end{definition}
Damit können wir $n$-Typen für jeden Homotopielevel $n$ definieren:
\begin{definition}
  Für jeden Typ $A$ sei der Typ $\isNType{n}$ wie folgt per Rekursion über $n:\N_{-2}$ definiert:
  \begin{align*}
    -2\mathrm{-Type}(A)&\colonequiv \isContr(A) \\
    (n+1)\mathrm{-Type}(A)&\colonequiv \prod_{x,y:A}n\mathrm{-Type}(x=_A y)
  \end{align*}
\end{definition}


\subsection{Höhere Induktive Typen}
\subsection{Überlagerungen}
\subsection{Homotopiegruppen}
