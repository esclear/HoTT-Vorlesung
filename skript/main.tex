% latexmk -pdflatex='xelatex %O %S' -pvc -pdf main.tex
\documentclass{hott}

\setenumerate[1]{label=(\alph*),topsep=0pt}
\setenumerate[2]{label=(\roman*),topsep=0pt}

\title{Vorlesungsskript zur Homotopietypentheorie}
\author{Felix Cherubini}

\makeindex

\begin{document}

\maketitle

\tableofcontents
 \pagebreak
Dieses Skript entsteht als Begleitmaterial zur Vorlesung ``Homotopietypentheorie'' (HoTT), die ich im Sommersemester 2021 an der Universität Augsburg halte.
Zweck dieses Skripts ist es, den Zuhörern und mir selbst zur Erinnerung an die Vorlesungsinhalte zu dienen --
bei der Beschäftigung mit dem Thema ist es hilfreich in Lehrbücher zu schauen.
Das sogenannte \href{https://homotopytypetheory.org/book/}{``HoTT-Book''} ist sicher eine gute Quelle.

Wer Fehler findet und diese korrigieren oder darauf aufmerksam machen will, kann das auf der github-Seite dieses Skripts machen:
\href{https://github.com/felixwellen/HoTT-Vorlesung}{https://github.com/felixwellen/HoTT-Vorlesung}.
Dort gibt es auch die Möglichkeit, sogenannte ``Issues'' anzulegen.
Hier könnten sie etwa darüber informieren, wenn sie eine Passage nicht verstehen oder einen Fehler gefunden haben.
Sie haben über sogenannte ``Pull requests'' die Möglichkeit, Fehler auch selbst zu korrigieren.

Die Homotopietypentheorie ist eine eigenständige Art Mathematik zu betreiben und basiert auf einer abhängigen Typentheorie.
Das bringt mit sich, dass wir zunächst Erlernen werden, wie man in Homotopietypentheorie Objekte konstruiert, Aussagen formuliert und Beweise führt.
Nach den Grundlagen werden wir uns in Richtung Homotopietheorie orientieren, einem Teilgebiet der Mathematik, in dem einige Vorzüge der Ho\-mo\-to\-pie\-ty\-pen\-theo\-rie zur Geltung kommen.

Im Folgenden werden wir nach und nach \begriff{Regeln} einführen (oder zumindest erwähnen), die schließlich zusammen eine Typentheorie ergeben.
Diese werden wir noch um das sogenannte \begriff{Univalenzaxiom} erweitern.

\section*{Regeln}
Wir werden zunächst nur Regeln kennenlernen, die Teil einer abhängigen Typentheorie sind.
Regeln einer (abhängigen) Typentheorie können etwa so aussehen:
\begin{mathpar}
\inferrule{\Gamma \yields f : A\to B \and \Gamma \yields t : A}{\Gamma \yields f(t) : B}{\mathrm{Beispiel}}
\end{mathpar}
Über dem waagrechten Strich stehen die Voraussetzungen und darunter, was aus den Voraussetzungen geschlossen werden darf.
Dabei kann man zum Beispiel ``$\Gamma \yields t : A$'' lesen als ``Im Kontext $\Gamma$ gibt es einen Term $t$ des Typs $A$''.
Wir werden nur anfangs mit Regeln arbeiten, um ein Verständnis für Typentheorie zu erlangen.
Irgendwann werden wir Regeln wie die obige wieder, wie in der Mathematik üblich, sprachlich formulieren,
etwa so
\begin{center}
  Für $f : A\to B$ und $t : A$ gibt es ein $f(t) : B$.
\end{center}
Wie wir später sehen werden, können diese Regeln zu Herleitungsbäumen kombiniert werden.
Dieses Kombinieren ist der vollständig formale Weg, Beweise in der Homotopietypentheorie zu führen.

Ein Kontext wie ``$\Gamma$'' darf man sich als Liste von Variablen zusammen mit ihren Typen vorstellen. Also etwa so:
\begin{mathpar}
  x_1 : A_1, \cdots ,x_n : A_n
\end{mathpar}
Dabei dürfen die Variablen nach ihrer Einführung verwendet werden.
So könnte etwa in der Konstruktion des Typs $A_2$ die Variable $x_1$ verwendet werden.
Dass der Kontext eine solche Liste ist, wird üblicherweise durch die strukturellen Regeln einer abhängigen Typentheorie festgelegt.
\begin{table}
  \centering
  \begin{tabular}{ll}
    Urteil                        & Bedeutung (evtl. im Kontext $\Gamma$) \\
    \hline
    $\Gamma\yields t : A$         & $t$ ist ein Term vom Typ $A$ \\
    $\Gamma\yields A$ Typ         & $A$ ist ein Typ \\
    $\Gamma\yields A\equiv B$     & $A$ und $B$ sind (urteils-)gleiche Typen \\
    $\Gamma\yields t\equiv s : A$ & $t$ und $s$ sind (urteils-)gleiche Terme des Typs $A$ \\
    $\Gamma$ Kontext              & $\Gamma$ ist ein Kontext
  \end{tabular}
  \caption{Urteile}
  \label{tab:urteile}
\end{table}

Ein Block der Form ``$\Gamma \yields t : A$''  ist ein spezielles \begriff{Urteil}.
Insgesamt gibt es die Urteile in \cref{tab:urteile}.

Eine Regel ist im Allgemeinen Fall nun von dieser Form:
\begin{mathpar}
\inferrule{\mathcal U_1 \and \dots \and \mathcal U_n}{\mathcal U_0}{\mathrm{Name}}
\end{mathpar}
Für $n\in\mathbb N=\{0,1,\dots\}$ und Urteile $\mathcal U_0,\dots,\mathcal U_n$.

\subsection*{Strukturregeln}

Neben den Regeln für einzelne Typen, die wir in den folgenden Abschnitten kennenlernen, gibt es sogenannte \begriff{strukturelle Regeln} oder \begriff{Strukturregeln}.
Diese legen fest, wie Grundsätzliches funktioniert, etwa wie Kontexte geformt werden dürfen und was man mit Gleichheitsurteilen anfangen darf.
Hier ist ein Beispiel, die sogenannte ``Weakening''-Regel oder \begriff{Abschwächungsregel}:
\begin{mathpar}
\inferrule{\Gamma \yields \mathcal U \and \Gamma \yields A \text{ Typ}}{\Gamma, x : A \yields \mathcal U}{\Weak}
\end{mathpar}
Das gilt für alles was man durch andere Regeln an der Stelle von $\mathcal U$ bekommen könnte.
Die folgende \begriff{Variablenregel} erlaubt es Variablen aus dem Kontext zu benutzen:
\begin{mathpar}
  \inferrule{\Gamma, x:A\text{ Kontext}}{\Gamma, x:A\yields x:A }{\mathrm{Var}}
\end{mathpar}
Die Regeln für die Urteilsgleichheit legen fest, dass diese eine Äquivalenzrelation auf Termen und Typen ist.
Außerdem gibt es Strukturregeln und sogenannte \begriff{Kongruenzregeln}, die es letztendlich erlauben, Terme und Typen durch Urteilsgleiches zu ersetzen.
Wir erlauben es uns einfach zu Ersetzen und führen diese Regeln nicht aus.

Weiter sind Typentheorien typischerweise so aufgebaut, dass wenn es möglich ist $t:A$ herzuleiten, es auch immer möglich ist, $A\text{ Typ}$ herzuleiten,
in welchem Fall es wieder möglich ist, herzuleiten, dass der Kontext in dem das gilt ein Kontext ist.
Allgemeiner erlauben wir uns stets notwendige Voraussetzungen für vorliegende Urteile zu verwenden.
Also etwa der Schluss von einem Urteil, in dem der Typ ``$A\to B$'' vorkommt, zum Urteil ``$A$ Typ''.
Wir wollen diese Tatsachen frei verwenden und wie Regeln einsetzen, die wir zusammenfassend mit ``$\mathrm{Str}$'' bezeichnen.

Diese Regeln führen wir hier nicht auf, eine gute Quelle, um die genauen Regeln im Bedarfsfall anzuschauen,
ist das (noch nicht erschienene) Lehrbuch von Egbert Rijke oder Anhang A.2 des HoTT-Books.


\section{Abhängige Typentheorie}
\input{1-abhängige-typentheorie.tex}

\section{Univalenz}
\subsection{Äquivalenzen}
\label{sub:aequivalenzen}

In diesem Abschnitt werden wir drei verschiedene Definitionen von Äquivalenzen einführen und zeigen,
dass es sich um den selben Begriff handelt. Einen haben wir bereits eingeführt - Funktionen mit kontrahierbaren Fasern heißen Äquivalenz.
Von einem Äquivalenzbegriff erwarten wir, dass es sich beim entsprechenden Typ $\isEquiv(f)$ um eine Aussage handelt.
Das hat zur Folge, dass wir es als eine Eigenschaft von Funktionen anssehen können eine Äquivalenz zu sein.
Damit ist dann auch der Typ der Äquivalenzen zwischen $A$ und $B$ eine Untertyp von $A\to B$, d.h. die Vergiß-Abbildung
\begin{mathpar}
  \pi_1:\left(\sum_{f:A\to B}\isEquiv(f)\right)\to (A\to B)
\end{mathpar}
ist injektiv.

Zur Vorbereitung beschäftigen wir uns noch etwas mit Homotopien.
\begin{bemerkung}
  Seien $A,B:\mU$ und $f,g:A\to B$.
  \begin{enumerate}
  \item Sei $H:f\sim g$. Für $\varphi:A'\to A$ gibt es eine Homotopie $H_{\varphi(\_)}:f\circ \varphi\sim g\circ \varphi$ und für $\psi:B\to B'$ gibt es eine Homotopie $\psi(H):\psi\circ f\sim \psi\circ g$. Diese Operationen nennt man \begriff{whiskering}.
  \item Sei $H:f\sim g$. Homotopien sind natürlich in folgendem Sinn: Für $p:x=_A y$ gilt:
    \begin{mathpar}
      H_x\kon g(p) = f(p)\kon H_{y}
    \end{mathpar}
  \item Für eine Homotopie $H:f\sim\id$ gilt $f(H_x)=H_{f(x)}$.
  \end{enumerate}
\end{bemerkung}


Wir beginnen mit einer scheinbar unbedeutenden Variation des Begriffs Quasi-Inverse und legen dabei mehr Notation für Äquivalenzen fest:

\begin{definition}
  Seien $A,B:\mU$ und $f:A\to B$.
  \begin{enumerate}
  \item $f$ hat eine \begriff{Linksinverse}, wenn es ein $g:B\to A$ gibt und $g\circ f\sim\id_A$.
  \item $f$ hat eine \begriff{Rechtsinverse} oder einen \begriff{Schnitt}, wenn es ein $h:B\to A$ gibt und $f\circ h\sim \id_B$.
  \item $f$ ist eine \begriff{Äquivalenz} oder eine Funktion mit \begriff{Links- und Rechtsinversen}, wenn
    \begin{mathpar}
      \isEquiv(f)\colonequiv\mathrm{LRInv}(f)\colonequiv  \left(\sum_{g:B\to A}g\circ f\sim\id_A\right)\times\left(\sum_{h:B\to A}f\circ h\sim\id_B\right)
    \end{mathpar}
  \item Wenn es eine Äquivanz zwischen $A$ und $B$ gibt, dann sagen wir $A$ ist \begriff{äquivalent} zu $B$.
  \item Für den Typ der Äquivalenzen zwischen $A$ und $B$ schreiben wir:
    \begin{mathpar}
      A\simeq B\colonequiv\sum_{f:A\to B}\isEquiv(f)
    \end{mathpar}
  \end{enumerate}
\end{definition}

Es wird sich herausstellen, dass $\mathrm{LRInv}(f)$ und $\mathrm{qinv}(f)$ für manche Funktionen keine äquivalenten Typen sind.
Insbesondere werden wir noch sehen, dass $\qinv(f)$ immer eine Aussage ist, $\mathrm{LRInv}(f)$ aber keine Aussage mehr sein muss, wenn $A$ und $B$ komplizierte Gleichheitstypen haben.
Die beiden Begriffe sind aber logisch äquivalent, d.h. es gibt für jedes $f:A\to B$ Abbildungen $\mathrm{LRInv}(f)\to \qinv(f)$ und $\qinv(f)\to\mathrm{LRInv}(f)$.
Zunächst halten wir die Vokabel ``logisch äquivalent'' fest:

\begin{definition}
  Zwei Typen $A$ und $B$ heißen \begriff{logisch äquivalent}, wenn es Funktionen $f:A\to B$ und $g:B\to A$ gibt.
\end{definition}

Bei Aussagen reicht die logische Äquivalenz bereits, damit die Typen äquivalent sind.
Damit werden die verschiedenen Varianten des Typs ``$\isEquiv(f)$'', die wir in diesem Abschnitt kennenlernen, äquivalente Typen sein,
wenn wir logische Äquivalenz gezeigt haben und dass es sich jeweils um Aussagen handelt. Letzteres werden wir allerdings noch Aufschieben.

\begin{bemerkung}
  \label{bem:lrinv-qinv}
  Seien $A,B:\mU$ und $f:A\to B$. Die Typen $\mathrm{LRInv}(f)$ und $\qinv(f)$ sind logisch äquivalent.
\end{bemerkung}
\begin{beweis}
  Nehmen wir zünachst an, es gibt eine Quasi-Inverse $g:B\to A$ mit $H:g\circ f\sim\id_A$ und $K:f\circ g\sim\id_B$. Dann ist
  \begin{mathpar}
    ((g,H),(g,K)):\mathrm{LRInv}(f)
  \end{mathpar}
  Seien nun andererseits $g,h:B\to A$ links- und rechtsinvers zu $f$. Wir zeigen, dass $g$ und $h$ homotop sind. Damit können wir dann zeigen, dass $g$ auch rechtsinvers ist.
  Durch whiskering erhalten wir:
  \begin{mathpar}
    g\sim g\circ (f\circ h) \sim (g\circ f)\circ h  \sim h
  \end{mathpar}
  Also ist $g$ rechtsinvers:
  \begin{mathpar}
    f\circ g\sim f\circ h\sim \id
  \end{mathpar}
\end{beweis}

Wenn wir also zu einer Funktion $f:A\to B$ eine beidseitige Inverse konstruieren, dann ist $f$ eine Äquivalenz (im Sinn einer Funktion mit Links- und Rechtsinversen).
Äquivalenzen haben die Operationen, die wir bereits von Gleichheiten kennen:

\begin{bemerkung}
  \begin{enumerate}
  \item Die Identität ist eine Äquivalenz.
  \item Eine Inverse einer Äquivalenz ist eine Äquivalenz.
  \item Seien $A,B,C:\mU$. Wenn $f:A\to B$ und $g:B\to C$ Äquivalenzen sind, dann ist $g\circ f:A\to C$ eine Äquivalenz.
  \end{enumerate}
\end{bemerkung}
\begin{beweis}
  Wir zeigen die Aussagen für den Äquivalenzbegriff der Funktion mit Links- und Rechtsinversen.
  Mit \cref{thm:aequivalenzen} gilt damit das gleiche für die anderen Äquivalenzbegriffe.
  \begin{enumerate}
  \item Die Identität ist ihre eigene Links- und Rechtsinverse. Die nötigen Homotopien gelten schon als urteilsmäßige von Funktionen.
  \item Wenn $f:A\to B$ eine Linksinverse $g:B\to A$ und eine Rechtsinverse $h:B\to A$ hat, dann ist bekommen wir mit Bemerkung \labelcref{bem:lrinv-qinv} eine Quasi-Inverse $f^{-1}:B\to A$.
    Und $f^{-1}$ ist eine Äquivalenz mit Linksinverser $f$ und Rechtsinverser $f$.
  \item Auf Übungsblatt 4 wird das für Quasi-Inversen gezeigt. Zusammen mit Bemerkung \labelcref{bem:lrinv-qinv} gilt das also auch für Äquivalenzen.
  \end{enumerate}
\end{beweis}

\begin{beispiel}[Transport ist Äquivalenz]
  \label{bsp:transp-aequiv}
  Für $A:\mU$, $B:A\to\mU$, $x,y:A$ und jedes $p:x=y$ ist $\transp_B(p):B(x)\to B(y)$ eine Äquivalenz mit Inverser $\transp_B(p^{-1}):B(y)\to B(x)$.
\end{beispiel}

Wir wollen nun den Äquivalenzbegriff der Links- und Rechtsinvertierbarkeit mit dem der kontrahierbaren Fasern in Zusammenhang bringen.
Wir werden allerdings erstmal eine Implikation klären und für die zweite einen dritten Äquivalenzbegriff einführen.

\begin{bemerkung}
  \label{bem:isContr-lrinv}
  Seien $A,B:\mU$ und $f:A\to B$. Wenn alle Fasern von $f$ kontrahierbar sind, dann hat $f$ Links- und Rechtsinverse, es gibt also eine Funktion:
  \begin{mathpar}
    \left(\prod_{y:B}\isContr(f^{-1}(y))\right) \to \mathrm{LRInv}(f)
  \end{mathpar}
\end{bemerkung}

\begin{beweis}
  Sei $k:\prod_{y:B}\isContr(f^{-1}(y))$.
  Wir konstruieren eine beidseitige Inverse $g:B\to A$ indem wir $y:B$ auf das Kontraktionszentrum der Faser $f^{-1}(y)$ abbilden:
  Für $\pi_1(k_y)$ ist ein Element von $f^{-1}(y)$, besteht also aus $x:A$ und $p:f(x)=y$, also können wir festlegen
  \begin{mathpar}
    g(y)\colonequiv x
  \end{mathpar}
  Dann ist $f(g(y))=f(x)=y$, also $g$ Rechtsinverse von $f$. Für $x:A$ sei $x'\colonequiv g(f(x))$.
  Es müssen $x:A$ und $x':A$ beide in $f^{-1}(f(x))$ liegen und da diese Faser kontrahierbar ist, gibt es eine Gleichheit $q:x=x'=g(f(x))$.
  Genauer haben wir durch die Kontrahierbarkeit eine Gleichheit $q':(x,\refl_{f(x)})=(x',p')$ in der Faser $f^{-1}(f(x))$.
  Diese Gleichheit können wir mit $\pi_1:f^{-1}(f(x))\to A$ abbilden, also $q\colonequiv \pi_1(q')$ setzen.
\end{beweis}

Die Umkehrung dieser Aussage ist komplizierter zu zeigen und wir werden zunächst zeigen, dass sich Links- Rechtsinverse in eine sogenannte kohärente Inverse überstzen lassen.
Kohärente Inverse $g:B\to A$ einer Funktion $f:A\to B$ haben eine Kompatiblität zwischen den beiden Homotopien $f\circ g\sim \id$ und $g\circ f\sim \id$.
Für eine Funktion $f:A\to B$ mit Inverser $g:B\to A$ besagt diese Kohärenz, dass die beiden Möglichkeiten eine Homotopie $f\circ g\circ f\sim f$ zu konstruieren punktweise gleich sind.

\begin{definition}
  Seien $A,B:\mU$ und $f:A\to B$.
  Eine Funktion $g:B\to A$ heißt \begriff{kohärente Inverse} von $f$, wenn $f$ und $g$ es Homotopien $H:g\circ f\sim \id$ und $K:f\circ g \sim \id$ gibt und
  \begin{mathpar}
    \mathrm{koh} : \prod_{x:A}f(H_x)=K_{f(x)}
  \end{mathpar}
  Wir schreiben $\mathrm{CohInv}(f)$ für den Typ der \begriff{kohärenten Inversen} von $f$.
\end{definition}

Wir wollen direkt einsehen, dass wir von dieser Sorte Äquivalenz die Kontrahierbarkeit der Fasern beweisen können:

\begin{bemerkung}
  \label{bem:qinv-equiv}
  Seien $A,B:\mU$ und $f:A\to B$. Wenn $f$ eine kohärente Inverse hat, dann sind die Fasern von $f$ kontrahierbar.
\end{bemerkung}
Vor dem Beweis noch ein Hilfslemma, das wir auch sonst noch wiederverwenden können:
\begin{lemma}
  \label{lem:gleichheit-in-faser}
  Seien $A,B$ Typen, $f:A\to B$ und $y:B$. Für $(x,q),(x',q'):f^{-1}(y)$ gibt es eine Funktion:
  \begin{mathpar}
    \prod_{p:x=_A x'}f(p)^{-1}\kon q=q'\to (x,q)=(x',q')
  \end{mathpar}
\end{lemma}
\begin{beweis}[von \cref{lem:gleichheit-in-faser}]
  Induktion über $p$ und $\sum_=$.
\end{beweis}

\begin{beweis}[von \labelcref{bem:qinv-equiv}]
  Habe also $f:A\to B$ eine kohärente Inverse, gebe es also $g:B\to A$, $H:g\circ f\sim \id$, $K:f\circ g \sim \id$ und $\mathrm{koh} : \prod_{x:A}f(H_x)=K_{f(x)}$.
  Wir müssen nun für jede Faser von $f$ eine Kontraktion angeben.
  Als Kontraktionszentrum für die Faser über $y:B$ wählen wir
  \begin{mathpar}
    k_y\colonequiv (g(y), K_y) : f^{-1}(y)
  \end{mathpar}
  Seien nun also $(x,q),(x',q'):f^{-1}(y)$. Damit haben wir auch $q\kon q'^{-1}:f(x)=f(x')$ und damit:
  \begin{mathpar}
    H_x^{-1}\kon g(q\kon q'^{-1}) \kon H_x':x=x'
  \end{mathpar}
  Darauf wenden wir jetzt $f$ an:
  \begin{mathpar}
    f(H_x^{-1}\kon g(q\kon q'^{-1}) \kon H_x'):f(x)=f(x')
  \end{mathpar}
  und berechnen mit natürlichkeit von Homotopien, der Kohärenz und Gruppoidgesetzen:
  \begin{align*}
    f(H_x^{-1}\kon g(q\kon q'^{-1}) \kon H_x')&=f(H_x^{-1})\kon f(g(q\kon q'^{-1})) \kon f(H_x') \\
                                              &=f(H_x^{-1})\kon K_{f(x)}\kon q\kon q'^{-1}\kon K_{f(x')}^{-1} \kon f(H_x') \\
                                              &=q\kon q'^{-1}
  \end{align*}
  Mit dem Lemma haben wir also die gewünschte Gleichheit.
\end{beweis}

\begin{bemerkung}
  Seien $A,B:\mU$ und $f:A\to B$ habe eine beidseitige Inverse $g:B\to A$.
  Dann ist $g$ auch eine kohärente Inverse von $f$.
\end{bemerkung}
\begin{beweis}
  Seien $g,h:B\to A$ und $H:g\circ f\sim\id$, $K:f\circ g\sim \id$. Setzte für $y:B$:
  \begin{mathpar}
    K'\colonequiv K_{f(g(y))}^{-1}\kon f(H_{g(y)})\kon K_{y}
  \end{mathpar}
  und rechne nach dass für $x:A$ gilt: $f(H_x)=K'_{f(x)}$.
\end{beweis}

\begin{theorem}
  \label{thm:aequivalenzen}
  Seien $A,B:\mU$ und $f:A\to B$, dann sind die folgenden Äquivalenzbegriffe logisch äquivalent:
  \begin{enumerate}
  \item Alle Fasern von $f$ sind kontrahierbar:
    \begin{mathpar}
      \isEquiv(f)\colonequiv\prod_{y:B}\isContr(f^{-1}(y))
    \end{mathpar}
  \item $f$ hat eine Linksinverse und eine Rechtsinverse:
    \begin{mathpar}
      \isEquiv(f)\colonequiv\mathrm{LRInv}\equiv  \left(\sum_{g:B\to A}g\circ f\sim\id_A\right)\times\left(\sum_{h:B\to A}f\circ h\sim\id_B\right)
    \end{mathpar}
  \item $f$ hat eine kohärente Inverse:
    \begin{mathpar}
      \isEquiv(f)\colonequiv\mathrm{CohInv}(f)\colonequiv \sum_{g:B\to A} \sum_{H:g\circ f\sim\id}\sum_{K:f\circ g\sim\id} \prod_{x:A}f(H_x)=K_{f(x)}
    \end{mathpar}
  \end{enumerate}
  Wir schreiben für alle drei Begriffe $\isEquiv(f)$, weil sich noch rausstellen wird, dass alle diese Typen Aussagen sind und damit alle drei Typen äquivalent sind.
\end{theorem}
\begin{beweis}
  Die Bemerkungen dieses Abschnitts beweisen per Ringschluss die logische Äquivalenz.
\end{beweis}

\subsection{Univalenz}
In diesem Abschnitt werden wir uns mit den Universen und ihren Gleichheitstypen beschäftigen.
Eine historisch wichtige Idee von Vladimir Voevodsky ist das \begriff{Univalenzaxiom},
das die Gleichheitstypen im Universum mit den Äquivalenzen von Typen identifiziert.
Dieses Axiom werden wir kennenlernen und Konsequenzen daraus ziehen.
Davor wollen wir aber noch allgemeine Konsequenzen aus der Existenz von Universen ziehen.

Mit Universen sind wir in der Lage Ungleichheit von Elementen eines Typs zu beweisen.
Diese definieren wir zunächst als Negation der Gleichheit:
\begin{definition}
  Seien $A:\mU$ und $x,y:A$. Dann ist $x$ \begriff{ungleich} $y$, wenn
  \begin{mathpar}
    x\not= y\colonequiv (x=y\to \leer)
  \end{mathpar}
\end{definition}
Man beachte, dass es sich bei diesem Typ stets um eine Aussage handelt.
Es sind also in $x\not= y$ keine so interessanten Dinge zu finden, wie im Gleichheitstyp und wir werden uns auch wenig mit Ungleichheiten beschäftigen.
\begin{bemerkung}
  Es gilt $1_\zwei \not= 0_\zwei$.
\end{bemerkung}
\begin{beweis}
  Per Rekursion können wir uns den folgenden abhängigen Typen definieren:
  \begin{align*}
    B&:\zwei\to\mU \\
    B(0_\zwei)&\colonequiv \leer \\
    B(1_\zwei)&\colonequiv \eins
  \end{align*}
  Um die Ungleichheit zu zeigen, dürfen wir $p:1_\zwei = 0_\zwei$ annehmen und müssen ein Element in $\leer$ konstruieren.
  Sei also $p:1_\zwei = 0_\zwei$. Damit haben wir auch eine Abbildung:
  \begin{mathpar}
    \transp_B(p):B(1_\zwei)\to B(0_\zwei)
  \end{mathpar}
  Und wegen $\ast:B(1_\zwei)\equiv \eins$ haben wir auch ein Element, das wir in diese Abbildung einsetzen können.
  Es ist also $\transp_B(p)(\ast):B(0_\zwei)\equiv \leer$.
\end{beweis}
Dieser Trick lässt sich auf alle Elemente induktiver Typen übertragen, für die wir ``verschiedene'' Werte vorgeben können.
Für einzelne natürliche Zahlen etwa:
\begin{bemerkung}
  Es gilt $0_{\N}\not= 1_{\N}\colonequiv \sucN(0_{\N})$.
\end{bemerkung}
\begin{beweis}
  Sei
  \begin{align*}
    B&:\N\to \mU \\
    B(0_{\N})&\colonequiv \leer \\
    B(1_{\N})&\colonequiv \eins
  \end{align*}
  womit für $p:0_{\N}=1_{\N}$ ein Element $\transp_B(p^{-1})(\ast):\leer$ gegeben ist.
\end{beweis}
Typischerweise interessiert man sich für eine vollständige Charakterisierung der Gleichheitstypen eines Typs.
Für die natürlichen Zahlen wäre das ein in $n,k:\N$ abhängiger Typ $B(n,k)$ sodass $B(n,k)\simeq (n=_\N k)$.
Das wird unser Ziel in \cref{sec:zahlen-gleichheit} sein.

Ohne Weiteres können wir bereits für $A,B:\mU_i$ den Typ $A=_{\mU_i}B$ der Gleichheiten zwischen $A$ und $B$ formen.
Dieser liegt allerdings im nächsthöheren Universum $\mU_{i+1}$, da wir in der Formierungsregel für Gleichheitstypen ``X ist ein Typ'' durch ``$X : \mU_i$'' ersetzen und dafür nur ``$\mU_i : \mU_{i+1}$'' in Frage kommt:
\begin{mathpar}
  \inferrule{\Gamma\yields \mU_i:\mU_{i+1}\and \Gamma\yields A,B:\mU_i}{\Gamma\yields A=_{\mU_i}B:\mU_{i+1}}{=\mathrm{F}}
\end{mathpar}

Damit lässt sich zeigen, dass gleiche Typen äquivalent sind.
\begin{bemerkung}
  Für $A,B:\mU$ gibt es eine Funktion
  \begin{mathpar}
    \transp_{\id_\mU}:A=_{\mU} B\to (A\simeq B)
  \end{mathpar}
  den \begriff{Universentransport}.
\end{bemerkung}
\begin{beweis}
  Seien $A,B:\mU_i$. Dann gibt es einen abhängigen Typ $X:\mU_i\to\mU_{i+1}$ und damit
  \begin{mathpar}
    \transp_{\id_\mU}\colonequiv\transp_{X}:A=_{\mU_i}B\to A\simeq B
  \end{mathpar}
  Nach Beispiel \labelcref{bsp:transp-aequiv} sind Transporte immer Äquivalenzen.
\end{beweis}

Es ist also möglich, aus einer Gleichheit von Typen eine Äquivalenz zu kontruieren.
Die Umkehrung, aus einer Äquivalenz auch eine Gleichheit konstruieren zu können, nennt man das Univalenzaxiom.
\begin{axiom}[Univalenzaxiom]
  Wir nehmen von nun an an, dass das \begriff{Univalenzaxiom} gilt: Für $A,B:\mU$ ist die Funktion
  \begin{mathpar}
    \transp_{\id_\mU}:A=_{\mU}B\to A\simeq B
  \end{mathpar}
  eine Äquivalenz ist. Die Inverse von $\transp_{\id_\mU}$ bezeichnen wir mit $\ua$\index{$\ua$}:
  \begin{mathpar}
    \ua:A\simeq B\to A=_{\mU}B
  \end{mathpar}
  Wir nehmen an, dass das Univalenzaxiom für jedes Universum gilt.
\end{axiom}

Mit Univalenz ergibt sich, dass es tatsächlich verschiedene Gleichheiten zwischen Elementen geben kann.
Um dafür ein Beispiel zu bekommen, wollen wir zunächst einen Typ von Äquivalenzen weit genug dafür verstehen:
\begin{bemerkung}
  Es gilt $\zwei\simeq (\zwei\simeq \zwei)$ (mit Funktionsextensionalität).
\end{bemerkung}
\begin{beweis}
  Zunächst geben wir der Äquivalenz, die die beiden Elemente von $\zwei$ vertauscht den Namen $s$:
  \begin{align*}
    s:&\zwei\simeq\zwei \\
    s(0_\zwei)&\colonequiv 1_\zwei \\
    s(1_\zwei)&\colonequiv 0_\zwei
  \end{align*}
  Damit können wir Elemente von $\zwei$ auf Äquivalenzen abbilden:
  \begin{align*}
    \varphi:&\zwei\to (\zwei\simeq\zwei) \\
    \varphi(0_\zwei)&\colonequiv \id_\zwei \\
    \varphi(1_\zwei)&\colonequiv s
  \end{align*}
  Für die Umkehrabbildung wählen wir:
  \begin{align*}
    \psi:&(\zwei\simeq\zwei)\to\zwei \\
    \psi&\colonequiv f\mapsto f(0_\zwei) \\
  \end{align*}
  Es gilt per $\ind{\zwei}$ durch die Urteilsgleichheiten $\psi\varphi(0_\zwei)\equiv 0_\zwei$ und $\psi\varphi(1_\zwei)\equiv 1_\zwei$ bereits $\psi\circ\varphi\sim \id$.
  Also ist noch zu zeigen:
  \begin{mathpar}
    \prod_{f:\zwei\simeq \zwei} \varphi(\psi(f))=f
  \end{mathpar}
  Oder nach auswerten von $\psi$ und mit Funktionsextensionalität:
  \begin{mathpar}
    \prod_{f:\zwei\simeq \zwei}\prod_{x:\zwei} \varphi(f(0_\zwei))(x)=f(x)
  \end{mathpar}
  Aus Beispiel \labelcref{bsp:einheit-kontrahierbar} wissen wir, dass:
  \begin{mathpar}
    t:\prod_{x:\zwei}(x=0_\zwei)\amalg(x=1_\zwei)
  \end{mathpar}
  womit wir die Fallunterscheidung ``$f(0)$ ist $1_\zwei$ oder $0_\zwei$'' umsetzen können:
  \begin{mathpar}
    f\mapsto x\mapsto t_{f(0_\zwei)} : \prod_{f:\zwei\simeq\zwei}\prod_{x:\zwei}(f(0_\zwei)=0_\zwei)\amalg(f(0_\zwei)=1_\zwei)
  \end{mathpar}
  Die beiden Fälle können wir abarbeiten, indem wir Funktionen auf den einzelnen Faktoren, in unseren Zieltyp angeben, wobei wir gleichzeitig mit $\ind{\zwei}$ eine Fallunterscheidung über $x$ machen.
  Insgesamt müssen wir also noch konstruieren:
  \begin{align*}
    F_{00}& : f(0_\zwei)=0_\zwei \to \varphi(f(0_\zwei))(0_\zwei)=f(0_\zwei) \\
    F_{01}& : f(0_\zwei)=0_\zwei \to \varphi(f(0_\zwei))(1_\zwei)=f(1_\zwei) \\
    F_{10}& : f(0_\zwei)=1_\zwei \to  \varphi(f(0_\zwei))(0_\zwei)=f(0_\zwei) \\
    F_{11}& : f(0_\zwei)=1_\zwei \to  \varphi(f(0_\zwei))(1_\zwei)=f(1_\zwei)
  \end{align*}
  Denn damit haben wir dann:
  \begin{mathpar}
    f\mapsto \ind{\zwei}(\dots,\ind{\amalg}(\dots,F_{00},F_{10})(t_{f(0_\zwei)}),\ind{\amalg}(\dots,F_{01},F_{11})(t_{f(0_\zwei)})): \prod_{f:\zwei\simeq \zwei}\prod_{x:\zwei} \varphi(f(0_\zwei))(x)=f(x)
  \end{mathpar}
  Die Funktionen $F_{00}$ und $F_{10}$ können mit Gruppoidgestzen konstruiert werden.
  Für die übrigen müssen wir verwenden, dass $0_\zwei\not=1_\zwei$.
  Etwa für $F_{01}$ nehmen wir also $p:f(0_\zwei)=0_\zwei$ an und verwenden wieder die Fallunterscheidung für die Gleichheit in $\zwei$:
  \begin{mathpar}
    (f(1_\zwei)=0_\zwei)\amalg(f(1_\zwei)=1_\zwei)
  \end{mathpar}
  Es geht also wieder darum, zwei Funktionen zu konstruieren. Für $f(1_\zwei)=1_\zwei)$ geht das wieder mit den üblichen Methoden, für die andere Gleichheit, können wir zusammen mit $p$ zeigen:
  \begin{mathpar}
    f(1_\zwei)=f(0_\zwei)
  \end{mathpar}
  und darauf die Inverse von $f$ anwenden um $1_\zwei=0_\zwei$ zu erhalten. Nun ist $0_\zwei\not=1_\zwei$ eine Abbildung von $1_\zwei=0_\zwei$ nach $\emptyset$.
  Da es von $\emptyset$ eine Abbildung in jeden beliebigen Typ gibt, haben wir insgesamt auch eine Abbildung von $1_\zwei=0_\zwei$ in die zu zeigende Behauptung.
  Das beweist den Fall $F_{01}$ und $F_{11}$ lässt sich analog zeigen.
\end{beweis}

\begin{beispiel}
  Mit $\zwei\simeq(\zwei\simeq\zwei)$ und per Univalenz gilt:
  \begin{mathpar}
    f:\zwei\simeq (\zwei=_{\mU}\zwei)
  \end{mathpar}
  Also: $f(0_\zwei)\not=f(1_\zwei)$.
\end{beispiel}

\begin{bemerkung}
  Seien $A,B,C:\mU$, $f:A\simeq B$ und $g:B\simeq C$. Es gelten:
  \begin{enumerate}
  \item $\ua(\id_A)=\refl_A$
  \item $\ua(f\circ g)=\ua(f)\kon \ua(g)$
  \item $\ua(f^{-1})=\ua(f)^{-1}$
  \end{enumerate}
\end{bemerkung}
\begin{beweis}
  \begin{enumerate}
  \item Es gilt $\transp_{\id_\mU}(\refl_A)=\id_A$ und damit $\ua(\id_A)=\ua(\transp_{\id_\mU}(\refl_A))=\refl_A$.
  \item Seien $p\colonequiv\ua(f)$ und $q\colonequiv\ua(g)$. Dann gilt $\transp_{\id_\mU}(p)=f$ und $\transp_{\id_\mU}(q)=g$ und damit:
    \begin{mathpar}
      \ua(f\circ g)=\ua(\transp_{\id_\mU}(p)\circ \transp_{\id_\mU}(q))=\ua(\transp_{\id_\mU}(p\kon q))=p\kon q=\ua(f)\kon\ua(g)
    \end{mathpar}
  \item Sei $p\colonequiv\ua(f)$, dann gilt:
    \begin{mathpar}
      \ua(f^{-1})=\ua(\transp_{\id_\mU}(p)^{-1})=\ua(\transp_{\id_\mU}(p^{-1}))=p^{-1}=\ua(f)^{-1}
    \end{mathpar}
  \end{enumerate}
\end{beweis}

Da wir nun wissen, dass Gleichheiten und Äquivalenzen dasselbe sind, können wir auch die Gleichheitsinduktion auf Äquivalenzen übertragen:
\begin{lemma}[Äquivalenzinduktion]
  Für $C:\prod_{A,B:\mU}A\simeq B\to \mU$ gibt es:
  \begin{mathpar}
    \ind{\simeq}:\left(\prod_{A:\mU}C(A,A,\id_A)\right)\to \prod_{A,B:\mU}\prod_{f:A\simeq B}C(A,B,f)
  \end{mathpar}
\end{lemma}
\begin{beweis}
  Durch Gleichheitsinduktion ergibt sich:
  \begin{mathpar}
    \ind{=}:\left(\prod_{A:\mU}C(A,A,\transp_{\id_\mU}(\refl_A))\right)\to\prod_{A,B:\mU}\prod_{p:A=_{\mU}B}C(A,B,\transp_{\id_\mU}(p))
  \end{mathpar}
\end{beweis}

Wir importieren die folgende Aussage (findet man im HoTT-Book am Ende von Kapitel 4): 
\begin{fakt}
  Univalenz impliziert Funktionsextensionalität.
\end{fakt}
Im nächsten Kapitel ergibt sich ein einfacher Beweis einer stärkeren Version von Funktionsextensionalität.
Wir verzichten daher im folgenden darauf, auf Verwendung von Funktionsextensionalität hinzuweisen.

\subsection{Faserungen und Gleichheitssätze}
Zunächst halten wir fest, dass abhängige Summen und abhängige Produkte mit äquivalenten Eingangsdaten äquivalente Typen produzieren:
\begin{bemerkung}
  \label{bem:ap-univalenz-sigma-prod}
  Seien $A:\mU$ und $B,B':A\to\mU$. Wenn $B$ und $B'$ punktweise äquivalent sind, also gilt $\prod_{x:A}B(x)\simeq B'(x)$, dann gilt auch:
  \begin{mathpar}
    \left(\sum_{x:A}B(x)\right)\simeq \left(\sum_{x:A}B'(x)\right)\text{ und }\left(\prod_{x:A}B(x)\right)\simeq \left(\prod_{x:A}B'(x)\right)
  \end{mathpar}
\end{bemerkung}
\begin{beweis}
  Aus der punktweisen Äquivalenz $\prod_{x:A}B(x)\simeq B'(x)$ machen wir durch punktweises Anwenden von $\ua$ eine punktweise Gleichheit $\prod_{x:A}B(x)=B'(x)$ und daraus mit Funktionsextensionalität eine Gleichheit $p:B=B'$.
  Dann ist aber auch
  \begin{mathpar}
    \mathrm{ap}(\prod,p):\left(\prod_{x:A}B(x)\right)=\left(\prod_{x:A}B'(x)\right)
  \end{mathpar}
  und damit $\simeq_=(\mathrm{ap}(\prod,p)):\left(\prod_{x:A}B(x)\right)\simeq\left(\prod_{x:A}B'(x)\right)$.
  Mit dem gleichen Argument können wir die entsprechende Aussage für abhängige Summen zeigen.
\end{beweis}
Mit Univalenz lässt sich auch die Gleichheit von abhängigen Typen konkretisieren:
\begin{bemerkung}
  Für $A:\mU$ und $B,B':A\to \mU$ gibt es eine Äquivalenz:
  \begin{mathpar}
    (B=B') \simeq \left(\prod_{x:A}B(x)\simeq B'(x)\right)
  \end{mathpar}
\end{bemerkung}
\begin{beweis}
  Mit Univalenz, Bemerkung \labelcref{bem:ap-univalenz-sigma-prod} und Funktionsextensionalität:
  \begin{mathpar}
    \left(\prod_{x:A}B(x)\simeq B'(x)\right)\simeq \left(\prod_{x:A}B(x)= B'(x)\right)\simeq (B=B')
  \end{mathpar}
\end{beweis}

\begin{definition}
  Sei $A:\mU$ und $B,B':A\to\mU$ zwei abhängige Typen.
  \begin{enumerate}
  \item Eine \begriff{faserweise Abbildung} ist ein Term
    \begin{mathpar}
      f:\prod_{x:A}\left(B(x)\to B'(x)\right)
    \end{mathpar}
  \item Für eine faserweise Abbildung gibt es eine induzierte Abbildung
    \begin{mathpar}
      \sum f:\left(\sum_{x:A}B(x)\right)\to \left(\sum_{x:A} B'(x)\right)
    \end{mathpar}
    gegeben durch $\sum f(x,b_x)\colonequiv (x,f_x(b_x))$.
  \end{enumerate}
\end{definition}

\begin{lemma}
  \label{lem:faser-equiv}
  Seien $A:\mU$ und $B:A\to\mU$. Dann gilt
  \begin{mathpar}
     \prod_{x:A} \pi_1^{-1}(x)\simeq B(x)
  \end{mathpar}
\end{lemma}
\begin{beweis}
  
\end{beweis} Später.

Mit Univalenz können wir folgern, dass für jeden Typ $A$ die abhängigen Typen auf $A$ dasselbe sind, wie Funktionen nach $A$.
\begin{theorem}[Objektklassifikation]
  \label{lem:objektklassifizierer}
  Sei $B:\mU$. Dann gilt:
  \begin{mathpar}
    \left(\sum_{A:\mU}(A\to B)\right)\simeq (B\to \mU)
  \end{mathpar}
\end{theorem}

\begin{beweis}
  Einen abhängigen Typ $B:A\to\mU$ bilden wir auf die Funktion
  \begin{mathpar}
    \pi_1:\left(\sum_{x:A}B(x)\right)\to A
  \end{mathpar}
  ab. Und eine Funktion $f:B\to A$ bilden wir auf den abhängigen Typ ihrer Fasern ab:
  \begin{mathpar}
    \mathrm{fib}_f\equiv \left((x:A)\mapsto f^{-1}(x)\right):A\to \mU
  \end{mathpar}
  Nach \cref{lem:faser-equiv} ist also bereits diese Konstruktion linksinvers zur vorangegangenen.
  Es ist also noch zu zeigen, dass die Fasern einer Abbildung sich zur Domäne aufsummieren, bzw. dass für alle
  $f:A\to B$ gilt:
  \begin{mathpar}
    e:\left(\sum_{y:B}(f^{-1}(y))\right) \simeq A
  \end{mathpar}
  und $f\circ e=\pi_1$ gilt. Wir können mit \cref{lem:pfade-kontrahierbar} berechnen:
  \begin{align*}
    \left(\sum_{y:B}(f^{-1}(y))\right) &\simeq \sum_{y:B}\sum_{x:A}f(x)=y \\
                                       &\simeq \sum_{x:A}\sum_{y:B}f(x)=y\\
                                       &\simeq \sum_{x:A}\eins\\
                                       &\simeq A
  \end{align*}
\end{beweis}

\begin{definition}
  Seien $A,A',B:\mU$ und $f:A\to B$, $f':A'\to B$. Für $g:A\to A'$ zusammen mit einer Homotopie $H:f\sim f'\circ g$ gibt es eine durch $g$ induzierte Abbildung auf den Fasern:
  \begin{mathpar}
    \mathrm{fib}(g):\prod_{y:B}f^{-1}(y)\to f^{'-1}(y)
  \end{mathpar}
  wobei $\mathrm{fib}(g)(x,p)\colonequiv (g(x), H_x^{-1}\kon p)$.
\end{definition}

\begin{lemma}
  Für $f:\prod_{x:A}B(x)\to B'(x)$ gilt: $f$ ist faserweise Äquivalenz, wenn $\sum f$ Äquivalenz ist.
\end{lemma}
\begin{beweis}
  Wird noch nachgeliefert...
\end{beweis}

\begin{theorem}[Fundamentaler Gleichheitssatz]\index{Fundamentaler Gleichheitssatz}
  \label{thm:fundamental-gleichheit}
  Seien $A:\mU$ und $a:A$. Für $B:A\to \mU$ und eine faserweise Abbildung $f:\prod_{x:A}(a=x)\to B(x)$, dann ist $f$ genau dann eine faserweise Äquivalenz, wenn
  \begin{mathpar}
    \sum_{x:A}B(x)
  \end{mathpar}
  kontrahierbar ist.
\end{theorem}
\begin{beweis}
  Mit \cref{lem:objektklassifizierer} wissen wir, dass $f$ genau dann eine faserweise Äquivalenz ist, wenn
  \begin{mathpar}
    \sum f:\left(\sum_{x:A}a=x\right)\to \left(\sum_{x:A}B(x)\right)
  \end{mathpar}
  eine Äquivalenz ist. In \cref{lem:pfade-kontrahierbar} haben wir gesehen, dass der linke Typ kontrahierbar ist, also gilt:
  \begin{mathpar}
    \left(\sum_{x:A}a=x\right)\simeq \eins
  \end{mathpar}
  Wenn nun auch $\eins\simeq \sum_{x:A}B(x)$ gilt, dann ist die Abbildung $\sum f$ bereits eine Äquivalenz, also auch $f$ eine faserweise Äquivalenz.
\end{beweis}

\begin{beispiel}
  Sei $\Eq{\zwei}:\zwei\to\zwei\to\mU$ gegeben durch Rekursion in beiden Argumenten:
  \begin{align*}
    \Eq{\zwei}(0_\zwei,0_\zwei)&\colonequiv\eins \\
    \Eq{\zwei}(0_\zwei,1_\zwei)&\colonequiv\leer \\
    \Eq{\zwei}(1_\zwei,0_\zwei)&\colonequiv\leer \\
    \Eq{\zwei}(1_\zwei,1_\zwei)&\colonequiv\eins 
  \end{align*}
  Mit \cref{thm:fundamental-gleichheit} können wir zeigen, dass
  \begin{mathpar}
    \prod_{x,y:\zwei}(x=_\zwei y)\simeq \Eq{\zwei}(x,y)
  \end{mathpar}
  indem wir für jedes $x:\zwei$ zeigen, dass $\sum_{y:\zwei}\Eq{\zwei}(x,y)$ kontrahierbar ist.
  Per $\zwei$-Induktion reicht es das für $x\equiv 0_\zwei,1_\zwei$ zu zeigen.
  Für
  \begin{mathpar}
    \sum_{y:\zwei}\Eq{\zwei}(0_\zwei,y)
  \end{mathpar}
  wählen wir als Kontraktionszentrum $(0_\zwei,\ast)$. Nun müssen wir zeigen, dass jeder andere Punkt gleich diesem Kontraktionszentrum ist, also mit $\sum$-Induktion:
  \begin{mathpar}
    \prod_{y:\zwei}\prod_{e:\Eq{\zwei}(0_\zwei,y)}(0_\zwei,\ast)=(y,e)
  \end{mathpar}
  Mit anwenden von $\zwei$-Induktion auf $y$ reicht es also, die beiden Fälle $\prod_{e:\Eq{\zwei}(0_\zwei,0_\zwei)}(0_\zwei,\ast)=(0_\zwei,e)$ und $\prod_{e:\Eq{\zwei}(0_\zwei,1_\zwei)}(0_\zwei,\ast)=(1_\zwei,e)$ abzuhandeln.
  Im ersten Fall kann $\eins$-Induktion angewandt werden und $\refl_{(0_\zwei,\ast)}$ ist eine Lösung. Im zweiten Fall ist $\leer$-Induktion eine Lösung. \\
  Analog lässt sich zeigen, dass $\sum_{y:\zwei}\Eq{\zwei}(1_\zwei,y)$, was den Beweis beendet.
\end{beispiel}

Es ist also insbesondere $\zwei$ ein 0-Typ bzw. eine Menge.


Wir wollen nun noch \cref{lem:faser-equiv} beweisen. Dazu brauchen wir das folgende alternative Induktionsprinzip für Gleichheit:

\begin{lemma}[Pfadinduktion mit Basispunkt]\index{Gleichheitsinduktion mit Basispunkt}\index{Pfadinduktion mit Basispunkt}
  Seien $A:\mU$ und $x:A$ fest gewählt. Für einen abhängigen Typen $C:\prod_{y:A}x=y \to\mU$ reicht es dann $c:C(x,\refl_x)$ zu konstruieren, um eine abhängige Funktion $\prod_{y:A}\prod_{p:x=y}C(y,p)$ zu erhalten.
\end{lemma}
\begin{beweis}
  Sei $c:C(x,\refl_x)$. Dann gibt es den abhängigen Typen
  \begin{mathpar}
    C'\colonequiv(z:\sum_{y:A}x=y)\mapsto C(\pi_1(z),\pi_2(z))
  \end{mathpar}
  Wir wissen, dass die Basis $\sum_{y:A}x=y$ kontrahierbar ist, also gibt es Gleichheiten $q_{y,p}:(x,\refl_x)=(y,p)$, für alle $y:A$ und $p:x=y$ und damit
  \begin{mathpar}
    \transp(q_{y,p}):C(x,\refl_x)\to C(y,p)
  \end{mathpar}
  Also
  \begin{mathpar}
    y\mapsto (p:x=y)\mapsto \transp(q_{y,p}):\prod_{y:A}\prod_{p:x=y}C(y,p)
  \end{mathpar}
\end{beweis}

Damit können wir zeigen:

\begin{lemma}
  Seien $A:\mU$ und $B:A\to\mU$. Für $C:\left(\sum_{x:A}B(x)\right)\to\mU$ gilt:
  \begin{mathpar}
    \left(\sum_{y:\sum_{x:A}B(x)}C(y)\right)\simeq \sum_{x:A}\sum_{b:B(x)}C((x,b))
  \end{mathpar}
\end{lemma}
\begin{beweis}
  Aus Übungsaufgabe 1 von Blatt 6 wissen wir, dass es einen Term gibt (mit $p_{(x,b)}\equiv\refl_{(x,b)}$):
  \begin{mathpar}
    p:\prod_{z:\sum_{x:A}B(x)}z=(\pi_1(z),\pi_2(z))
  \end{mathpar}
  Es gibt die beiden folgenden Kandidaten für zueinander inversen Abbildungen:
  \begin{align*}
    \varphi&:\left(\sum_{y:\sum_{x:A}B(x)}C(y)\right)\to \sum_{x:A}\sum_{b:B(x)}C((x,b))\\
    \varphi((y,c))&\colonequiv (\pi_1(y),(\pi_2(y),\transp_C(p_y)(c))) \\
    \psi&: \left(\sum_{x:A}\sum_{b:B(x)}C((x,b))\right)\to\left(\sum_{y:\sum_{x:A}B(x)}C(y)\right) \\
    \psi((x,(b,c)))&\colonequiv((x,b),c)
  \end{align*}
  Damit gilt $\varphi\circ\psi\sim\id$ bereits punktweise urteilsmäßig, es ist also noch zu zeigen, dass auch $\psi\circ\varphi\sim\id$ gilt.
  Für $y:\sum_{x:A}B(x)$ und $c:C(y)$ gilt:
  \begin{align*}
    \psi(\varphi(y,c))&=\psi(\pi_1(y),(\pi_2(y),\transp_{C}(p_y)(c))) \\
    &=((\pi_1(y),\pi_2(y)), \transp_{C}(p_y)(c))
  \end{align*}
  Letzteres ist allerdings mit $\sum_=$ und $p_y$ gleich zu $(y,c)$.
\end{beweis}

Damit können wir nun \cref{lem:faser-equiv} beweisen:

\begin{beweis}[von \cref{lem:faser-equiv}]
  Seien $A:\mU$, $B:A\to\mU$ und $x:A$. Für $\pi_1:\sum_{x:A}B(x)\to A$ gilt:
  \begin{align*}
    \pi_1^{-1}(x)&\equiv \sum_{y:\sum_{x:A}B(x)}\pi_1(y)=x\\
                 &\simeq \sum_{x':A}\sum_{b:B(x')}\pi_1((x',b))=x \\
                 &\equiv \sum_{x':A}\sum_{b:B(x')} x'=x \\
                 &\simeq \sum_{x':A}\sum_{p:x'=x}B(x')\\
                 &\simeq B(x)
  \end{align*}
\end{beweis}


\subsection{Gleichheit natürlicher Zahlen}
\label{sec:zahlen-gleichheit}
\begin{definition}
  $\Eq{\N}:\N\to\N\to\mU$ definieren wir durch doppelte Rekursion wie folgt:
  \begin{align*}
    \Eq{\N}(0_\N,    0_\N)     &\colonequiv\einheit \\
    \Eq{\N}(0_\N,    \sucN(k)) &\colonequiv\leer \\
    \Eq{\N}(\sucN(n),0_\N)     &\colonequiv\leer \\
    \Eq{\N}(\sucN(n),\sucN(k)) &\colonequiv\Eq{\N}(n,k) 
  \end{align*}
\end{definition}

\begin{theorem}[Gleichheit natürlicher Zahlen]
  \label{thm:gleichheit-nat}
  Die faserweise Abbildung
  \begin{mathpar}
    \prod_{n,l:\N}n=l\to \Eq{\N}(n,l)
  \end{mathpar}
  gegeben durch $\refl_n\mapsto \ast$ ist eine faserweise Äquivalenz.
\end{theorem}
\begin{beweis}
  Nach \cref{thm:fundamental-gleichheit} reicht es zu zeigen, dass für jedes $n:\N$ der Typ
  \begin{mathpar}
    \sum_{l:\N}\Eq{\N}(n,l)
  \end{mathpar}
  kontrahierbar ist. Als Kontraktionszentrum wählen wir $(n,\ast)$.
  Nach Induktion über $n$,$l$ und Ausschluss aller Fälle mit leerer Domäne bleiben noch zu konstruieren:
  \begin{align*}
    &\prod_{e:\Eq{\N}(0,0)}(0,e)=(0,\ast) \\
    &\prod_{e:\Eq{\N}(\sucN(n),\sucN(l))}(\sucN(l),e)=(\sucN(n),\ast)
  \end{align*}
  Im ersten Fall lässt sich die Gleichheit stets durch die kontrahierbarkeit von $\eins$ konstruieren.
  Im zweiten Fall dürfen wir die Induktionshypothese verwenden:
  \begin{mathpar}
    \mathrm{IH}:\prod_{e:\Eq{\N}(n,l)}(l,e)=(n,\ast)
  \end{mathpar}
  Um etwas damit anfangen zu können, definieren wir uns folgenden Funktion:
  \begin{align*}
    f&:\left(\sum_{n:\N}\Eq{\N}(n,l)\right)\to \left(\sum_{n:\N}\Eq{\N}(n,\sucN(l))\right) \\
    f((n,e))&\colonequiv (n+1,e)
  \end{align*}
  Damit gilt dann für $e:\Eq{\N}(\sucN(n),\sucN(l))$:
  \begin{mathpar}
    f(\mathrm{IH}_e):\left(f((l,e))=f((n,\ast))\right)\equiv \left((\sucN(l),e)=(\sucN(n),\ast)\right)
  \end{mathpar}
\end{beweis}

\begin{bemerkung}
  $\N$ ist ein 0-Typ.
\end{bemerkung}
\begin{beweis}
  Wir wissen dass die Typen $\leer$ und $\eins$ Aussagen sind durch \cref{bsp:leer-eins-hlevel} und \cref{bem:kontrahierbar-folgt-aussage}.
  Mit $\N$-Induktion ist also $\Eq{\N}(n,k)$ für alle $n,k\in\N$ eine Aussage.
  Nach Definition ist ein Typ ein 0-Typ oder eine Menge, wenn alle Gleichheitstypen Aussagen sind, also sind wir mit \cref{thm:gleichheit-nat} fertig.
\end{beweis}

\subsection{Gleichheit algebraischer Strukturen}

Wir wollen in diesem Abschnitt am Beispiel der Halbgruppen sehen, dass Gleichheiten zwischen algebraischen Objekten genau den Isomorphismen entsprechen.
Da es sich etwa beim Typ der Halbgruppen um eine abhängige Summe handeln wird und wir die Gleichheitstypen vollständig verstehen wollen,
werden wir zunächst die Chrakterisierung von Gleichheitstypen in abhängigen Summen abschließen.

\begin{lemma}[Gleichheit in abhängigen Summen]
  
\end{lemma}

\section{Homotopietheorie}
\subsection{Höhere Induktive Typen}
\label{subsec:HITs}
Höhere induktive Typen haben neben Konstruktoren der Bauart, die wir bereits von den Induktiven Typen kennen, sogenannte \begriff{höhere Konstruktoren},
deren Werte in Gleichheitstypen des höheren induktiven Typs liegen.
Später werden wir auch höhere induktive Typen mit höheren Konstruktoren, die in iterierten Gleichheitstypen liegen, sehen.
Zunächst wird der Typ $S¹$ zentral sein. Dieser hat wie der Typ $\eins$ einen Punkt-Konstruktor $\ast:S^1$ und zusätzlich einen
höheren Konstruktor $l:\ast =_{S^1}\ast$.

Eine Rekursionsregel ist leicht formuliert: Um eine Funktion $f:S^1\to A$ zu definieren, muss man $f(\ast):A$ und $f(l):f(\ast)=_A f(\ast)$ vorgeben.
Das Induktionsprinzip ist etwas unhandlicher und wir wollen zunächst ein paar Definition machen, die uns die Formulierung erleichtern und schließlich nach der Einführung der neuen Regeln
das Rekursionsprinzip aus dem Induktionsprinzip folgern.

In \cref{lem:gleichheit-summe} haben wir festgestellt, dass ein Teil einer Gleichheit in einer abhängigen Summe von der Form
\begin{mathpar}
  \transp_B(p)(b)=b'
\end{mathpar}
ist. Das werden wir jetzt als allgemeine Definition von Gleichheiten in abhängigen Typen verwenden.
\begin{definition}[Abhängige Gleichheit]
  Seien $A:\mU$, und $B:A\to \mU$. Seien weiter $x,y:A$, $b:B(x)$ und $b':B(y)$.
  \begin{enumerate}
  \item Sei $p:x=_A y$. Ein \begriff{abhängier Pfad} oder eine \begriff{abhängige Gleichheit} zwischen $b$ und $b'$ ist ein Element des Typs
    \begin{mathpar}
      \left(b =_p^B b'\right)\colonequiv \left(\transp_B(p)(b)=b'\right)
    \end{mathpar}
  \item Sei $s:\prod_{x:A}B(x)$ eine abhängige Funktion. Die folgende Funktion ist die Anwendung eine abhängigen Funktion auf eine Gleichheit:
    \begin{mathpar}
      \mathrm{apd}(s):\prod_{p:x=_A y}s(x)=_p^B s(y)
    \end{mathpar}
    und festgelegt durch $\mathrm{apd}(s,\refl_x)\colonequiv\refl_b$. Wir schreiben auch $s(p)$ für $\mathrm{apd}(s,p)$.
  \end{enumerate}
\end{definition}

Damit können wir die folgenden Regeln für den induktiven Einheitskreis leichter formulieren.
Zunächst wollen wir noch die abhängige Variante mit der bekannten Anwendung von Funktionen auf Gleichheiten vergleichen:

\begin{bemerkung}
  \label{bem:transpconst}
  Seien $A,B:\mU$. Dann gilt:
  \begin{enumerate}
  \item Für $x,y:A$, $p:x=_A y$ und alle $b:B$ erhalten wir eine abhängige Gleichheit
    \begin{mathpar}
      \transpconst_{p,b}:\transp_{\_\mapsto B}(p)(b)=b
    \end{mathpar}
    durch Induktion über $p$ und $\transpconst_{\refl_x,b}\colonequiv \refl_b$.
  \item Für $f:A \to B$ und $p:x =_A y$ gilt $\mathrm{apd}(f,p)= \transpconst_{p,f(x)}\kon\mathrm{ap}(f,p)$.
  \end{enumerate}
\end{bemerkung}
\begin{beweis}
  \begin{enumerate}
  \item Bereits erledigt.
  \item Induktion über $p$.
  \end{enumerate}
\end{beweis}


Anders als bei den Konstruktoren induktiver Typen, werden wir bei höheren Konstruktoren keine urteilsmäßige Gleichheit für Berechnungen fordern,
sondern nur Gleichheit.

\begin{regeln}
  Es gibt einen Typ $S^1:\mU$ den wir den (höheren induktiven) \begriff{Kreis}\index{Kreis}\index{$S^1$} nennen.
  $S^1$ ist der höhere induktive Typ mit den Konstruktoren:
  \begin{align*}
    \ast:S^1 \\
    l:\ast =_{S^1}\ast
  \end{align*}
  Das heißt für jeden abhängigen Typ $B:S^1\to\mU$ reicht es $b:B(\ast)$ und $b_l:b =_l^B b$ vorzugeben,
  um eine abhängige Funktion $s:\prod_{x:S^1}B(x)$ zu definieren.
  Es gibt also eine Funktion:
  \begin{mathpar}
    \ind{S^1}:\prod_{B:S^1\to\mU}\prod_{b:B(\ast)}\left((b =_l^B b)\to\prod_{x:S^1}B(x)\right)
  \end{mathpar}
  die $\begriff{Kreisinduktion}$.
  Es gelten folgende Berechnungsregeln:
  \begin{align*}
    \ind{S^1}(B,b,b_l)(\ast)&\equiv b \\
    \mathrm{apd}(\ind{S^1}(B,b,b_l),l)\equiv \ind{S^1}(B,b,b_l)(l)&=b_l
  \end{align*}
  Wir werden abhängige Funktionen $s:\prod_{x:S^1}B(x)$ auch durch Fallunterscheidung wie folgt angeben:
  \begin{align*}
    s(\ast)\colonequiv b \\
    s(l):= b_l
  \end{align*}
\end{regeln}

\begin{bemerkung}
  Wie bereits bei den induktiven Typen, werden wir auch hier darauf verzichten, ein allgemeines Schema anzugeben.
  Unter Homotopietypentheorie versteht man eine Typentheorie wie soweit eingeführt, in der jeder sinnvolle höhere induktive Typ existiert.
  Wir werden lediglich Beispiele von sinnvollen höheren induktiven Typen kennenlernen.
\end{bemerkung}

Wir wollen zunächst das Rekursionprinzip für den Kreis herleiten.

\begin{bemerkung}
  Für $A:\mU$, $a:A$ und $a_l:a=_A a$ gibt es stets eine Funktion $\rec{S^1}(A,a,a_l):S^1\to A$ mit
  \begin{align*}
    \rec{S^1}(A,a,a_l)(\ast)\equiv a \\
    \rec{S^1}(A,a,a_l)(l)=a_l
  \end{align*}
\end{bemerkung}
\begin{beweis}
  Wir verwenden Kreisinduktion für den konstanten abhängigen Typ $\_\mapsto A:S^1\to\mU$.
  Es gibt eine Äquivalenz
  \begin{mathpar}
    (a =_l a) \simeq (a =_A a)
  \end{mathpar}
  durch Linkskonkatenation mit $\transpconst_{l,a}^{-1}$.
  Wir verwenden die Inverse, um aus $a_l:a=_A a$ die Gleichheit $\transpconst_{l,a}\kon a_l:a=_l a$ zu produzieren.
  Damit gilt für
  \begin{mathpar}
    \rec{S^1}(A,a,a_l)\colonequiv \ind{S^1}(\_\mapsto A, a, \transpconst_{l,a}\kon a_l)
  \end{mathpar}
  nach \cref{bem:transpconst}:
  \begin{align*}
    \mathrm{ap}(\rec{S^1}(A,a,a_l),l)&\equiv \mathrm{ap}(\ind{S^1}(\_\mapsto A, a, \transpconst_{l,a}\kon a_l),l) \\
                                     &=\transpconst_{l,a}^{-1}\kon\mathrm{apd}(\ind{S^1}(\_\mapsto A, a, \transpconst_{l,a}\kon a_l),l) \\
                                     &=\transpconst_{l,a}^{-1}\kon\transpconst_{l,a}\kon a_l \\
                                     &=a_l
  \end{align*}
  Nach Definition gilt außerdem $\rec{S^1}(A,a,a_l)(\ast)\equiv a$.
\end{beweis}

Bevor wir den Kreis weiter kennenlernen, betrachten wir noch einen weiteren, sehr ähnlichen höheren induktiven Typ.

\begin{regeln}
  Es gibt einen Typ $I:\mU$, das \begriff{Intervall} mit den folgenden Konstruktoren:
  \begin{align*}
    0_I:I \\
    1_I:I \\
    s:0_I =_I 1_I
  \end{align*}
  Daraus ergibt sich das folgende Induktionsprinzip, die \begriff{Intervallinduktion}:
  \begin{mathpar}
    \ind{I}:\prod_{B:I\to\mU}\prod_{b_0:B(0_I)}\prod_{b_1:B(1_I)}b_0 =_s^B b_1 \to\prod_{x:I}B(x) 
  \end{mathpar}
  mit Berechnungsregeln:
  \begin{align*}
    \ind{I}(B,b_0,b_1,b_s)(0_I)\equiv b_0 \\
    \ind{I}(B,b_0,b_1,b_s)(1_I)\equiv b_1 \\
    \mathrm{apd}(\ind{I}(B,b_0,b_1,b_s),s)\equiv\ind{I}(B,b_0,b_1,b_s)(s) = b_s \\
  \end{align*}
\end{regeln}

\begin{bemerkung}
  Das Intervall $I$ ist kontrahierbar.
\end{bemerkung}
\begin{beweis}
  Als Kontraktionszentrum wählen wir $0_I$. Nun wollen wir zeigen:
  \begin{mathpar}
    \prod_{x:I}0_I=x
  \end{mathpar}
  Mit Intervallinduktion reicht es dafür festzustellen, dass $\refl:0_I=0_I$ und $s:0_i=1_I$ gelten und zu zeigen:
  \begin{mathpar}
    \refl =_s s
  \end{mathpar}
  oder anders formuliert: $\transp_{0_I=\_}(s)(\refl)=s$. Das haben wir aber bereits allgemein geklärt in \cref{lem:transp-lpath}.
\end{beweis}

Trotzdem können wir mit dem Intervall etwas neues zeigen.
Davor brauchen wir noch das Rekursionsprinzip.
\begin{lemma}[Intervallrekursion und ihre Eindeutigkeit]
  Sei $A:\mU$.
  \begin{enumerate}
  \item Für $a,a':A$ und $a_s:a=_A a'$ gibt es
    \begin{mathpar}
      \rec{I}(A,a,a',a_s):I\to A
    \end{mathpar}
    mit $\rec{I}(A,a,a',a_s)(0_I)\equiv a$, $\rec{I}(A,a,a',a_s)(1_I)\equiv a'$ und $\rec{I}(A,a,a',a_s)(s) = a_s$.
  \item Für $f:I\to A$ gilt:
    \begin{mathpar}
      f=\rec{I}(A,f(0_I),f(1_I),f(s))
    \end{mathpar}
  \end{enumerate}
\end{lemma}
\begin{beweis}
  \begin{enumerate}
  \item Wie bei der Kreisrekursion.
  \item Noch nicht entschieden, ob das eine Übungsaufgabe ist.
  \end{enumerate}
\end{beweis}

Zunächst erweitern wir die Definition von Homotopien auf abhängige Typen:
\begin{definition}
  Seien $A:\mU$, $B:A\to \mU$ und $f,g:\prod_{x:A}B(x)$. Eine \begriff{abhängige Homotopie}\index{$\sim$} von $f$ nach $g$ ist ein Term von
  \begin{mathpar}
    f\sim g\colonequiv \prod_{x:A}f(x)=g(x)
  \end{mathpar}
  Weiter verwenden wir in diesem Abschnitt die Funktion
  \begin{mathpar}
    \mathrm{hap}\colonequiv (p:f=g)\mapsto (x:A)\mapsto\mathrm{ap}(f\mapsto f(x),p) :f=g\to f\sim g
  \end{mathpar}
\end{definition}

\begin{theorem}[Abhängige Funktionsextensionalität]
  Seien $A:\mU$ und $B:A\to\mU$. Dann gilt für abhängige Funktionen $f,g:\prod_{x:A}B(x)$:
  \begin{enumerate}
  \item Es gibt eine Funktion $\varphi:f\sim g \to f = g$ und es gilt $\mathrm{hap}(\varphi(H))=H$ für jede abhängige Homotopie $H:f\sim g$.
  \item Die Funktion
    \begin{mathpar}
      \mathrm{hap} :f=g\to f\sim g
    \end{mathpar}
    ist eine Äquivalenz.
  \end{enumerate}
\end{theorem}
\begin{beweis}
  \begin{enumerate}
  \item   Zunächst konstruieren wir für $H:\prod_{x:A}f(x)=g(x)$ eine Gleichheit $p_H:f=g$.
  Für $x:A$ können wir die Gleichheit $f(x)=g(x)$ durch eine Abbildung $I\to B(x)$ ersetzen
  und damit $H$ durch die Abbildung
  \begin{mathpar}
    \tilde{H}\colonequiv (x:A)\mapsto \rec{I}(B(x),f(x),g(x),H_x) : \prod_{x:A}I\to B(x)
  \end{mathpar}
  Nun vertauschen wir $A$ und $I$:
  \begin{mathpar}
    \tilde{\tilde{H}}\colonequiv (i:I)\mapsto (x:A)\mapsto \tilde{H}(x)(i) : I\to\prod_{x:A}B(x)
  \end{mathpar}
  Und damit haben wir $p_H\colonequiv \tilde{\tilde{H}}(s) : f=g$.

  Nun können wir durch kommutieren von $\mathrm{ap}$ und $\circ$ sehen, dass gilt:
  \begin{align*}
    \mathrm{hap}(\varphi(H))&\equiv (x:A)\mapsto\mathrm{ap}(f\mapsto f(x),\varphi(H)) \\
                            &\equiv (x:A)\mapsto\mathrm{ap}(f\mapsto f(x),\mathrm{ap}((i:I)\mapsto (y:A)\mapsto \tilde{H}(y)(i),s)) \\
                            &= (x:A)\mapsto \mathrm{ap}((i:I)\mapsto \tilde{H}(x)(i),s) \\
                            &= (x:A)\mapsto H_x
  \end{align*}
\item Nun können wir den Gleichheitssatz verwenden, wenn wir es schaffen zu zeigen, dass
  \begin{mathpar}
    \sum_{g:\prod_{x:A}B(x)}f\sim g
  \end{mathpar}
  kontrahierbar ist. Dazu berechnen wir erstmal mit Gleichheitsinduktion über $p:g=g'$,
  dass für den entsprechenden Transport und eine Homotopie $H:f\sim g$ gilt:
  \begin{mathpar}
    \transp_{g\mapsto f\sim g}(p)(H)=(x:A)\mapsto H_x\kon \mathrm{hap}(p)_x
  \end{mathpar}
  Sei $(f,x\mapsto \refl_{f(x)})$ unser Kontraktionszentrum.
  Dann ist für jedes weitere $(g,H):\sum_{g:\prod_{x:A}B(x)}f\sim g$ durch die Konstruktion am Anfang des Beweises eine Gleichheit $p_H:f=g$ gegeben.
  Um zu zeigen, dass $(f,x\mapsto \refl_{f(x)})=(g,H)$ gilt, müssen wir nach der Charakterisierung in \cref{lem:gleichheit-summe} also nur noch zeigen:
  \begin{mathpar}
    \transp_{g\mapsto f\sim g}(p_H)(x\mapsto\refl_{f(x)})=H
  \end{mathpar}
  Linke und rechte Seite dieser Gleichung sind abhängige Funktionen des Typs $\prod_{x:A}f(x)=g(x)$,
  also können wir die Konstruktion ``$p_H$'' anwenden, um die benötigte Gleichheit aus einer Punktweisen Gleichheit, also aus 
  \begin{mathpar}
    \transp_{g\mapsto f\sim g}(p_H)(x\mapsto\refl_{f(x)})\sim H
  \end{mathpar}
  zu folgern. Diese gilt aber bereits nach der Berechnung des Transports und (a).
\end{enumerate}
\end{beweis}

Nachdem wir nun die vermutlich einfachsten nicht-trivialen höheren induktiven Typen (HIT) kennen gelernt haben,
wollen wir uns zum Abschluss des Abschnitts noch einem etwas komplizierteren HIT widmen,
dem sogenannten Mengenquotient.

\begin{regeln}
  Seien $A:\mU$ und $R:A\to A\to \mU$. Dann ist $A / R:\mU$ der höhere induktive Typ mit den folgenden Konstruktoren:
  \begin{align*}
    [\_]&:A\to A / R \\
    \mathrm{eq /}&:\prod_{x,y:A}R(x,y)\to [x] =_{A / R} [y] \\
    \mathrm{set /}&:\prod_{x,y:A / R}\prod_{p,q : x=y}p=q
  \end{align*}
  Der Typ $A / R$ heißt \begriff{Mengenquotient}.
\end{regeln}

\begin{bemerkung}
  $A / R$ ist stets eine Menge. Und es gibt folgendes Rekursionsprinzip:
  Für jede Menge $B$, Funktion $f:A\to B$ und Nachweis, dass die Relation respektiert wird,
  also einen Term in
  \begin{mathpar}
    r:\prod_{x,y:A}R(x,y)\to f(x)=_B f(y)
  \end{mathpar}
  gibt es eine Funktion
  \begin{mathpar}
    \rec{A / R}(B,f,r) : A / R \to B
  \end{mathpar}
  Analog gibt es ein vereinfachtes Induktionsprinzip.
  Für jede Aussage $P:A / R \to \mU$ (also $P(x)$ Aussage für jedes $x:A / R$) reicht es $P([a])$ für alle $a:A$ zu zeigen, um $\prod_{x:A / R}P(x)$ zu zeigen.
\end{bemerkung}
\begin{beweis}
  Auf die Fälle für die höheren Konstruktoren kann jeweils verzichtet werden, da die entpsrechenden Gleichheiten in Mengen bzw. Aussagen stets existieren.
\end{beweis}

Wir werden nun die Konstruktion von $\Z$ als Quotient der Menge der Paare in $\N\times \N$ mithilfe von Mengenquotienten nachbauen.
Die Paare $(n,k):\N\times\N$ darf man sich als Differenz ``$n-k$'' vorstellen, was auch die folgende Relation erklärt.

\begin{definition}
  Sei für $(n,k),(n',k'):\N\times\N$:
  \begin{mathpar}
    (n,k) \sim_\Z (n',k')\colonequiv n+k'=_\N n'+k
  \end{mathpar}
  Die \begriff{Ganzen Zahlen}\index{$\Z$} sind wie folgt als Mengenquotient gegeben:
  \begin{mathpar}
    \Z\colonequiv \N\times\N / {\sim_\Z}
  \end{mathpar}
\end{definition}

\begin{definition}
  Die Funktion $\mathrm{succ}_\Z:\Z \to \Z$ sei durch Rekursion gegeben durch:
  \begin{mathpar}
    s(n,k)\colonequiv (\sucN(n),k) 
  \end{mathpar}
  und den Nachweis der Wohldefiniertheit:
  Sei $(n,k)\sim_\Z (n',k')$, dann gilt $n+k'=n'+k$, also auch $\sucN(n+k')=\sucN(n'+k)$.
  Letzteres ist nach der Definition der Addition $\sucN(n)+k'=\sucN(n')+k$, also gilt $s(n,k)\sim_\Z s(n',k')$.
\end{definition}

\subsection{n-Typen}
In \cref{def:kontr-aussage-menge} hatten wir bereits kontrahierbare Typen, Aussagen und Mengen definiert.
Die Mengen haben den alternativen Namen $0$-Typen und zu Aussagen, kann man auch $-1$-Typen und zu kontrahierbaren Typen $-2$-Typen sagen.
Wir werden in diesem Abschnitt $n$-Typen definieren, für $n=-2,-1,0,1,\dots$.
Dazu sei zunächst:
\begin{definition}
  Es sei $\N_{-2}$ der induktive Typ mit Konstruktoren $-2:\N_{-2}$ und $\sucHL:\N_{-2}\to\N_{-2}$.
  Für $\sucHL(n)$ schreiben wir $n+1$.
  Die Elemente von $\N_{-2}$ nennen wir auch \begriff{Abschneidungslevel}.
\end{definition}
Damit können wir $n$-Typen für jeden Abschneidungslevel $n$ definieren:
\begin{definition}
  \begin{enumerate}
  \item Für jeden Typ $A$ sei der Typ $\isNType{n}$ wie folgt per Rekursion über $n:\N_{-2}$ definiert:
    \begin{align*}
      \isNType{(-2)}(A)&\colonequiv \isContr(A) \\
      \isNType{(n+1)}(A)&\colonequiv \prod_{x,y:A}\isNType{n}(x=_A y)
    \end{align*}
  \item Der \begriff{Typ der $n$-Typen} ist für $n:\N_{-2}$:
    \begin{mathpar}
      n\mathrm{-Type}\colonequiv \sum_{A:\mU} \isNType{n}(A)
    \end{mathpar}
    Diesen Typ gibt es also für jeden Universenlevel.
  \end{enumerate}
\end{definition}

\begin{lemma}
  Sind Typen $A,B:\mU$ äquivalent und ist $A$ ein $n$-Typ, dann ist auch $B$ ein $n$-Typ.
\end{lemma}
\begin{beweis}
  Univalenz.
\end{beweis}

\begin{lemma}
  \label{lem:pi-n-type}
  Sei $A:\mU$ und $B:A\to\mU$ so, dass $B(x)$ für jedes $x:A$ ein $n$-Typ ist.
  Dann ist $\prod_{x:A}B(x)$ ein $n$-Typ.
\end{lemma}
\begin{beweis}
  Wir beweisen zunächst den Fall $n\equiv -2$. Seien also die $B(x)$ kontrahierbar.
  Dann gibt es eine abhängige Funktion $z:\prod_{x:A}B(x)$, die jedes $x:A$ auf das Kontraktionszentrum von $B(x)$ abbildet.
  Es reicht also zu zeigen, dass $\prod_{x:A}B(x)$ eine Aussage ist.
  Seien dazu $f,g:\prod_{x:A}B(x)$. Wir wollen $f=g$ zeigen.
  Mit abhängiger Funktionsextensionalität reicht es zu zeigen, dass $f\sim g$ gilt.
  Wir wissen aber, dass für jedes $x:A$ bereits $f(x)=g(x)$ gilt, da $B(x)$ eine Aussage ist.

  Wir zeigen nun die eigentliche Aussage per Induktion über den Abschneidungslevel.
  Seien nun also die $B(x)$ $n+1$-Typen.
  Für $f,g:\prod_{x:A}B(x)$ gilt $f=g \simeq f\sim g \equiv \prod_{x:A}f(x)=_{B(x)}g(x)$.
  Letzteres ist ein Gleichheitstyp in einem $n+1$-Typ, also ein $n$-Typ.
  Damit können wir die Induktionshypothese anwenden und sehen, dass auch $f=g$ ein $n$-Typ sein muss und damit dass $\prod_{x:A}B(x)$ ein $n+1$-Typ ist.
\end{beweis}

\begin{lemma}
  \label{lem:isprop-hlevel}
  Sei $A:\mU$. Für jedes $n:\N_{-2}$ gilt $\isProp(\isNType{n}(A))$.
\end{lemma}
\begin{beweis}
  Induktion über den Abschneidungslevel.
  Wir zeigen also zunächst: $\isProp(\isContr(A))$.
  Seien also
  \begin{mathpar}
    (a,H),(a',H'):\isContr(A)\equiv\sum_{x:A}\prod_{y:A}x=y
  \end{mathpar}
  Mit \cref{lem:gleichheit-summe} müssen wir also zeigen:
  \begin{mathpar}
    \sum_{p:a=a'}\transp_{x\mapsto \prod_{y:A}x=y}(p)(H)=H'
  \end{mathpar}
  wobei der Transport die Linkskonkatenation mit $p^{-1}$ ist.
  Wir wählen $p\colonequiv H_{a'}$. Dann ist noch zu zeigen:
  \begin{mathpar}
    \prod_{y:A}H_{a'}^{-1}\kon H_y=H'_y
  \end{mathpar}
  Das lässt sich etwas überraschend auf dem Umweg über die Verallgemeinerung
  \begin{mathpar}
    \prod_{q:x=y}H_{x}^{-1}\kon H_y=q
  \end{mathpar}
  mit Gleichheitsinduktion zeigen. Damit ist also $\isContr(A)$ eine Aussage.

  Jetzt nehmen wir also an, dass $\isNType{n}(A)$ eine Aussage ist und zeigen, dass $\isNType{(n+1)}(A)$ eine Aussage ist.
  Nach Definition gilt
  \begin{mathpar}
    \isNType{(n+1)}(A)\equiv \prod_{x,y:A}\isNType{n}(x=_A y)
  \end{mathpar}
  Für festes $x:A$ ist also mit \cref{lem:pi-n-type} der Typ $\prod_{y:A}\isNType{n}(x=_A y)$ eine Aussage.
  Mit erneutem Anwenden von \cref{lem:pi-n-type}, ist also auch $\prod_{x:A}\prod_{y:A}\isNType{n}(x=_A y)$ eine Aussage.
\end{beweis}


Damit können wir sofort einsehen:

\begin{korollar}
  Seien $A,B:\mU$ und $f:A\to B$.
  Die folgende Definition von $\isEquiv(f)$ ist eine Aussage:
  \begin{mathpar}
    \prod_{y:B}\isContr(f^{-1}(y))
  \end{mathpar}
\end{korollar}

\begin{beweis}
  Zunächst wissen wir durch \cref{lem:isprop-hlevel}, dass $\isContr(f^{-1}(y))$ für jedes $y:B$ eine Aussage ist.
  Mit \cref{lem:pi-n-type} wissen wir außerdem, dass das abhängige Produkt über Aussagen wieder eine Aussage ist.
\end{beweis}

Wir haben in \cref{lem:pi-n-type} bereits gesehen, dass die Gesamtheit der $n$-Typen für festes $n$ unter abhängigen Produkten abgeschlossen ist.
Nun werden wir sehen, dass Ähnliches für den Großteil der Konstruktionen gilt, die wir soweit kennen.

\begin{bemerkung}
  \label{bem:n-type-cumulative}
  \begin{enumerate}
  \item Jeder $n$-Typ ist auch ein $(n+1)$-Typ.
  \item Wenn $A$ ein $n$-Typ ist, dann sind auch alle Gleichheitstypen $x=_A y$ für $x,y:A$ wieder $n$-Typen.
  \end{enumerate}
\end{bemerkung}
\begin{beweis}
  \begin{enumerate}
  \item Sei zunächst $n\equiv -2$, also $A$ kontrahierbar mit Zeugen $(z,H):\isContr(A)$.
    Für $x,y:A$ wählen wir $H^{-1}_x\kon H_y: x=_A y$ als Kontraktionszentrum.
    Per Gleichheitsinduktion gibt es für beliebiges $p:x=y$ eine Gleichheit $H_x^{-1}\kon H_y=p$.

    Für den Induktionsschritt sei nun $A$ ein $(n+1)$-Typ.
    Damit sind alle $x=_A y$ $n$-Typen. Nach Induktionshypothese ist $x=y$ also auch ein $(n+1)$-Typ und damit $A$ ein $(n+2)$-Typ.
  \item Für $n\equiv -2$ haben wir das bereits in (a) nachgerechnet.
    Nach Definition ist für einen $n+1$-Typ $A$ jeder Gleichheitstyp ein $n$-Typ, also nach (a) auch ein $n+1$-Typ.
  \end{enumerate}
\end{beweis}

Wenn man bedenkt, dass es für jeden Typ $A$ eine Äquivalenz $A\simeq \sum_{x:A}\eins$ gibt, wird klar, dass für abhängige Summen
kein direktes Analogon zu \cref{lem:pi-n-type} gelten kann. Weiter wäre das seltsam, weil der Spezialfall von Produkten symmetrisch ist.
Es liegt also Nahe, für abhängige Summen auch Bedingungen an die Basis zu stellen, um den Abschneidungslevel zu erhalten:

\begin{lemma}
  \label{lem:sum-n-type}
  Seien $A:\mU$ und $B:A\to\mU$. Wenn $A$ ein $n$-Typ und $B(x)$ für jedes $x:A$ ein $n$-Typ ist, dann ist auch die abhängige Summe $\sum_{x:A}B(x)$ ein $n$-Typ.
\end{lemma}
\begin{beweis}
  Sei $n\equiv -2$, also $A$ kontrahierbar und für jedes $x:A$ $B(x)$ kontrahierbar.
  Wir wählen $(z,b):\sum_{x:A}B(x)$ als Kontraktionszentrum, für ein Kontraktionszentrum $z:A$ von $A$ und $b:B(z)$ von $B(z)$.
  Seien $z':A$ und $b':B(z')$. Es gilt:
  \begin{mathpar}
    (z,b)=(z',b') \simeq \sum_{p:z=z'}\transp_B(p)(b)=b'
  \end{mathpar}
  Die beiden Gleichheiten auf der rechten Seite gibt es jeweils wegen Kontrahierbarkeit.

  Seien nun $A$ und jedes $B(x)$ ein $(n+1)$-Typ. Dann ist ohne Einschränkung zu zeigen, dass für $x,y:A$ und $b:B(x)$, $b':B(y)$ der Typ
  \begin{mathpar}
    (x,b)=(y,b') \simeq \sum_{p:x=y}\transp_B(p)(b)=b'
  \end{mathpar}
  ein $n$-Typ ist. Die rechte Seite ist aber eine abhängige Summe von $n$-Typen, also nach Induktionshypothese ein $n$-Typ.
\end{beweis}

\begin{bemerkung}
  Sei $n\geq 0$. Koprodukte von $n$-Typen sind $n$-Typen.
\end{bemerkung}
\begin{beweis}
  Seien $A,B:\mU$ und $(n+1)\geq -1$ ein Abschneidungslevel.
  Wir zeigen, dass $A\amalg B$ ein $(n+1)$-Typ ist, indem wir zeigen, dass alle Gleichheitstypen von $A\amalg B$ $n$-Typen sind.
  Nach Übungsaufgabe 3 von Blatt 7, ist der Gleichheitsyp des Koprodukts $A\amalg B$ faserweise äquivalent zu:
  \begin{align*}
    \Eq{\amalg}(\iota_1(a),\iota_1(a'))&\colonequiv (a=_A a') \\
    \Eq{\amalg}(\iota_1(a),\iota_2(b))&\colonequiv \leer \\
    \Eq{\amalg}(\iota_2(b),\iota_1(a))&\colonequiv \leer \\
    \Eq{\amalg}(\iota_2(b),\iota_2(b'))&\colonequiv (b=_B b')
  \end{align*}
  Der Typ $\leer$ ist eine Aussage. Da $n\geq -1$ ist, ist also $\leer$ mit \Cref{bem:n-type-cumulative} auch ein $n$-Typ.
  Die Typen $a=a'$ und $b=b'$ sind auch jeweils $n$-Typen, also sind alle Gleichheitstypen von $A\amalg B$ $n$-Typen.
\end{beweis}

\begin{bemerkung}
  \begin{enumerate}
  \item Seien $A,B:\mU$ und $f:A\to B$. Die Fasern von $\mathrm{ap}(f,\_):x=y\to f(x)=f(y)$ sind genau dann $n$-Typen, wenn jede Faser von $f$ ein $(n+1)$-Typ ist.
  \item Untertypen von $n$-Typen sind $n$-Typen.
  \end{enumerate}
\end{bemerkung}
\begin{beweis}
  \begin{enumerate}
  \item Die Aussage folgt, wenn die Gleichheitstypen jeder Faser von $f$ äquivalent zu Fasern von $\mathrm{ap}(f,\_)$ sind und umgekehrt.
    Seien also $y:B$ und $(x,p),(x',p'):f^{-1}(y)$, dann gilt:
    \begin{align*}
      (x,p)=(x',p') &\simeq \sum_{q:x=x'}\transp_{f(\_)=y}(q)(p)=p' \\
                    &\simeq \sum_{q:x=x'}f(q)^{-1}\kon p=p' \\
                    &\simeq \sum_{q:x=x'}f(q)=p'^{-1}\kon p \\
                    &\simeq \mathrm{ap}(f,\_)^{-1}(p'^{-1}\kon p)
    \end{align*}
    Und für $p:f(x)=f(y)$ gilt:
    \begin{align*}
      \mathrm{ap}(f,\_)^{-1}(p) &\sum_{q:x=y}f(q)=p \\
      &\simeq \sum_{q:y=x}f(q)^{-1}\kon p=\refl_{f(x)}\\
                                &\simeq ((y,p)=_{f^{-1}(f(x))}(x,\refl_{f(x)}))
    \end{align*}
  \item Das folgt aus (a) für $n\equiv -1$ und Übungsaufgabe 4 von Blatt 9.
  \end{enumerate}
\end{beweis}

Es ist nicht der Fall, dass das Universum der $n$-Typen, also der Typ
\begin{mathpar}
  n\hbox{-}\mathrm{Type}\equiv\sum_{A:\mU}\isNType{n}(A)
\end{mathpar}
wieder ein $n$-Typ ist. Ein Gegenbeispiel ist durch $(\zwei=\zwei)\simeq \zwei$ gegeben: $\zwei$ ist ein $0$-Typ, aber es gibt zwei verschiedene Gleichheiten zwischen $\zwei$ und $\zwei$ im Typ $\nType{0}$.
Letzterer könnte also bestenfalls noch ein $1$-Typ sein. Das ist tatsächlich so:

\begin{bemerkung}
  \label{bem:level-of-n-typ}
  Der Typ der $n$-Typen ist ein $(n+1)$-Typ.
\end{bemerkung}
\begin{beweis}
  Für $n\equiv -2$, sind alle Typen äquivalent zu $\eins$ und $(\eins=\eins) \simeq \eins$ ist ein (-2)-Typ, also auch ein (-1)-Typ.
  
  Seien nun $A,B:\mU$ zwei (n+1)-Typen. Wir wollen zeigen, dass im Typ der (n+1)-Typen also für Paare $(A,X),(B,Y):\nType{(n+1)}$ gilt, dass $(A,X)=(B,Y)$ ein $n$-Typ ist.
  Da $X,Y$ nur Elemente von Aussagen sind gilt:
  \begin{mathpar}
    \left((A,X)=_{\nType{(n+1)}}(B,Y) \right)\simeq \left(A=_\mU B\right) \simeq (A\simeq B)
  \end{mathpar}
  Letzteres ist ein Untertyp von $A\to B$, was nach \cref{lem:pi-n-type} ein $n$-Typ ist.
\end{beweis}

\subsection{Überlagerungen I}

Bevor wir uns dem eigentlichen Thema zuwenden, brauchen wir noch eine Variante der ganzen Zahlen mit einem Rekursionsprinzip,
das uns auch Abbildungen in nicht notwendigerweise 0-abgeschnittene Typen erlaubt.

\begin{regeln}
  Zunächst sei $\N_1:\mU$\index{$\N_1$} der induktive Typ mit Konstruktoren $1_{\N_1}:\N_1$ und $\mathrm{succ}_{\N_1}:\N_1\to\N_1$.
  Nun sei $\Z':\mU$ der induktive Typ mit Konstruktoren:
  \begin{align*}
    0_{\Z'}&:\Z' \\
    \mathrm{pos}&:\N_1\to\Z' \\
    \mathrm{neg}&:\N_1\to\Z'
  \end{align*}
  Wir nennen den Typ $\Z'$ die \begriff{(induktiven) ganzen Zahlen}\index{$\Z'$}\index{$\Z$} und werden auch später auf den Strich verzichten und einfach $\Z$ schreiben.
  Als Induktionsprinzip ergibt sich also, dass wir Konstruktionen für die positiven Zahlen, die negativen Zahlen und die 0 ausführen müssen.
\end{regeln}

\begin{bemerkung}
  Die Typen $\N_1$ und $\Z'$ sind Mengen.
\end{bemerkung}
\begin{beweis}
  Nach \cref{thm:gleichheit-nat} ist der Typ $\N$ eine Menge und dieser ist äquivalent zu $\N_1$.
  Für $\Z'$ führen wir einen Beweis mit der üblichen Methode. Sei dazu $\mathrm{Eq}_{\Z'}:\Z'\to\Z'\to\mU$ gegeben durch:
  \begin{align*}
    \mathrm{Eq}_{\Z'}(0_{\Z'},0_{\Z'})&\colonequiv \eins \\
    \mathrm{Eq}_{\Z'}(\mathrm{pos}(n),\mathrm{pos}(k))&\colonequiv n=_{\N_1}k \\
    \mathrm{Eq}_{\Z'}(\mathrm{neg}(n),\mathrm{neg}(k))&\colonequiv n=_{\N_1}k \\
    \text{übrige Fälle }&\colonequiv \leer
  \end{align*}
  Den Reflexivitätsterm $r_{\Z'}:\prod_{x:\Z'}\mathrm{Eq}_{\Z'}(x,x)$ können wir definieren als $\ast$, im Fall $x\equiv 0$ und als $\refl_{n}$ in den Fällen $x\equiv \mathrm{pos}(n)$ und $x\equiv \mathrm{neg}(n)$.
  Nun ist etwa für $x\equiv \mathrm{pos}(n)$ zu zeigen, dass
  \begin{mathpar}
    \sum_{x:\Z'}\mathrm{Eq}_{\Z'}(\mathrm{pos}(n),x)
  \end{mathpar}
  kontrahierbar ist. Per Induktion über $x$ ist das nur noch für $x\equiv \mathrm{pos}(k)$ zu zeigen, da in allen anderen Fällen nur etwas für alle Elemente des leeren Typs zu zeigen ist.
  Im Fall $x\equiv \mathrm{pos}(k)$ müssen wir allerdings nur zeigen, dass ein Element $(k,p):\sum_{k:\N_1}n=k$ gleich $(n,\refl)$ ist - das können wir mit \cref{lem:pfade-kontrahierbar} zeigen.
\end{beweis}

\begin{bemerkung}
  Es gilt $\Z\simeq \Z'$. Wir schreiben daher ab jetzt $\Z$ für beide.
\end{bemerkung}
\begin{beweis}
  Übungsaufgabe.
\end{beweis}

Nun können wir uns den Überlagerungen zuwenden. Speziell werden wir mit der universellen Überlagerung des Kreises beginnen.
In der Topologie wird diese als Kopie der reellen Gerade $\R$ konstruiert, die als Helix über dem topologischen Kreis $\mathbb{S}^1=\{(x,y)\in\R^2\vert x^2+y^2=1\}$ liegt.
Als konkrete Projektion von $\R$ auf $\mathbb{S}^1$ kann etwa
\begin{mathpar}
  (t:\R) \mapsto (\cos(2\pi t),\sin(2\pi t))
\end{mathpar}
gewählt werden. Unser folgender abstrakter Nachbau dieser Situation handelt zwar nur von den Homotopietypen der beteiligten topologischen Räume, ist aber trotzdem überraschend ähnlich.
So werden etwa die Fasern der abstrakten Projektion ebenfalls $\Z$ sein.
Tatsächlich werden wir sogar genau damit beginnen.

Unser Ziel wird zunächst sein, zu beweisen, dass der Gleichheitstyp $\ast=_{S^1}\ast$ äquivalent zu $\Z$ ist.
Dazu geben wir diesen spezielleren Gleichheitstypen noch ihren üblichen Namen.

\begin{definition}
  \begin{enumerate}
  \item Ein \begriff{punktierter Typ} ist ein Paar $(A,a):\sum_{A:\mU}A$. Wir werden oft von punktierten Typen $A:\mU$ sprechen und implizit den Punkt mit $\ast:A$ bezeichnen.
  \item Der \begriff{Schleifenraum}\index{$\Omega$} eines punktierten Typs $(A,\ast)$ ist der Typ $\Omega(A,\ast)\colonequiv (\ast=_A\ast)$.
  \end{enumerate}
\end{definition}

\begin{definition}
  Die \begriff{universelle Überlagerung} von $S^1$ ist der abhängige Typ $\tilde{S^1}:S^1\to\mU$ gegeben durch
  \begin{mathpar}
    \tilde{S^1}\colonequiv \rec{S^1}(\mU, \Z, \ua({\mathrm{succ}}_\Z))
  \end{mathpar}
  Es gilt also $\tilde{S^1}(\ast)=\Z$ und $\mathrm{ap}(\tilde{S^1},l)=\ua(\mathrm{succ}_{\Z})$.
\end{definition}

Die letztere Gleichheit lässt sich etwas greifbarer machen und entspricht im topologischen Fall der Monodromieaktion eines Erzeugers der Fundamentalgruppe.


\begin{bemerkung}
  Es gilt $\transp_{\tilde{S^1}}(l)=_{\Z\to\Z}\mathrm{succ}_\Z$.
\end{bemerkung}
\begin{beweis}
  Allgemein gilt für einen abhängigen Typ $B:A\to\mU$:
  \begin{mathpar}
    \prod_{x,y:A}\prod_{p:x=y}\ua(\transp_B(p))=\mathrm{ap}(B,p)
  \end{mathpar}
  was sich per Induktion über $p$ beweisen lässt, da $\ua(\id_{B(x)})=\refl_{B(x)}$ gilt.
  Damit gilt also
  \begin{mathpar}
    \ua(\transp_{\tilde{S^1}}(l))=\mathrm{ap}(\tilde{S^1},l)=\ua(\mathrm{succ}_{\Z})
  \end{mathpar}
  womit die Behauptung folgt, weil $\ua$ eine Äquivalenz ist.
\end{beweis}

Das bedeutet, dass wir den Transport in $\tilde{S^1}$ benutzen können, um Gleichheiten der Form $\ast=_{S^1}\ast$, also Elemente des Schleifenraums $\Omega(S^1,\ast)$, mit $l$ zu vergleichen.
Da $l$ einmal den Kreis durchläuft, kann man mit der folgenden Definition messen, wie oft eine Gleichheit in $\Omega(S^1,\ast)$ den Kreis $S^1$ durchläuft.

\begin{definition}
  Die \begriff{Windungszahl} einer Gleichheit in $\Omega(S^1,\ast)$ ist der Wert unter der Abbildung
  \begin{align*}
    w&:\Omega(S^1,\ast)\to \Z \\
    w(p)&\colonequiv \transp_{\tilde{S^1}}(p)(0)
  \end{align*}
\end{definition}

Die Windungszahl wird sich später als Äquivalenz herausstellen und hat die Eigenschaften eines Gruppenhomomorphismus.
Wir zeigen das zunächst nur für Addition mit 1.

\begin{bemerkung}
  \label{bem:hom-windungszahl}
  Für alle $p:\Omega(S^1,\ast)$ gilt $w(p\kon l)=\mathrm{succ}_{\Z}(w(p))$.
\end{bemerkung}
\begin{beweis}
  Das ist durch die Verträglichkeit von Konkatenation und Komposition von Transporten gegeben:
  \begin{align*}
    w(p\kon l)&=\transp_{\tilde{S^1}}(p\kon l)(0) \\
              &=(\transp_{\tilde{S^1}}(l)\circ\transp_{\tilde{S^1}}(p))(0) \\
              &=\mathrm{succ}_\Z(\transp_{\tilde{S^1}}(p)(0)) \\
    &=\mathrm{succ}_\Z(w(p)) 
  \end{align*}
\end{beweis}

Wenn die Windungszahl eine Äquivalenz ist, bedeutet das, dass jedes $p:\Omega(S^1,\ast)$ von der Form $l^k$ für $k:\Z$ ist.
Also sollte die Inverse von $w$ durch $k$-fache Konkatenation von $l:\Omega(S^1,\ast)$ gegeben sein.

\begin{definition}
  Die \begriff{k-fache Konkatenation}\index{$\_^k$} $\_^k:\Z\to\Omega(S^1,\ast)$ ist $\Z'$- und $\N_1$-rekursiv gegeben durch
  \begin{align*}
    l^0&\colonequiv \refl_\ast \\
    l^{\mathrm{pos}(1)}&\colonequiv l \\
    l^{\mathrm{pos}(n+1)}&\colonequiv l^{\mathrm{pos}(n)}\kon l \\
    l^{\mathrm{neg}(1)}&\colonequiv l^{-1} \\
    l^{\mathrm{neg}(n+1)}&\colonequiv l^{\mathrm{neg}(n)}\kon l^{-1}
  \end{align*}
\end{definition}

Anstatt direkt zu zeigen, dass $\_^k$ und $w$ zueinander invers sind, werden wir eine faserweise Äquivalenz der Form
\begin{mathpar}
  \prod_{x:S^1}\ast=_{\tilde{S^1}}x\to \tilde{S^1}(x)
\end{mathpar}
konstruieren.

\begin{lemma}
  \label{lem:ext-windungszahl}
  Es gibt eine faserweise Abbildung 
  \begin{mathpar}
    W:\prod_{x:S^1}\ast=_{\tilde{S^1}}x\to \tilde{S^1}(x)
  \end{mathpar}
  mit:
  \begin{align*}
    W(\ast)\colonequiv w : \ast=\ast\to\Z 
  \end{align*}
\end{lemma}
\begin{beweis}
  Um einzusehen, dass $W$ durch $S^1$-Induktion definiert werden kann, müssen wir berechnen, wie die Abbildung $w$ entlang von $l:\ast=\ast$ transportiert wird.
  Dazu überlegen wir zunächst, was der Transport in einer abstrakteren Situation wäre, nämlich in einem abhängigen Typ der Form:
  \begin{mathpar}
    (x:A) \mapsto B(x)\to C(x)
  \end{mathpar}
  für zwei abhängige Typen $B:A\to\mU$ und $C:A\to\mU$. Die naheliegende Vermutung ist, dass wir mit den Transporten in $B$ und $C$ prä- und postkomponieren müssen.
  Es kann mit Induktion über $p$ geklärt werden, dass gilt:
  \begin{mathpar}
    \transp_{(x:A) \mapsto B(x)\to C(x)}(p)(f)=\transp_C(p) \circ f \circ \transp_B(p^{-1})
  \end{mathpar}
  Den Transport in $x\mapsto \ast=x$ kennen wir, das ist Rechtskonkatenation. Damit die induktive Definition von $W$ wie behauptet funktioniert,
  müssen wir also verifizieren, dass gilt:
  \begin{mathpar}
    \transp_{\tilde{S^¹}}(l) \circ w \circ \transp_{\ast=\_}(l^{-1}) = w
  \end{mathpar}
  Das lässt sich wie folgt punktweise für $p:\ast=\ast$ berechnen:
  \begin{align*}
    \transp_{\tilde{S^¹}}(l) \circ w \circ \transp_{\ast=\_}(l^{-1})(p) &= \transp_{\tilde{S^¹}}(l) \circ w(p\kon l^{-1}) \\
                                                                        &= \transp_{\tilde{S^¹}}(l) \circ \transp_{\tilde{S^1}}(p\kon l^{-1})(0) \\
                                                                        &= \transp_{\tilde{S^¹}}(l) \circ \transp_{\tilde{S^1}}(l^{-1})\circ \transp_{\tilde{S^1}}(p)(0) \\
                                                                        &=\transp_{\tilde{S^1}}(p)(0) \\
                                                                        &=w(p)
  \end{align*}
\end{beweis}

\begin{lemma}
  Es gibt auch eine Fortsetzung der Potenzfunktion $k\mapsto l^{k}:\Z\to\Omega(S^1,\ast)$, also eine faserweise Abbildung
  \begin{mathpar}
    P:\prod_{x:S^1}\tilde{S^1}(x)\to \ast=x
  \end{mathpar}
  mit $P_\ast=k\mapsto l^k$
\end{lemma}
\begin{beweis}
  Um $P$ induktiv definieren zu können, müssen wir zeigen:
  \begin{mathpar}
    \transp_{(x:S^1)\mapsto \tilde{S^1}(x)\to \ast=x}(l)(k\mapsto l^k)=(k\mapsto l^k)
  \end{mathpar}
  Analog zum beweis von \cref{lem:ext-windungszahl} ergibt sich:
  \begin{align*}
    \transp_{(x:S^1)\mapsto \tilde{S^1}(x)\to \ast=x}(l)(k\mapsto l^k) &=\transp_{\ast=\_}(l) \circ (k\mapsto l^k) \circ \transp_{\tilde{S^1}}(l^{-1}) \\
                                                                          &=(p\mapsto p\kon l) \circ (k\mapsto l^k) \circ \transp_{\tilde{S^1}}(l^{-1}) \\
                                                                          &=(k\mapsto l^{\mathrm{succ}_\Z(k)}) \circ \mathrm{succ}_\Z^{-1} \\
                                                                       &=(k\mapsto l^k)
  \end{align*}
\end{beweis}

\begin{theorem}
  Es gilt:
  \begin{enumerate}
  \item $W$ und $P$ sind faserweise zueinander invers.
  \item $w:\Omega(S^1,\ast)\to \Z$ ist eine Äquivalenz mit Inversem $k\mapsto l^{k}:\Z\to\Omega(S^1,\ast)$.
  \end{enumerate}
\end{theorem}
\begin{beweis}
  \begin{enumerate}
  \item   Wir rechnen punktweise nach. Sei $x:S^1$, dann soll für alle $p:\ast=x$ gelten:
  \begin{mathpar}
    P_x(W_x(p))=p
  \end{mathpar}
  Dank Induktion mit Basispunkt müssen wir das nur für $p\equiv \refl$ berechnen:
  \begin{mathpar}
    P_\ast(W_\ast(\refl))=P_\ast(0)=\refl
  \end{mathpar}
  Für die andere Richtung müssen wir zeigen, dass für alle $x:S^1$ und $k:\tilde{S^1}(x)$ gilt:
  \begin{mathpar}
    W_x(P_x(y))=k
  \end{mathpar}
  Das wollen wir mit Kreisinduktion zeigen. Zunächst berechnen wir also:
  \begin{mathpar}
    W_\ast(P_\ast(k))=W_\ast(l^{k})
  \end{mathpar}
  Wir sind mit dem Induktionsangfang fertig, wenn $W_\ast(l^k)\equiv w(l^k)=k$ gilt. Das können wir mit $\Z'$-Induktion und $\N_1$-Induktion zeigen:
  \begin{align*}
    w(l^0)&=\transp_{\tilde{S^1}}(\refl)(0)=0 \\
    w(l^{\mathrm{pos}(1)})&=w(l)=\mathrm{succ}_\Z(0)=\mathrm{pos}(1)  \\
    w(l^{\mathrm{pos}(n+1)})&=w(l^{\mathrm{pos}(n)}\kon l)= \mathrm{succ}_\Z(w(l^{\mathrm{pos}(n)}))=\mathrm{pos}(n+1) \\
    w(l^{\mathrm{neg}(1)})&=w(l^{-1})=\mathrm{succ}^{-1}(0)=\mathrm{neg}(1) \\
    w(l^{\mathrm{neg}(n+1)})&=w(l^{\mathrm{neg}(n)}\kon l^{-1})=\mathrm{succ}^{-1}(w(l^{\mathrm{neg}(n)}))=\mathrm{neg}(n+1)
  \end{align*}
  Nun ist noch zu zeigen, dass Transport der soeben konstruierten Gleichheit entlang $l:\ast=\ast$, wieder diese Gleichheit ist.
  Was auch immer wir genau konstruieren müssen, es ist auf jeden Fall eine Gleichheit zwischen Gleichheiten im Typ $\Z$.
  Da letzterer Typ eine Menge ist, existieren also alle Gleichheiten dieser Art.
\item In (a) für $x:S^1$ den Punkt $\ast$ einsetzen.
  \end{enumerate}
\end{beweis}

Die Eigenschaft der abhängigen Typen $\tilde{S^1}$ und $(x:S^1)\mapsto \ast=x$, dass alle Werte Mengen sind, ist bereits die definierende Eigenschaft von Überlagerungen.
Allerdings werden wir die Projektionen dieser abhängigen Typen Überlagerungen nennen.
Dabei sollte man bedenken, dass es sich hierbei nur um die Homotopietypen von Überlagerungen handelt.

\begin{definition}
  Sei $A:\mU$ ein Typ.
  \begin{enumerate}
  \item Ein abhängiger Typ $B:A\to \mU$ heißt $n$-abgeschnitten, wenn jedes $B(x)$ ein $n$-Typ, also $n$-abgeschnitten ist.
  \item Eine Abbildung heißt $n$-abgeschnitten, wenn alle ihre Fasern $n$-abgeschnitten sind.
  \item Eine Abbildung $f:A\to B$ heißt \begriff{Überlagerung}, wenn sie $0$-abgeschnitten ist.
  \end{enumerate}
\end{definition}

\subsection{Sphären, Homotopiegruppen und n-Abschneidungen}

\begin{definition}
  Seien $(A,\ast_A)$ und $(B,\ast_B)$ punktierte Typen.
  \begin{enumerate}
  \item Eine \begriff{punktierte Abbildung} oder \begriff{Abbildung punktierter Typen}\index{$\to^\ast$} ist eine Funktion $f:A\to B$ zusammen mit einer Gleichheit $p_f:f(\ast_A)=\ast_B$.
  Der Typ der punktierten Abbildungen ist:
  \begin{mathpar}
    (A,\ast_A)\to^\ast (B,\ast_B)\colonequiv  \sum_{f:A\to B}f(\ast_A)=\ast_B
  \end{mathpar}
  Wir werden auch $A\to^\ast B$ schreiben, wenn die Punktierung klar ist und etwa $(f,p_f):(A,\ast_A)\to^\ast (B,\ast_B)$, um der Gleichheit der punktierten Abbildung einen Namen zu geben.
\item Der \begriff{Typ der punktierten Typen} ist
  \begin{mathpar}
    \mU^\ast\colonequiv\sum_{A:\mU}A
  \end{mathpar}
\item Den Schleifenraum $\Omega(A,\ast_A)$ werden wir nun auch als punktierten Raum mit dem Punkt $\refl_{\ast_A}$ auffassen.
\item Für ein punktierte Abbildung $(f,p_f):(A,\ast_A)\to^\ast (B,\ast_B)$ ist
  \begin{mathpar}
    \Omega(f,p_f) \colonequiv (p:\ast_A =\ast_A) \mapsto p_f^{-1}\kon f(p) \kon p_f
  \end{mathpar}
\item Die Typen $\eins,\zwei,\N$ und $\Z$ sind jeweils durch $0$ punktiert. Alle Summen $\sum_{x:A}B(x)$ mit punktiertem $A$ und $B(\ast)$ sind punktiert und abhängige Produkte $\prod_{x:A}B(x)$ mit punktierten $B(x)$ sind punktiert.
  \end{enumerate}
\end{definition}

\begin{definition}[n-facher Schleifenraum]
  Für $n:\N$ und $(A,\ast_A)$ sei der \begriff{$n$-fache Schleifenraum} $\Omega^n(A,\ast_A)$ rekursiv gegeben durch:
  \begin{align*}
    \Omega^0&\colonequiv (A,\ast_A) \\
    \Omega^{n+1}&\colonequiv \Omega(\Omega^n(A,\ast_A))
  \end{align*}
  wobei wir Schleifenräume stets als punktiert auffassen.
\end{definition}

\begin{lemma}
  \label{lem:schleifenraum-n-typ}
  Sei $A:\mU$, dann sind für $n\geq -1$ äquivalent:
  \begin{enumerate}[label=\roman*)]
  \item $A$ ist ein $(n+1)$-Typ.
  \item Der Schleifenraum $\Omega(A,x)$ ist ein $n$-Typ für jedes $x:A$.
  \end{enumerate}
\end{lemma}
\begin{beweis}[Idee]
  Jeder Gleichheitstyp $x=_A y$ ist äquivalent zum Schleifenraum $\Omega(A,x)$, wenn es ein $p:x=_A y$ gibt.
  Tatsächlich reicht diese Aussage schon, denn für $n\geq -1$ und jedes $X$ gilt:
  \begin{mathpar}
    (X\to \isNType{n}(X))\to \isNType{n}(X)
  \end{mathpar}
\end{beweis}

\begin{definition}
  Seien $A,B,C:\mU$ und $f:A\to C$, $g:B\to C$. Es ist also ein Winkel gegeben:
  \begin{center}
    \begin{tikzcd}
      & B\arrow[d, "g"] \\
      A\arrow[r, "f", swap] & C
    \end{tikzcd}
  \end{center}
  \begin{enumerate}
  \item Sei $X:\mU$. Ein \begriff{$X$-Kegel}\index{Kegel} besteht aus Abbildungen $\varphi:X\to A$ und $\psi:X\to B$ zusammen mit einer Homotopie $H:f\circ \varphi \sim g\circ \psi$.
    Einen Kegel zusammen mit dem Winkel nennt man \begriff{Quadrat}:
    \begin{center}
      \begin{tikzcd}
        X\arrow[d,"\varphi"]\arrow[r,"\psi",swap] & B\arrow[d,"g"] \\
        A\arrow[r,"f",swap] & C 
      \end{tikzcd}
    \end{center}
      Wir schreiben $\mathrm{Cone(f,g,X)}\colonequiv\sum_{\varphi:X \to A}\sum_{\psi:X \to B}f\circ\varphi \sim g\circ\psi$ für den Typ der $X$-Kegel über dem gegebenen Winkel.
  \item Der Kegel mit Spitze
    \begin{mathpar}
      \mathrm{PB}(f,g)\colonequiv \sum_{x:A}\sum_{y:B}f(x)=g(y)
    \end{mathpar}
    und Projektionen $\pi_1:\mathrm{PB}(f,g)\to A$, $\pi_2\colonequiv \pi_1\circ \pi_2:\mathrm{PB}(f,g)\to B$ und der Homotopie $\pi_2\circ\pi_2:f\circ \pi_1 \sim g\circ \pi_2$ heißt \begriff{Pullback} von $f$ entlang $g$.
  \item Für einen $X$-Kegel $(\varphi,\psi,H)$ heißt die Abbildung
    \begin{mathpar}
      V(X,\varphi,\psi,H)\colonequiv (x:X)\mapsto (\varphi(x),\psi(x),H_x) : X \to \mathrm{PB}(f,g)
    \end{mathpar}
    \begriff{Vergleichsabbildung} und der $X$-Kegel zusammen mit dem Winkel heißt \begriff{Pullbackquadrat}, wenn $V(X,\varphi,\psi,H)$ eine Äquivalenz ist.
  \end{enumerate}
\end{definition}

\begin{bemerkung}
  Sei 
  \begin{center}
    \begin{tikzcd}
      & B\arrow[d, "g"] \\
      A\arrow[r, "f", swap] & C
    \end{tikzcd}
  \end{center}
  ein Winkel.
  \begin{enumerate}
  \item Der Pullback von $f$ und $g$ vervollständigt den Winkel zu einem Pullbackquadrat.
  \item Sei $A$ punktiert, dann ist der Schleifenraum wie folgt ein Pullback:
    \begin{center}
      \begin{tikzcd}
        \Omega(A,\ast)\ar[r]\ar[d] & \eins\ar[d, "\ast"] \\
        \eins\ar[r, "\ast", swap] & A
      \end{tikzcd}
    \end{center}
  \item Wenn $f$ punktiert ist, dann ist die Faser von $f$ wie folgt ein Pullback:
    \begin{center}
      \begin{tikzcd}
        f^{-1}(\ast)\ar[r]\ar[d] & A\ar[d, "f"] \\
        \eins\ar[r, "\ast", swap] & B
      \end{tikzcd}
    \end{center}
  \item Die Begriffe Quadrat und Pullbackquadrat sind invariant unter Ersetzen von äquivalenten Typen und homotopen Abbildungen.
  \end{enumerate}
\end{bemerkung}

\begin{lemma}
  Sei
  \begin{center}
    \begin{tikzcd}
      X\arrow[d,"\varphi"]\arrow[r,"\psi",swap] & B\arrow[d,"g"] \\
      A\arrow[r,"f",swap] & C 
    \end{tikzcd}
  \end{center}
  ein Quadrat mit Homotopie $H$, dann sind äquivalent:
  \begin{enumerate}
  \item Das Quadrat ist ein Pullbackquadrat, bzw. die Vergleichsabbildung $V(X,\varphi,\psi,H):X\to\mathrm{PB}(f,g)$ ist eine Äquivalenz.
  \item Für jedes $Y$ ist die Abbildung, die eine Funktion $\vartheta:Y\to X$ abbildet auf den $Y$-Kegel
    \begin{center}
    \begin{tikzcd}
      Y\arrow[d,"\varphi\circ \vartheta"']\arrow[r,"\psi\circ \vartheta"] & B\arrow[d,"g"] \\
      A\arrow[r,"f",swap] & C 
    \end{tikzcd}
    \end{center}
    mit Homotopie $H_\vartheta$, eine Äquivalenz.
  \item Jeder $Y$-Kegel $(\varphi',\psi',H')$ faktorisiert eindeutig über $X$. Das heißt, dass der Typ der Abbildungen $\vartheta:Y\to X$ zusammen mit Homotopien $h:\varphi\circ \vartheta\sim\varphi'$, $k:\psi\circ \vartheta\sim\psi'$ sowie $f(h)^{-1}\kon H_\vartheta\kon g(k) = H'$ kontrahierbar ist.
  \item Die induzierte faserweise Abbildung $\prod_{x:A}\varphi^{-1}(x)\to g^{-1}(f(x))$ ist eine faserweise Äquivalenz.
  \end{enumerate}
\end{lemma}

\begin{theorem}[Pullback-Pasting Teil 1]
  Seien
  \begin{center}
    \begin{tikzcd}
      Y\ar[r]\ar[d] & X\arrow[r]\arrow[d] & A\arrow[d] \\
      D\ar[r] & B\arrow[r] & C 
    \end{tikzcd}
  \end{center}
  jeweils Quadrate, es gebe also entsprechende Homotopien.
  \begin{enumerate}
  \item  Durch Komposition der Homotopien gibt es auch ein drittes Quadrat der Gestalt:
    \begin{center}
      \begin{tikzcd}
        Y\ar[rr]\ar[d] &  & A\arrow[d] \\
        D\ar[rr] &  & C 
      \end{tikzcd}
    \end{center}
  \item Wenn die beiden kleinen Quadrate Pullbacks sind, dann ist auch das dritte, zusammengesetzte ``Quadrat'' ein Pullback.
  \end{enumerate}
\end{theorem}
\begin{beweis}
  Man benutzt die Äquivalenz zu faserweisen Äquivalenzen. Dann ergibt sich die Aussage durch punktweises Anwenden von 2-aus-3 für Äquivalenzen.
\end{beweis}

\begin{theorem}[Fasersequenz]
  Sei $(f,p_f):(A,\ast)\to^\ast (B,\ast)$ eine punktierte Abbildung.
  Wir verzichten hier auf Angabe der Punktierung und verwenden die Schreibweise $-\Omega f\colonequiv p\mapsto f(p^{-1})$.
  Dann gibt es eine Sequenz:
  \begin{center}
    \begin{tikzcd}
      \Omega A \ar[r,"-\Omega f"] & \Omega B \ar[r] & f^{-1}(\ast) \ar[r,"\pi_1"] & A \ar[r,"f"] & B
    \end{tikzcd}
  \end{center}
  wobei bei aufeinanderfolgenden Abbildungen stets die linke die Faser der rechten ist - bis auf Äquivalenz.
  Weiter ist die Abbildung $\Omega B\to f^{-1}(\ast)$ gegen durch: $(\ast,p_f\kon\_)$.
\end{theorem}
\begin{beweis}
  Betrache:
  \begin{center}
    \begin{tikzcd}
      \mathrm{PB}(\ast,\pi)\ar[r]\ar[d] & \eins\ar[d,"\ast"] \\
      f^{-1}(\ast) \ar[r,"\pi"]\ar[d] & A \ar[d,"f"] \\
      \eins\ar[r,"\ast",swap] & B
    \end{tikzcd}
  \end{center}
  Durch Pullback-Pasting wissen wir, $\mathrm{PB}(\ast,\pi)$ auch Pullback des zusammengesetzen Quadrats ist.
  Aber auch $\Omega B$, oder etwas genauer $\ast=f(\ast)$ ist Pullback des zusammengesetzen Quadrats, also können wir $\mathrm{PB}(\ast,\pi)$ äquivalent ersetzen durch $\Omega B$.
  Wir müssen allerdings etwas genauer hinschauen, um zu sehen, was für eine Abbildung $\Omega B\to f^{-1}(\ast)$ wir bekommen.
  Zunächst ist der Pullback:
  \begin{mathpar}
    \mathrm{PB}(\ast,\pi_1)\equiv \left(\sum_{y:\sum_{x:A}f(x)=\ast}\pi_1(y)=\ast\right)\simeq \sum_{x:A} (f(x)=\ast)\times (x=\ast)
  \end{mathpar}
  Letzterer Typ ist nun äquivalent zu $\ast=f(\ast)$ durch: $(x,p,q)\mapsto p^{-1}\kon f(q)$, da $p^{-1}\kon f(q)$ der Wert der zusammengesetzten Homotopie im großen Quadrat ist.
  Die Projektion von $\ast=f(\ast)$ nach $f^{-1}(\ast)$ muss also $(r:\ast=f(\ast))\mapsto (\ast,r^{-1})$ sein. Die Äquivalenz zu $\ast=\ast$ bzw. $\Omega B$ ist durch die Punktierung $p_f:f(\ast)=\ast$ von $f$ gegeben.
  Die gesuchte Abbildung ist also $(p:\Omega B)\mapsto (\ast,p\kon p_f^{-1})$.
  
  Im so bestimmten Diagram nehmen wir wieder einen Pullback:
  \begin{center}
    \begin{tikzcd}
      \mathrm{PB}(\dots,\ast)\ar[r]\ar[d] & \Omega B\ar[r]\ar[d] & \eins\ar[d,"\ast"] \\
      \eins\ar[r] &f^{-1}(\ast) \ar[r,"\pi_1"]\ar[d] & A \ar[d,"f"] \\
      & \eins\ar[r,"\ast",swap] & B
    \end{tikzcd}
  \end{center}
  Genauer ist der neue Pullback: $\mathrm{PB}((\ast,p_f\kon\_),\ast)\equiv\sum_{p:\Omega B}(\ast,p_f\kon p)=(\ast,p_f)$.
  Durch waagrechtes Pullback-Pasting muss dieser Pullback allerdings äquivalent zu $\Omega A$ sein.
\end{beweis}

\begin{regeln}[Einhängung]
  Zu jedem Typ $A:\mU$ gibt es einen Typen $\Sigma A:\mU$, die \begriff{Suspension}\index{Sigma} oder \begriff{Einhängung} von $A$.
  Die Einhängung ist ein höherer induktiver Typ mit folgenden Konstruktoren:
  \begin{align*}
    N&:\Sigma A \\
    S&:\Sigma A \\
    m&:A\to N=_{\Sigma A}S
  \end{align*}
  $\Sigma A$ ist durch $N$ punktiert.
\end{regeln}

\begin{bemerkung}
  Es gilt: $(\Sigma\zwei)\simeq S^1$
\end{bemerkung}
\begin{beweis}
  Wird noch nachgetragen.
\end{beweis}

\begin{definition}
  Die (induktive) \begriff{$n$-Sphäre}\index{$S^n$} ist rekursiv gegeben durch
  \begin{align*}
    S^0&\colonequiv \zwei \\
    S^{n+1}&\colonequiv \Sigma S^n
  \end{align*}
  Da $\zwei$ punktiert ist und $\Sigma$ punktierte Typen erhält, können wir $S^n$ als punktierten Typen auffassen.
\end{definition}

\begin{regeln}
  Sei $n\geq -1$.
  Die $n$-Abschneidung eines Typen $A$ ist ein höherer induktiver Typ $\|A\|_n:\mU$ gegen durch die folgenden Konstruktoren:
  \begin{align*}
    |\_|_n&:A\to \|A\|_n \\
    \mathrm{Narbe}&:(S^{n+1}\to \|A\|_n)\to \|A\|_n \\
    \mathrm{Speiche}&:\prod_ {s:S^{n+1}\to \|A\|_n}\prod_{x:S^{n+1}} \mathrm{Narbe}(s)=s(x)
  \end{align*}
\end{regeln}

\begin{theorem}
  \label{thm:n-truncation}
  Sei $A:\mU$, dann gilt:
  \begin{enumerate}
  \item $\|A\|_n$ ist ein $n$-Typ.
  \item Für $n$-abgeschnittenes $B:\|A\|_n\to\mU$ und $s_0:\prod_{a:A}B(|a|_n)$ gibt es $s:\prod_{x:\|A\|_n}B(x)$ mit $s(|a|_n)\equiv s_0(a)$.
  \end{enumerate}
\end{theorem}

Wir werden den Beweis in mehreren Schritten führen, die auch für sich genommen interessante Aussagen sind.
Zentral für viele Dinge, ist dabei der folgende Satz:
\begin{theorem}
  \label{thm:adjunktion-schleifenraum}
  Seien $X$ und $Y$ punktierte Typen, dann gibt es eine Äquivalenz:
  \begin{mathpar}
    (\Sigma X \to^\ast Y) \simeq (X \to^\ast \Omega Y)
  \end{mathpar}
\end{theorem}
\begin{beweis}
  Die Äquivalenz lässt sich, unter anderem durch $\Sigma$-Rekursion auf die folgende Lücke ``??'' reduzieren:
  \begin{align*}
    &\Sigma X \to^\ast Y \\
    \equiv& \sum_{f:\Sigma X\to Y}f(N) = \ast \\
    \simeq & \sum_{\sum_{y_N,y_S:Y}X\to (y_N=y_S)} y_N=\ast \\
    \simeq & \sum_{y_N,y_S:Y}(X\to (y_N=y_S))\times (y_N=\ast) \\
    \simeq & \sum_{\tilde{y}:\left(\sum_{y_N:Y}y_N=\ast\right)}\sum_{y_S:Y}(X\to (\pi(\tilde{y})=y_S)) \\
    \simeq & \sum_{y_S:Y}(X\to (\ast=y_S)) \\
    ?? &  \\
    \simeq & \sum_{f:X\to \ast=_Y \ast}f(\ast)=\refl_\ast \\
    \simeq & \sum_{f:X\to \Omega Y}f(\ast)=\refl_\ast \\
    \simeq & X\to^\ast \Omega Y
  \end{align*}
  Nun schließen wir die Lücke. Seien also $y_S:Y$ und $f:X\to \ast=y_S$.
  Dann ist
  \begin{mathpar}
    \tilde{f}\colonequiv (x:X)\mapsto f(x) \kon f(\ast)^{-1} : X\to \ast=_Y \ast
  \end{mathpar}
  und es gilt $\tilde{f}(\ast)=\refl_\ast$. Wir haben also eine Abbildung gefunden.
  Sei nun $f:X\to \ast=_Y \ast$ mit $f(\ast)=\refl$. Dann ist
  \begin{mathpar}
    (\ast,f): \sum_{y_S:Y}(X\to (\ast=y_S))
  \end{mathpar}
  Um zu sehen, dass die Abbildungen invers zueinander sind, müssen wir also noch zeigen:
  \begin{mathpar}
    (y_S,f)=(\ast,\tilde{f})
  \end{mathpar}
  Dazu stellen wir fest, dass der Transport entlang $f(\ast)$ in $X\to(\ast=\_)$ punktweise Konkatenation mit $f(\ast)^{-1}$ ist.
  Wegen $\transp_{X\to(\ast=\_)}(f(\ast))(f)=\tilde{f}$ sind wir also fertig.

  Um den Beweis abzuschließen, bleibt also noch zu zeigen, dass für jedes $f:X\to\ast=\ast$ und $q:f(\ast)=\refl_\ast$ gilt:
  \begin{mathpar}
    (f,q)=(\tilde{f},\dots)
  \end{mathpar}
  wobei ``$\dots$'' eine Gleichheit ist, die durch Gruppoidrechengesetze gegeben ist.
  Zunächst können wir $q$ benutzen, um eine Homotopie zwischen $f$ und $\tilde{f}$:
  \begin{mathpar}
    \mathrm{ap}(f(x)\kon \_^{-1}, q) : f(x)\kon f(\ast)^{-1}=f(x)\kon \refl_\ast
  \end{mathpar}
  Bis auf Gruppoidrechengesetze liefert das die benötigte Homotopie.
  Mit dieser Homotopie haben wir auch eine Gleichheit $s:\tilde{f}=f$.
  Entlang dieser können wir im abhängigen Typ $g\mapsto g(\ast)=\refl_\ast$ transportieren.
  Als Transport erhalten wir:
  \begin{mathpar}
    \transp_{\_(\ast)=\refl_\ast}(s)=\mathrm{ap}(\_(\ast),s)^{-1}\kon\_
  \end{mathpar}
  Damit die Paare oben gleich sind, muss also gelten: $(\mathrm{ap}(\_(\ast),s)^{-1}\kon\_)(\dots)=q$

  Zunächst bekommen wir: $\mathrm{ap}(\_(\ast),s)=\mathrm{ap}(f(\ast)\kon\_^{-1},q)\kon\dots$, wobei wir wieder ``$\dots$'' für Gruppoidgesetze schreiben.
  Um Induktion über $q$ anwenden zu können, abstrahieren wir die Situation leicht und verwenden statt $f(\ast)$ ein beliebiges $p:\ast=\ast$.
  Dafür ergibt sich:
  \begin{mathpar}
    \dots\kon\mathrm{ap}(p\kon \_^{-1},q)\kon\dots=q^{-1}
  \end{mathpar}
  In der Situation oben berechnet sich der Wert des Transports entlang $s$ also wie gewünscht zu $q$.
\end{beweis}

Damit können wir sehen, dass Abbildungen $S^n\to A$ nichts anderes sind als iterierte Schleifen:

\begin{korollar}
  \label{kor:schleifenraum-abbildungen}
  Für punktiertes $A$ und jedes $n:\N$ gilt:
  \begin{mathpar}
    (S^n\to^\ast A)\simeq \Omega^n A
  \end{mathpar}
\end{korollar}
\begin{beweis}
  Wir beweisen per Induktion über $n$.
  Für $n\equiv 0$ müssen wir folgende Äquivalenz zeigen:
  \begin{mathpar}
    (\zwei\to^\ast A)\simeq A
  \end{mathpar}
  Da es sich um punktierte Abbildunge handelt, ist die linke Seite nichts anderes als $(\eins\to A)\simeq A$.
  Für den Induktionsschritt verwenden wir \cref{thm:adjunktion-schleifenraum}:
  \begin{align*}
    (S^{n+1}\to^\ast A)&\equiv (\Sigma S^n\to^\ast A) \\
    &\simeq (S^n\to^\ast \Omega A) \\
    &\simeq \Omega^n(\Omega A)
  \end{align*}
\end{beweis}


In \cref{lem:schleifenraum-n-typ} hatten wir gesehen, dass $n+1$-Typen genau die Typen mit $n$-Typen $\Omega(A,x)$.
Das funktioniert in eine Richtung sogar für iterierte Schleifenräume:
\begin{lemma}
  Sei $n\geq -1$. Ein Typ $A$ ist genau dann ein $n$-Typ, wenn für alle $x:A$ der Typ $\Omega^{n+1}(A,x)$ kontrahierbar ist.
\end{lemma}
\begin{beweis}
  Sei $n\equiv -1$. Dann ist zu zeigen: $A$ ist eine Aussage, wenn $A$ kontrahierbar ist. Andererseits kann mit $x:A$ gezeigt werden, dass die Aussage $A$ auch kontrahierbar ist.
  
  Für den Induktionsschritt müssen wir zeigen: $A$ ist genau dann ein $(n+1)$-Typ, wenn $\Omega^{n+2}(A,x)$ für alle $x:A$ kontrahierbar ist.
  Nach \cref{lem:schleifenraum-n-typ} reicht es zu zeigen, dass die Kontrahierbarkeit der $\Omega^{n+2}(A,x)$ äquivalent dazu ist, dass jedes $\Omega(A,x)$ ein $n+1$ Typ ist.
  Nach der Induktionshypothese ist dieser Typ genau dann ein $(n+1)$-Typ, wenn
  $\Omega^{n+1}(\Omega(A,x),p)$ für jedes $p:\Omega(A,x)$ kontrahierbar ist.

  Wir sind also fertig, wenn wir zeigen können: $\Omega^{n+1}(\Omega(A,x),p)\simeq \Omega^{n+2}(A,x)$.
  Es sollen also die Schleifen $p=p$ äquivalent sein zu den Schleifen $\refl_\ast=\refl_\ast$.
  Nun ist $\_\kon p^{-1}$ eine Äquivalenz und damit auch:
  \begin{mathpar}
    \mathrm{ap}(\_\kon p^{-1},\_):p=p\to p\kon p^{-1} = p\kon p^{-1}
  \end{mathpar}
  Letzteres ist durch Einsatz der Gruppoidgesetze äquivalent zu $\refl_\ast=\refl_\ast$.
\end{beweis}
\begin{korollar}
  \label{kor:n-abgeschnitten-kontrahierbarkeit}
  Sei $n\geq -1$. Ein Typ $A$ ist genau dann ein $n$-Typ, wenn für alle $x:A$, der Typ $(S^{n+1},\ast)\to^\ast(A,x)$ kontrahierbar ist.
\end{korollar}
\begin{beweis}
  Folgt aus dem Lemma und \cref{kor:schleifenraum-abbildungen}.
\end{beweis}

\begin{beweis}[\cref{thm:n-truncation}]
  \begin{enumerate}
  \item Per Konstruktion ist der Raum der punktierten Abbildungen $S^{n+1}\to^\ast\|A\|_n$ kontrahierbar: Als Kontraktionszentrum wählen wir die konstante Abbildung $c_\ast\colonequiv x\mapsto \ast $, die durch $\refl_\ast$ punktiert ist.
    Für eine punktierte Abbildung $(r,p_r):(S^{n+1})\to^{\ast}(\|A\|_n,\ast)$ müssen wir zunächst eine Homotopie $r\sim c_\ast$ finden.
    Für $x:S^{n+1}$ sei diese Homotopie gegeben durch
    \begin{mathpar}
      \mathrm{Speiche}(x)\kon \mathrm{Speiche}(\ast)^{-1}\kon p_r : r(x)=c_\ast(x)
    \end{mathpar}
    Für eine Gleichheit von punktierten Abbildungen müssen wir nun noch feststellen:
    \begin{mathpar}
      (\mathrm{Speiche}(\ast)\kon \mathrm{Speiche}(\ast)^{-1}\kon p_r)^{-1}\kon p_r=\refl_\ast
    \end{mathpar}
  \item Zunächst betrachten wir das Induktionsprinzip für $ \|A\|_n$. Sei also $B:\|A\|_n\to \mU$. Um $s:\prod_{x:\|A\|_n}B(x)$ induktiv zu konstruieren, brauchen wir folgende Daten:
    \begin{itemize}
    \item $s_0:\prod_{a:A}B(|a|_n)$ 
    \item Für alle $r:S^{n+1}\to \|A\|_n$ und $r':\prod_{x:S^{n+1}}B(r(x))$ ein $N_{r,r'}:B(\mathrm{Narbe(r)})$.
    \item Für alle $r:S^{n+1}\to \|A\|_n$, $r':\prod_{x:S^{n+1}}B(r(x))$ und jedes $x:S^{n+1}$ eine abhängige Gleichheit $r(x)=_{\mathrm{Speiche}(x)}^B N_{r,r'}$.
    \end{itemize}
    Die Daten für Narben und Speichen müssen wir also aus der $n$-Abgeschnittenheit von $B$ herleiten, um das gewünschte Induktionprinzip zu erhalten.
    Narbe und Speichen in $B(x)$  erhalten wir aber genau aus der $n$-Abgeschnittenheit der $B(x)$ und \cref{kor:n-abgeschnitten-kontrahierbarkeit}.
    
  \end{enumerate}
\end{beweis}

\begin{bemerkung}
  Wenn $A$ ein $n$-Typ ist, dann ist $|\_|_n:A\to \|A\|_n$ eine Äquivalenz.
\end{bemerkung}
\begin{beweis}
  Übungsblatt 12.
\end{beweis}

\begin{definition}
  \begin{enumerate}
  \item Eine \begriff{Gruppe} ist ein 0-Typ $G$ zusammen mit:
    \begin{itemize}
    \item $\_\cdot\_:G\to G\to G$
    \item $\_^{-1}:G\to G$
    \item $e:G$
    \end{itemize}
    und Gleichheiten:
    \begin{itemize}
    \item $\prod_{x,y,z:G}(x\cdot y)\cdot z=x\cdot (y\cdot z)$
    \item $\prod_{x:G}x\cdot x^{-1}=e=x^{-1}\cdot x$
    \item $\prod_{x:G}x\cdot e =x=e\cdot x$
    \end{itemize}
  \item Seien $G$ und $H$ Gruppen mit Verknüpfungen $\_\cdot_G\_$ und  $\_\cdot_H\_$. Eine Abbildung $f:G\to H$ ist ein \begriff{Gruppenhomomorphismus}, wenn es
    \begin{mathpar}
      \prod_{x,y:G}f(x\cdot_G y)=f(x)\cdot_H f(y)
    \end{mathpar}
    gibt. 
  \end{enumerate}
\end{definition}

\begin{definition}
  Sei $(A,\ast)$ punktiert und $n:\N$. Die $n$-te \begriff{Homotopiegruppe} von $A$ ist der Typ
  \begin{mathpar}
    \pi_n(A,\ast)\colonequiv \|\Omega^n(A,\ast)\|_0
  \end{mathpar}
  Für einen weiteren punktierten Typen $(B,\ast)$ und $(f,p_f):(A,\ast)\to^\ast(B,\ast)$ ist
  \begin{mathpar}
    \pi_n(f,p_f)\colonequiv \|\Omega^n(f,p_f)\|_0
  \end{mathpar}
\end{definition}

\begin{beispiel}
  Wir wissen: $\pi_1(S^1)\simeq \Z$.
\end{beispiel}

\begin{bemerkung}
  Für einen punktierten Typ $(A,\ast)$ und $n\geq 1$ ist $\pi_n(A,\ast)$ stets eine Gruppe.
  Für einen weiteren punktierten Typen $(B,\ast)$ und $(f,p_f):(A,\ast)\to^\ast(B,\ast)$ ist $\pi_n(f,p_f)$ ein Gruppenhomomorphismus.
\end{bemerkung}

\begin{definition}
  Seien $A,B$ punktiert und $f:A\to B$ eine punktierte Abbildung.
  \begin{enumerate}
  \item Das \begriff{Bild} von $f$ ist der folgende Typ:
    \begin{mathpar}
      \mathrm{im}(f)\colonequiv \sum_{b:B}\|\sum_{x:A}f(x)=b\|
    \end{mathpar}
  \item Der \begriff{Kern} von $f$ ist der folgende Typ:
    \begin{mathpar}
      \mathrm{Kern}(f)\colonequiv \sum_{x:A}f(x)=\ast
    \end{mathpar}
  \end{enumerate}
\end{definition}

\begin{theorem}
  Sei $f:A\to^{\ast}B$ eine punktierte Abbildung. Dann gibt es eine lange exakte Sequenz von punktierten $0$-Typen:
  \begin{center}
    \begin{tikzcd}
      \dots\ar[r] & \pi_{n+1}(F)\ar[r] & \pi_{n+1}(A)\ar[r,"(-1)^{n+1}\pi_{n+1}(f)"] & \pi_{n+1}(B)\ar[r] & \pi_n(F)\ar[r] & \pi_n(A)\ar[r,"(-1)^n\pi_n(f)"] & \pi_n(B)\ar[r] & \dots
    \end{tikzcd}
  \end{center}
  
\end{theorem}
\begin{beweis}
  Lassen wir hier aus. 
\end{beweis}
\begin{bemerkung}
  Man kann auch zeigen, dass es sich für $n\geq 2$ um eine lange exakte Sequenz von abelschen Gruppen handelt.
  Im Abschnitt mit $n=1$ ist es eine exakte Sequenz von Gruppen, allerdings im Allgemeinen auch mit Antihomomorphismen.
\end{bemerkung}


\subsection{Hopf-Faserung}

\begin{definition}
  Ein Typ $A$ heißt \begriff{$n$-zusammenhängend}, wenn $\|A\|_n$ kontrahierbar ist.
  Im Fall $n=0$ sagt man auch nur zusammenhängend.
\end{definition}

\begin{bemerkung}
  $S^1$ ist $0$-zusammenhängend und allgemeiner kann man zeigen: $S^{n+1}$ ist $n$-zusammenhängend.
\end{bemerkung}
\begin{beweis}
  Übungsblatt 12.
\end{beweis}

Wir werden sehen+glauben: Es gibt eine Abbildung $S^3\to S^2$ mit Faser $S^1$.
Daraus kann man mit der Fasersequenz interessante Schlüsse ziehen.

\begin{definition}
  Ein \begriff{H-Raum} ist ist ein punktierter Typ $A$ mit
  \begin{itemize}
  \item einer Operation $\mu:A\to A\to A$
  \item Homotopien $\mu(\ast,\_)\sim \id$ und $\mu(\_,\ast)\sim\id$
  \end{itemize}
\end{definition}

\begin{lemma}
  Sei $A$ ein zusammenhängender H-Raum. Dann sind die Abbildungen $\mu(x,\_)$ und $\mu(\_,x)$ für alle $x:A$ Äquivalenzen.
\end{lemma}
\begin{beweis}
  Da wir zeigen wollen, dass etwas eine Äquivalenz für alle $x:A$ ist, wollen wir ein abhängige Funktion in einem $-1$-abgeschnittenen abhängigen Typen $P:A\to\mU$ konstruieren.
  Äquivalen können wir also auch $P:A\to (-1)\hbox{-}\mathrm{Type}$ konstruieren.
  Da $(-1)\hbox{-}\mathrm{Type}$ nach \cref{bem:level-of-n-typ} ein $0$-Typ ist, faktorisiert $P$ über den kontrahierbaren Typ $\|A\|_0$,
  d.h. wir haben $P':\|A\|_0\to (-1)\hbox{-}\mathrm{Type}$ mit $P'\circ |\_|_0=P$.
  Um $P'$ zu zeigen, reicht es das etwa für $|\ast|_0$ zu machen. Aber $P'(|\ast|_0)$ ist einfach die Aussage, dass $\mu(\ast,\_)$ und $\mu(\_,\ast)$ Äquivalenzen sind und das stimmt, weil sie homotop zur Identität sind.
\end{beweis}

\begin{definition}
  Für einen zusammenhängenden H-Raum $A$ ist die \begriff{Hopf-Konstruktion} gegeben als der abhängige Typ $H:\Sigma A\to\mU$ mit:
  \begin{align*}
    H(N)&\colonequiv A \\
    H(S)&\colonequiv A \\
    H(m(a))&:= \ua(\mu(a,\_))
  \end{align*}
\end{definition}

\begin{beispiel}
  Auf $S^1$ gibt es eine Gleichheit $p:\id=\id$, gegeben durch eine Homotopie $H$ festgelegt durch $H(\ast)\colonequiv l$ und Nachrechnen des entsprechenden Transports.
  Auf $S^1$ können wir eine H-Raum-Struktur rekursiv definieren:
  \begin{align*}
    \mu(\ast,\_)&\colonequiv \id \\
    \mu(l,\_)&:=  p
  \end{align*}
  Das entspricht der komplexen Multiplikation auf $S^1$ und definiert eine H-Raum Struktur auf $S^1$.
  Der zugehörige abhängige Typ $H_{S^1}:\Sigma S^1\to\mU$ heißt \begriff{Hopf-Faserung}.
\end{beispiel}

\begin{fakt}
  Es gilt $\sum_{x:\Sigma S^1}H_{S^1}(x)\simeq S^3$.
\end{fakt}

\begin{theorem}
  Es gilt:
  \begin{enumerate}
  \item $\Omega^3S^3\simeq \Omega^3S^2$
  \item $\pi_2(S^2)\simeq \Z$
  \end{enumerate}
\end{theorem}
\begin{beweis}[Ansatz]
  Das lässt sich anhand der Fasersequenz für die Hopf-Faserung $S^1\to S^3\to S^2$ erkennen.
\end{beweis}

\subsection{Eilenberg-MacLane Räume}
Das Thema dieses Abschnitts wird im HoTT-Buch nur sehr knapp erwähnt und kann im Artikel ``Eilenberg-MacLane Spaces in Homotopy Type Theory'' von Dan Licata und Eric Finster nachgelesen werden.

Zu einem punktierten Typ $A$ gibt es eine Sequenz von Gruppen, die Homotopiegruppen $\pi_n(A,\ast)$ für $n\geq 1$.
Damit liegt die Frage nahe, ob es auch für jede Sequenz von Gruppen $(G_i)_{i\geq 1}$ einen punktierten Typen $A$ mit $\pi_i(A)\simeq G_i$ gibt.
Mit der verständlichen Einschränkung, dass alle $G_i$ für $i\geq 2$ abelsch sind, lässt sich diese Frage positiv beantworten.
Wir werden hier zumindest sehen, dass es für jede Gruppe $G$ einen Typ $K(G,1)$ gibt, für den gilt:
\begin{align*}
  \pi_0(K(G,1))&\simeq \eins \\
  \pi_1(K(G,1))&\simeq G \\
  \pi_n(K(G,1))&\simeq \eins\quad\text{ für $n\geq 2$}
\end{align*}
Wir geben direkt einen höheren induktiven Typen an, der das Problem weitgehend löst:
\begin{regeln}
  Sei $G$ eine Gruppe. Dann ist $\widetilde{K(G,1)}$ der höhere induktive Typ mit Konstruktoren:
  \begin{align*}
    \ast&:\widetilde{K(G,1)} \\
    \mathrm{eq}&:\prod_{g:G}\ast=_{\widetilde{K(G,1)}}\ast \\
    \mathrm{comp}&:\prod_{g,h:G} \mathrm{eq}(g\cdot h)=\mathrm{eq}(g)\kon \mathrm{eq}(h)
  \end{align*}
\end{regeln}

\begin{definition}
  Der (erste) \begriff{Eilenberg-MacLane Raum}\index{$K(G,1)$} zu einer Gruppe $G$ ist der punktierte Typ
  \begin{mathpar}
    K(G,1)\colonequiv \|\widetilde{K(G,1)}\|_1
  \end{mathpar}
\end{definition}

\begin{bemerkung}
  $\|\ast=_{\widetilde{K(G,1)}}\ast\|_0=\Omega K(G,1)$
\end{bemerkung}
\begin{beweis}
  Ohne Beweis.
\end{beweis}

\begin{fakt}
  Es gibt eine Äquivalenz
  \begin{mathpar}
    \rho:\Omega K(G,1) \simeq G
  \end{mathpar}
  mit $\rho(|\mathrm{eq}(g)|_1)=g$ für alle $g:G$.
\end{fakt}

\subsection{Kohomologie}
Die Inhalte in diesem Abschnitt kann man teilweise im Artikel ``Cellular Cohomology in Homotopy Type Theory'' und in Evan Cavallos Masterarbeit nachlesen.
Für jede ablesche Gruppe $A$ gibt es eine Folge von punktierten, Typen
\begin{mathpar}
  A, BA, B^2A, B^3A, \dots
\end{mathpar}
sodass $\Omega (B^{n+1}A)=B^nA$ gilt und $B^{n+1}A$ $n$-zusammenhängend ist.
\begin{definition}
  Für einen Typ $X$ und eine abelsche Gruppe $A$ ist die n-te \begriff{Kohomologiegruppe} von $X$ mit Koeffizienten in $A$ gegeben als:
  \begin{mathpar}
    H^n(X,A)\colonequiv \| X\to B^n A\|_0
  \end{mathpar}
\end{definition}
\begin{beispiel}
  \begin{enumerate}
  \item Für den Typ $\eins$ gilt: $H^0(\eins,A)\simeq A$ und $H^{n+1}(\eins,A)=\|B^{n+1}A\|_0\simeq\eins$, da $B^{n+1}A$ stets 0-zusammenhängend ist.
  \item Für $\zwei$ gilt: $H^0(\zwei,A)\simeq A\times A$ und $H^{n+1}(\zwei,A)=\|B^{n+1}A\|_0\times \|B^{n+1}A\|_0\simeq\eins$
  \item Für $A\equiv \Z$ und $X\equiv S^1$ gilt: $H^1(S^1,\Z)=\Z$ (mit etwas Rechnen).
  \end{enumerate}
\end{beispiel}

\begin{bemerkung}
  Sei $f:X\to Y$.
  Für festes $n:\N$ ist $H^n$ funktoriell, es gibt also
  \begin{mathpar}
    f^\ast:H^n(Y,A)\to H^n(X,A)
  \end{mathpar}
  und für $g:Y\to Z$ gilt:
  \begin{mathpar}
    f^\ast \circ g^\ast = (g\circ f)^\ast
  \end{mathpar}
\end{bemerkung}

\begin{lemma}
  Für punktierte Typen $A$ und $B$ gilt:
  \begin{mathpar}
    (A\to \Omega B)\simeq \Omega (A \to B)
  \end{mathpar}
  wobei $A\to B$ durch die Abbildung $\_\mapsto \ast$ punktiert ist.
\end{lemma}
\begin{beweis}
  \begin{align*}
    (A\to \Omega B) &\simeq (A\to \ast=_B \ast)\\
                    &\simeq (\_\mapsto \ast)\sim (\_\mapsto \ast) \\
                    &\simeq (\_\mapsto \ast) = (\_\mapsto \ast)\\
                    &\simeq \Omega (A\to B)
  \end{align*}
\end{beweis}


\begin{bemerkung}
  Alle Kohomologiegruppen sind abelsch.
\end{bemerkung}
\begin{beweis}
  $H^n(X,A)=\|X\to B^nA\|_0=\|X\to \Omega^2 B^{n+2}A\|_0=\|\Omega^2 (X\to  B^{n+2}A)\|_0$
\end{beweis}

Auch hinter der Einhängung steht eine allgemeinere Konstruktion, der sogenannte Pushout:

\begin{regeln}[Pushout]
  Seien $A,B,C:\mU$ und $f:C\to A$, $g:C\to B$, also folgenden Situation gegeben:
  \begin{center}
    \begin{tikzcd}
      C\ar[r,"f"]\ar[d,"g"] & A \\
      B & 
    \end{tikzcd}
  \end{center}
  Dann gibt es einen Typen $A\amalg_C B\colonequiv \mathrm{PO}(f,g):\mU$, den Pushout von $f$ und $g$, der der höhere induktive Typ mit folgenden Konstruktoren ist:
  \begin{align*}
    \iota_1&:A\to \mathrm{PO}(f,g) \\
    \iota_2&:B\to \mathrm{PO}(f,g) \\
    \mathrm{glue}&:\prod_{c:C}\iota_1(f(c))=\iota_2(g(c))
  \end{align*}
\end{regeln}

\begin{bemerkung}
  Sei $A:\mU$, dann gilt: $\Sigma A\simeq \mathrm{PO}((a:A)\mapsto \ast, (a:A)\mapsto \ast)$.
\end{bemerkung}

Durch die Rekursion vom Pushout erhält man folgendes:
\begin{mathpar}
  H^1(U\amalg_{U\cap V} V,A)=\left\| \sum_{f:U\to BA}\sum_{g:V\to BA} \prod_{x:U\cap V}f(x)=_{BA}g(x) \right\|_0
\end{mathpar}

Das hilft allerdings noch nicht wirklich beim Berechnen von Kohomologiegruppen.
Eine hilfreiche Aussage über Kohomologie von Pushouts ist die Mayer-Vietoris-Sequenz.
Auf diese werden wir jetzt zum Abschluss hinarbeiten.

\begin{lemma}
  Wenn
  \begin{center}
    \begin{tikzcd}
      X\ar[r]\ar[d] & B\ar[d] \\
      A\ar[r] & C
    \end{tikzcd}
  \end{center}
  ein Pullback ist, dann auch
  \begin{center}
    \begin{tikzcd}
      X\ar[r,"f\circ\pi_1"]\ar[d] & C\ar[d,"\Delta"] \\
      A\times B\ar[r,"f\times g"] & C\times C
    \end{tikzcd}
  \end{center}
\end{lemma}
\begin{beweis}[Idee]
  Die Kegeltypen sind auf hinreichend gutmütige Art äquivalent: \\
  Die rechte Abbildung im zweiten Pullback lässt sich äquivalent ersetzen durch:
  \begin{mathpar}
    (x,y,p)\mapsto p: \sum_{x,y:C}x=y\to C\times C
  \end{mathpar}
  Damit entspricht ein $Z$-Kegel auf dem zweiten Winkel einer Auswahl eines Paars von Abbildungen $\phi:Z\to A$ und $\psi:Z\to B$ zusammen mit einer abhängigen Abbildung, die für jedes $z:Z$ eine Gleichheit $f(\phi(z))=g(\psi(z))$ auswählt (bis auf Gleichheit). Das sind dieselben Daten wie für einen $Z$-Kegel auf dem ersten Winkel.
\end{beweis}

\begin{lemma}
  Sei $A$ punktiert, dann gibt es einein Pullback
  \begin{center}
    \begin{tikzcd}
      \Omega A \ar[r]\ar[d] & \eins\ar[d] \\
      \Omega A \times \Omega A\ar[r,"d"] & \Omega A
    \end{tikzcd}
  \end{center}
  Mit $d(p,q)\equiv q\kon p^{-1}$.
\end{lemma}
\begin{beweis}
  Es reicht, das für den kanonischen Pullback nachzurechnen:
  \begin{align*}
    &\sum_{(p,q):\Omega A\times \Omega A}\sum_{\ast:\eins} q\kon p^{-1}=\refl_\ast\\
    \simeq & \sum_{p:\Omega A}\sum_{q:\Omega A}q=p \\
    \simeq & \sum_{p:\Omega A}\eins \\
    \simeq & \Omega A
  \end{align*}
\end{beweis}

\begin{lemma}
  Sei $S$ ein Typ und
  \begin{center}
    \begin{tikzcd}
      C\ar[r,"g"]\ar[d,"f",swap] & B\ar[d,"\iota_2"] \\
      A\ar[r,"\iota_1",swap] & X
    \end{tikzcd}
  \end{center}
  ein Pushout. Dann ist
  \begin{center}
    \begin{tikzcd}
      (X\to S)\ar[r,"\_\circ\iota_2"]\ar[d,,"\_\circ\iota_1",swap] & (B\to S)\ar[d,"\_\circ g"] \\
      (A\to S)\ar[r,"\_\circ f",swap] & (C\to S)
    \end{tikzcd}
  \end{center}
\end{lemma}
\begin{beweis}
  \begin{align*}
    &(X\to S) \\
    \simeq &\sum_{\phi:A\to S}\sum_{\psi:B\to S}\prod_{c:C}\phi(f(c))=\psi(g(c)) \\
    \simeq &\sum_{\phi:A\to S}\sum_{\psi:B\to S}\phi \circ f=\psi \circ g \\
    \simeq &\sum_{\phi:A\to S}\sum_{\psi:B\to S}(\_\circ f)(\phi)=(\_ \circ g)(\psi) \\
    \simeq &\mathrm{PB}(\_\circ f,\_ \circ g)
  \end{align*}
\end{beweis}

\begin{theorem}
  Für jeden Pushout
  \begin{center}
    \begin{tikzcd}
      C\ar[r,"g"]\ar[d,"f",swap] & B\ar[d,"\iota_2"] \\
      A\ar[r,"\iota_1",swap] & X
    \end{tikzcd}
  \end{center}
  gibt es für jede abelsche Gruppe $K$ eine lange exakte Sequenz von abelschen Gruppen:
  \begin{center}
    \begin{tikzcd}
      & & \dots\ar[r] & H^{n-1}(C,K)\ar[dll] \\
      & H^n(X,K)\ar[r] & H^n(A,K)\times H^n(B,K)\ar[r,"g^\ast-f^\ast"] & H^n(C,K)\ar[dll] \\
      & H^{n+1}(X,K)\ar[r] & H^{n+1}(A,K)\times H^{n+1}(B,K)\ar[r] & \dots
    \end{tikzcd}
  \end{center}
\end{theorem}
\begin{beweis}
  Zunächst verwandeln wir den gegebenen Pushout für jedes $n:\N$ in einen Pullback:
  \begin{center}
    \begin{tikzcd}
      (X\to B^nK)\ar[r,"\_\circ\iota_2"]\ar[d,"\_\circ\iota_1",swap] & (B\to B^nK)\ar[d,"\_\circ g"] \\
      (A\to B^nK)\ar[r,"\_\circ f",swap] & (C\to B^nK)
    \end{tikzcd}
  \end{center}
  Nun die beiden Lemmata davor anwenden, um per Pullback-Pasting ein großes Faserquadrat zu erhalten:
  \begin{center}
    \begin{tikzcd}
      (X\to B^nK)\ar[r]\ar[d,swap,"\_\circ\iota_1 \iota_2"] &(C\to B^nK)\ar[d]\ar[r] & \eins\ar[d]  \\
      (A\to B^nK)\times (B\to B^nK)\ar[r,"\_\circ f g"] & (C\to B^nK)\times (C\to B^nK)\ar[r,"d"] & (C\to B^nK)
    \end{tikzcd}
  \end{center}
  Darauf können wir die lange exakte Fasersequenz anwenden, um das gewünschte Resultat zu erhalten.
\end{beweis}

\begin{korollar}
  Für $n>0$ und $k>0$ gilt $H^n(S^k,\Z)\simeq \Z$ genau dann, wenn $k=n$ und sonst gilt $H^n(S^k,\Z)\simeq \eins$.
\end{korollar}
\begin{beweis}[Idee]
  Einsetzen:
  \begin{center}
    \begin{tikzcd}
      & & \dots\ar[r] & H^{n-1}(S^k,\Z)\ar[dll] \\
      & H^n(S^{k+1},\Z)\ar[r] & H^n(1,\Z)\times H^n(1,\Z)\ar[r,"-"] & H^n(S^k,\Z)\ar[dll] \\
      & H^{n+1}(S^{k+1},\Z)\ar[r] & H^{n+1}(1,\Z)\times H^{n+1}(1,\Z)\ar[r] & \dots
    \end{tikzcd}
  \end{center}
  \dots und alles nachrechnen.
\end{beweis}


\printindex

\end{document}
