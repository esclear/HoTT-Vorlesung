\documentclass{uebung}

\DeclareMathOperator*{\whisk}{\ast}

\begin{document}
\maketitle{13}

\begin{exercise}[Die Windungszahl ist ein Gruppenhomomorphismus]
\end{exercise}

\begin{exercise}[Kohomologie des Kreises I]
  \begin{enumerate}
    \item Zeige, dass $H^1(S^1,\Z) = \|\sum_{x:S^1} x=x\|_0$.
  \end{enumerate}
  Wir wollen nun verstehen, warum $\|\sum_{x:S^1} x=x\|_0$ äquivalent zu $\|\ast = \ast\|_0$ ist.
  \begin{enumerate}[start=2]
    \item Konstruiere eine faserweise Abbildung $F:\prod_{x:S^1} (x = x) \to (\ast = \ast)$, sodass $F_\ast\equiv\id_{\ast = \ast}$.
    \item Zeige, dass die induktiv definierte Abbildung
      \begin{align*}
        f:\left(\sum_{x:S^1} x=x\right) &\to (\ast=\ast)\\
        (x,p)&\mapsto F_xp
      \end{align*}
      eine Retraktion ist mit Rechtsinversem
      \begin{align*}
        g:(\ast=\ast) &\to \left(\sum_{x:S^1} x=x\right)\\
        q &\mapsto (\ast,q)
      \end{align*}
    \item Finde einen Term des folgenden Typs:
      $$
      \prod_{x:S^1} \ast=x \to \prod_{p:x=x} (g\circ f)(x,p) = (x,p).
      $$
    \item Konstruiere daraus einen Term des Typs
      $$
      \prod_{x:S^1} \|\ast=x\|_{-1} \to \prod_{p:x=x} |(g\circ f)(x,p)|_0 = |(x,p)|_0.
      $$
    \item Folgere schließlich, dass $H^1(S^1,\Z) = \Z$ ist.
  \end{enumerate}
\end{exercise}

\begin{exercise}[Kohomologie des Kreises II]
\end{exercise}

\end{document}
