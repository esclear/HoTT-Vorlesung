\documentclass{uebung}

\begin{document}
\maketitle{10}

\begin{exercise}[Koprodukte von kontrahierbaren Typen und Aussagen]
  Seien $A,B:\mU$.
  Zeige:
  \begin{enumerate}
    \item Sind $A$ und $B$ kontrahierbar, so ist $A \amalg B$ nicht kontrahierbar.
    \item Sind $A$ und $B$ Aussagen, so ist $A \amalg B$ eine Aussage genau dann, wenn $A \to \neg B$.
  \end{enumerate}
\end{exercise}

\begin{exercise}[Retrakte von $n$-Typen]
  Sei $r:A \to B$ eine \emph{Retraktion}, existiere also ein Rechtsinverses $s:B \to A$ zu $r$.
  \begin{enumerate}
    \item Finde eine Retraktion $s(x) =_A s(y) \to x=_B y$ für alle $x,y:B$.
    \item Sei $n\geq -2$ ein Abschneidungslevel und $A$ ein $n$-Typ.
      Zeige, dass $B$ ein $n$-Typ ist.
  \end{enumerate}
\end{exercise}

\begin{exercise}[Diese anderen ganzen Zahlen]
  Die ganzen Zahlen $\Z$ wurden als Mengenquotient von $\N\times\N$ nach der Relation $\sim_\Z$ eingeführt.
  \begin{enumerate}
    \item Zeige, dass $\sim_Z$ eine Äquivalenzrelation auf $\N$ ist.
  \end{enumerate}
  Im Folgenden konstruieren wir ein Repräsentantensystem für diese Äquivalenzrelation.
  \begin{enumerate}[start=2]
    \item Zeige, dass für je zwei natürliche Zahlen $n,k:\N$ eine natürliche Zahl $m:\N$ existiert, sodass $(n,k) \sim_\Z (m,0)$ oder $(n,k) \sim_\Z (0,m)$ gilt.
    \item Zeige, dass für je zwei natürliche Zahlen $m,m':\N$ die folgenden Implikationen gelten:
      \begin{align*}
        (m,0) \sim_\Z (n,0) &\to m=n\\
        (0,m) \sim_\Z (0,n) &\to m=n\\
        (m,0) \sim_\Z (0,n) &\to m=n=0
      \end{align*}
  \end{enumerate}
  Sei nun $\Z'$ der induktiv definierte Typ der ganzen Zahlen aus Abschnitt 3.3.
  \begin{enumerate}[start=4]
    \item Finde eine Funktion $\N\times\N\to\Z'$, welche eine Äquivalenz zwischen $\Z$ und $\Z'$ induziert.
  \end{enumerate}
  Ab sofort können wir also $\Z$ und $\Z'$ miteinander identifizieren.
\end{exercise}

\begin{exercise}[Die Windungszahl ist ein Gruppenhomomorphismus]
\end{exercise}

\end{document}
