\documentclass{uebung}

\begin{document}
\maketitle{12}

\begin{exercise}[Die universelle Eigenschaft der Abschneidung]
  Sei $n:\N_{-2}$ ein Abschneidungslevel und $A:\mU$ ein Typ.
  \begin{enumerate}
    \item Konstruiere das Rekursionsprinzip für für die $n$-Abschneidung $\|A\|_n$ , finde also einen Term
      $$
      \rec{\|A\|_n}:\prod_{B:\nType{n}}(A \to B) \to (\|A\|_n \to B),
      $$
      sodass $\rec{\|A\|_n}(B,f)(|a|_n) \equiv f(a)$ für jedes $a:A$.
    \item Konstruiere das Eindeutigkeitsprinzip für für die $n$-Abschneidung $\|A\|_n$:
      $$
      \mathrm{uniq}_{\|A\|_n}:\prod_{B:\nType{n}}\prod_{g,g':\|A\|_n \to B} \left((g\circ |\_|_n) \sim (g'\circ |\_|_n)\right) \to g \sim g'
      $$
    \item Zeige, dass $\rec{\|A\|_n}$ faserweise invers ist zu folgender Abbildung:
      \begin{align*}
        &|\_|_n^*:\prod_{B:\nType{n}}(\|A\|_n \to B) \to (A \to B)\\
        &|\_|_n^*(B,g) \colonequiv g \circ |\_|_n
      \end{align*}
    \item Folgere, dass $A$ genau dann ein $n$-Typ ist, wenn $|\_|_n:A\to \|A\|_n$ eine Äquivalenz ist.
  \end{enumerate}
\end{exercise}

\begin{exercise}[Funktorialität der Abschneidung]
\end{exercise}

\end{document}
