\documentclass{uebung}

\begin{document}
\maketitle{6}

\begin{exercise}[Abhängige Summen verhalten sich sinnvoll]
  Seien $A:\mU$ und $B,C:A\to\mU$. Zeige dass gilt:
  \begin{enumerate}
  \item $\left(\sum_{x:A}\sum_{b:B(x)}C(x)\right)\simeq\left(\sum_{x:A}\sum_{c:C(x)}B(x)\right)$.
  \item $\left(\sum_{z:\sum_{x:A}B(x)}C(\pi_1(z))\right)\simeq\left(\sum_{x:A}\sum_{b:B(x)}C(x)\right)$.
  \item $\left(\sum_{x:A}\sum_{b:B(x)}\eins\right)\simeq\sum_{x:A}B(x)$.
  \end{enumerate}
\end{exercise}

\begin{exercise}[Kontrahierbare Basen darf man kürzen]
  Seien $A:\mU$ kontrahierbar mit Kontraktionszentrum $a:A$ und $B:A\to\mU$. Zeige dass gilt:
  \begin{mathpar}
    \left(\sum_{x:A}B(x)\right)\simeq B(a)
  \end{mathpar}
\end{exercise}

\begin{bonus}[Jeder abhängige Typ besteht aus Fasern einer Abbildung]
  Seien $A:\mU$, $B:A\to\mU$.
  Zeige unter Verwendung der Aussagen aus Aufgabe 1 und 2, dass es eine faserweise Äquivalenz
  \begin{mathpar}
    f:\prod_{x:A} (\pi_1^{-1}(x)\simeq B(x))
  \end{mathpar}
  gibt, wobei $\pi_1:\sum_{x:A}B(x)\to A$ ist.
\end{bonus}

\end{document}
