\documentclass{uebung}

\begin{document}
\maketitle{4}

\begin{exercise}[Kontrahierbare Typen sind äquivalent zu $\einheit$]
  Sei $A$ ein kontrahierbarer Typ.
  Finde Funktionen $f:A\to\einheit$ und $g:\einheit\to A$ welche invers zueinander sind.
\end{exercise}

\begin{exercise}[Rechnen mit Homotopien und Inversen]
  Seien $A,B,C,D$ Typen und $g:B\to C$ eine Funktion.
  \begin{enumerate}
    \item Sei $g':B\to C$ eine Funktion mit $g\sim g'$, also $g$ homotop zu $g'$.
    Zeige:
    \begin{enumerate}
      \item Für $h:C\to D$ ist $h\circ g$ homotop zu $h\circ g'$.
      \item Für $f:A\to B$ ist $g\circ f$ homotop zu $g'\circ f$.
    \end{enumerate}
  \item Seien $i,i':C\to B$ jeweils invers zu $g$.
    Dann sind $i$ und $i'$ homotop zueinander.
  \item Sei $i:C\to B$ invers zu $g$ und seien $h:C\to D$ und $j:D\to C$ zueinander inverse Funktionen.
    Dann sind $h\circ g$ und $i\circ j$ invers zueinander.
  \end{enumerate}
\end{exercise}

\begin{exercise}[Intuitionistische Logik unter der Curry-Howard Korrespondenz]
  Seien $A,B$ Typen.
  Wir setzen $\neg A:\equiv A\to\leer$ und $A\leftrightarrow B:\equiv (A\to B)\times (B\to A)$.
  Zeige:
  \begin{multicols}{2}
    \begin{enumerate}
      \item $A\to\neg\neg A$
      \item $(A\to B)\to(\neg B\to\neg A)$
      \item $\neg\neg\neg A\to\neg A$
      \item\label{deMorgan} $\neg(A+B) \leftrightarrow (\neg A\times \neg B)$
      \item $\neg(A\times \neg A)$
      \item $\neg\neg(A+\neg A)$
    \item $(A+\neg A)\to (\neg\neg A\to A)$.
    \end{enumerate}
  \end{multicols}
  Sei $\mathrm{deMorgan}$ der Term aus Teilaufgabe \ref{deMorgan}. Ist $\pi_1(\mathrm{deMorgan})$ invers zu $\pi_2(\mathrm{deMorgan})$?
\end{exercise}

\begin{exercise}[Alternative Charakterisierung von Aussagen]
  Sei $A$ ein Typ.
  Zeige, dass $A$ genau dann eine Aussage ist, wenn alle Gleichheitstypen von $A$ kontrahierbar sind, das heißt, konstruiere Funktionen
  \begin{enumerate}
    \item $\prod_{x,y:A}\isContr(x=_Ay)\to\isProp(A)$,
    \item \textbf{Bonus.} $\isProp(A)\to\prod_{x,y:A}\isContr(x=_Ay)$.
  \end{enumerate}
  Verwende dies, um zu zeigen, dass
  \begin{enumerate}[start=3]
    \item $\isContr(A)\to\isProp(A)$, {\tiny Tipp:Blatt 3}
    \item $\isProp(A)\to\isSet(A)$.
  \end{enumerate}
\end{exercise}

\begin{bonus}[Kontrahierbarkeit ist eine Aussage]
  Zeige, dass für jeden Typ $A$ der Typ
  \begin{mathpar}
    \isContr'(A)\equiv\sum_{c:A}\prod_{x:A}c=x
  \end{mathpar}
  ein Aussage ist. Folgere daraus, dass auch $\isContr(A)$ eine Aussage ist.
\end{bonus}

\end{document}
