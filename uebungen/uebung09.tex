\documentclass{uebung}

\begin{document}
\maketitle{9}

\begin{exercise}[Intervallrekursion]
  Sei $A:\mU$ ein Typ.
  Zeige:
  \begin{enumerate}
  \item Für $a,a':A$ und $a_s:a=_A a'$ gibt es
    \begin{mathpar}
      \rec{I}(A,a,a',a_s):I\to A
    \end{mathpar}
    mit $\rec{I}(A,a,a',a_s)(0_I)\equiv a$, $\rec{I}(A,a,a',a_s)(1_I)\equiv a'$ und $\rec{I}(A,a,a',a_s)(s)\equiv a_s$.
  \item Für Funktionen $I\to A$ gilt folgendes Eindeutigkeitsprinzip:
    $$
    \prod_{f:I\to A} f = \rec{I}(A,f(0_I),f(1_I),f(s))
    $$
  \item Es ist $(I \to A) \simeq \sum_{x,y:A} x =_A y$.
  \end{enumerate}
\end{exercise}

\begin{exercise}[Die Vorgängerfunktion auf $\Z$]
  Definiere eine Inverse $\mathrm{pred}_\Z : \Z \to \Z$ zur Nachfolgerfunktion $\mathrm{succ}_\Z:\Z\to\Z$.
\end{exercise}

\end{document}
