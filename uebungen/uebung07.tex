\documentclass{uebung}

\begin{document}
\maketitle{7}

\begin{exercise}
  Zeige einmal mit Äquivalenzinduktion, einmal ohne: Äquivalenzen induzieren Äquivalenzen (also ap) auf Gleichheitstypen. 
\end{exercise}

\begin{exercise}
  Folgere aus $\sum_{A:\mU}A\simeq B$ kontrahierbar für jedes $B:\mU$ das Univalenzaxiom.
\end{exercise}

\begin{exercise}
  Zeige: $\sum f$ induziert $f$ zwischen den Fasertypen.
\end{exercise}

\begin{exercise}[Gleichheitstyp von Koprodukten]
  Sei $A,B:\mU$ und $\Eq{\amalg}:A\amalg B\to A\amalg B\to \mU$ gegeben durch:
  \begin{align*}
    \Eq{\amalg}(\iota_1(a),\iota_1(a'))&\colonequiv (a=_A a') \\
    \Eq{\amalg}(\iota_1(a),\iota_2(b))&\colonequiv \leer \\
    \Eq{\amalg}(\iota_2(b),\iota_1(a))&\colonequiv \leer \\
    \Eq{\amalg}(\iota_2(b),\iota_2(b'))&\colonequiv (b=_B b')
  \end{align*}
  Zeige mit Theorem 2.3.6 der Vorlesung dass eine geeignete faserweise Abbildung für $x:A\amalg B$
  \begin{mathpar}
    f_x:\prod_{y:A\amalg B}(x=y)\to\Eq{\amalg}(x,y)
  \end{mathpar}
  eine Äquivalenz ist.
\end{exercise}

\end{document}
