\documentclass{uebung}

\begin{document}
\maketitle{11}

\begin{exercise}[Funktorialität der Verschleifung]
  Seien $(f,p_f):(A,a)\to (B,b)$ und $(g,p_g):(B,b)\to (C,c)$ punktierte Abbildungen.
  \begin{enumerate}
    \item Finde eine Gleichheit $p_{g\circ f}:(g\circ f)(a)=c$, sodass $(g\circ f,p_{g\circ f})$ eine punktierte Abbildung $(A,a)\to (C,c)$ wird.
    \item Zeige, dass $\Omega(g,p_g)\circ\Omega(f,p_f) \sim \Omega(g\circ f,p_{g\circ f})$ als Abbildungen $A\to C$.
    \item Mit Funktionsextensionalität sind die beiden Abbildungen sogar gleich.
      Sind sie aber auch gleich als punktierte Abbildungen?
  \end{enumerate}
\end{exercise}

\end{document}
