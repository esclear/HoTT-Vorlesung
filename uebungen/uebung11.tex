\documentclass{uebung}

\begin{document}
\maketitle{11}

\begin{exercise}[Schnitte von $\tilde{S^1}$]
  Finde Terme der folgenden Typen:
  \begin{enumerate}
    \item $\prod_{x:S^1}\|\ast =_{S^1} x\|$
    \item $\neg\left(\prod_{x:S^1}\ast =_{S^1} x\right)$
      {\tiny Tipp: Sonst ließe sich folgern, dass jeder Typ eine Menge ist, sogar Universen!}
  \end{enumerate}
\end{exercise}

\begin{exercise}[Funktorialität der Verschleifung]
  Seien $(f,p_f):(A,a)\to (B,b)$ und $(g,p_g):(B,b)\to (C,c)$ punktierte Abbildungen.
  \begin{enumerate}
    \item Finde eine Gleichheit $p_{g\circ f}:(g\circ f)(a)=c$, sodass $(g\circ f,p_{g\circ f})$ eine punktierte Abbildung $(A,a)\to (C,c)$ wird.
    \item Zeige, dass $\Omega(g,p_g)\circ\Omega(f,p_f) \sim \Omega(g\circ f,p_{g\circ f})$ als Abbildungen $A\to C$.
    \item Mit Funktionsextensionalität sind die beiden Abbildungen sogar gleich.
      Sind sie aber auch gleich als punktierte Abbildungen?
  \end{enumerate}
\end{exercise}

\begin{exercise}[$\Omega f$ ist eine Vergleichsabbildung]
  Sei $(A,a)$ ein punktierter Typ.
  Laut Bemerkung 3.4.5 ist der Schleifenraum von $(A,a)$ die Spitze eines Pullback-Quadrates:
  \begin{center}
    \begin{tikzcd}
      \Omega(A,a)
      \arrow[r]
      \arrow[d]
      & \einheit
      \arrow[d, "a"]
      \\
      \einheit
      \arrow[r, "a"]
      & A
    \end{tikzcd}
  \end{center}
  Hier fassen wir den Basispunkt $a:A$ als die Abbildung $\rec\einheit(A,a):\einheit \to A$ auf.
  Sei nun $(f,p_f):(A,a)\to (B,b)$ eine punktierte Abbildung.
  Finde ein kommutatives Quadrat der Form
  \begin{center}
    \begin{tikzcd}
      \Omega(A,a)
      \arrow[r, "{\Omega(f,p_f)}"]
      \arrow[d,"\sim" sloped]
      & \Omega(B,b)
      \arrow[d,"\sim" sloped]
      \\
      \mathrm{PB}(a,a)
      \arrow[r]
      &
      \mathrm{PB}(b,b)
    \end{tikzcd}
    .
  \end{center}
\end{exercise}

\begin{exercise}[Die Yoneda-Einbettung erhält und reflektiert Äquivalenzen]
  Für beliebige $B,C:\mU$ sei folgende Abbildung gegeben:
  \begin{align*}
    \mathrm{y}:(B\to C) &\to \prod_{A:\mU} (A \to B) \to (A \to C)\\
    \mathrm{y}(g)_A &\colonequiv (f \mapsto g\circ f)
  \end{align*}
  Zeige, dass $g:B \to C$ genau dann eine Äquivalenz ist, wenn $\mathrm{y}(g)$ eine faserweise Äquivalenz ist.
\end{exercise}

\begin{bonus}[Universelle Eigenschaft von Pullbacks I]
  Sei
  \begin{tikzcd}
    A \arrow[r,"f"] &B \arrow[r,leftarrow,"g"] &C
  \end{tikzcd}
  ein Winkel und $X:\mU$ ein Typ.
  Der Typ der $X$-Kegel über diesem Winkel ist definiert als
  $$
  \mathrm{Cone}(f,g,X) \colonequiv \sum_{\varphi:X\to A}\sum_{\psi:X\to B} f\circ\varphi \sim g\circ\psi.
  $$
  Ist $(\varphi,\psi,H)$ ein solcher $X$-Kegel, so erhalten wir für jedes $Y:\mU$ eine kanonische Abbildung
  \begin{align*}
    &\Phi\colonequiv\Phi_{X,\varphi,\psi,H}(Y):(Y \to X) \to \mathrm{Cone}(f,g,Y)\\
    &\Phi(\vartheta)\colonequiv (\varphi\circ\vartheta,\psi\circ\vartheta,H_\vartheta).
  \end{align*}
  Zeige:
  \begin{enumerate}
    \item Falls $X\equiv\mathrm{PB}(f,g)$ und $(\varphi,\psi,H)\equiv(\pi_1,\pi_2,\pi_3)$ der kanonische $X$-Kegel, so ist $\Phi$ invers zur Vergleichsabbildung aus der Vorlesung
      $$
      V(Y,\_,\_,\_):\mathrm{Cone}(f,g,Y)\to (Y\to\mathrm{PB}(f,g)).
      $$
    \item Das Quadrat
      \begin{center}
        \begin{tikzcd}
          X
          \arrow[d,"\varphi"']
          \arrow[r,"\psi"]
          & B
          \arrow[d,"g"]
          \arrow[dl, phantom, "H" font=\small]
          \\
          A
          \arrow[r,"f"']
          & C
        \end{tikzcd}
      \end{center}
      ist genau dann ein Pullback-Quadrat, wenn $\Phi$ für jedes $Y$ eine Äquivalenz ist.
  \end{enumerate}
\end{bonus}

\begin{bonus}[Universelle Eigenschaft von Pullbacks II]
  Sei
  \begin{tikzcd}
    A \arrow[r,"f"] &B \arrow[r,leftarrow,"g"] &C
  \end{tikzcd}
  ein Winkel und seien $(\varphi,\psi,H):\mathrm{Cone}(f,g,X)$ sowie $(\varphi',\psi',H'):\mathrm{Cone}(f,g,Y)$ Kegel darüber.
  Der Typ der \emph{Kegelmorphismen} $(\varphi',\psi',H')\to (\varphi,\psi,H)$ sei definiert als
  $$
  \sum_{\vartheta:Y \to X}
  \sum_{h:\varphi\circ\vartheta\sim\varphi'}
  \sum_{k:\psi\circ\vartheta\sim\psi'}
  h^{-1} \kon H_\vartheta \kon k = H'.
  $$
  Ist dieser Typ für alle $Y:\mU$ und jeden $Y$-Kegel $(\varphi',\psi',H')$ kontrahierbar, so nennen wir $(\varphi,\psi,H)$ einen \emph{terminalen Kegel}.
  \begin{enumerate}
    \item Zeige, dass der Typ $(\varphi',\psi',H')\to (\varphi,\psi,H)$ äquivalent zur Faser der Abbildung $\Phi_{X,\varphi,\psi,H}(Y)$ über $(\varphi',\psi',H')$ ist.
    \item Folgere, dass das Quadrat
      \begin{center}
        \begin{tikzcd}
          X
          \arrow[d,"\varphi"']
          \arrow[r,"\psi"]
          & B
          \arrow[d,"g"]
          \arrow[dl, phantom, "H" font=\small]
          \\
          A
          \arrow[r,"f"']
          & C
        \end{tikzcd}
      \end{center}
      genau dann ein Pullbackquadrat ist, wenn $(\varphi,\psi,H)$ ein terminaler Kegel mit Spitze $X$ ist.
  \end{enumerate}
\end{bonus}

\end{document}
