\documentclass{uebung}

\begin{document}
\maketitle{8}

\begin{exercise}[Rückzug abhängiger Typen und deren Summen]
  Sei $f:A \to B$ eine Funktion und $C : B \to \mU$ ein abhängiger Typ über $B$.
  Dies induziert einen abhängigen Typen $C \circ f : A \to \mU$ über $A$.
  \begin{enumerate}
    \item Sei $p:a=_A a'$. 
      Zeige $\transp_{C\circ f}(p) = \transp_C (f(p))$.
    \item Finde eine kanonische Abbildung $\sum_{x:A} C(f(x)) \to \sum_{y:B} C(y)$.
    \item Sei $f$ eine Äquivalenz.
      Zeige $\sum_{x:A} C(f(x)) \simeq \sum_{y:B} C(y)$.
  \end{enumerate}
\end{exercise}

\begin{exercise}[Gleichheit punktierter Typen]
  Ein \emph{punktierter Typ} ist ein Paar bestehend aus einem Typen $A$ und einem Term $a:A$.
  Den Typ $\sum_{A:\mU}A$ aller punktierten Typen in einem Universum $\mU$ bezeichnen wir mit $\mU_*$.

  Eine \emph{punktierte Äquivalenz} zwischen zwei punktierten Typen $(A,a)$, $(B,b)$ ist eine Äquivalenz $f:A\simeq B$ mit $f(a)=b$.
  Den Typ aller punktierten Äquivalenzen zwischen $(A,a)$ und $(B,b)$ definieren wir als
  $$
  (A,a) \simeq_* (B,b) \colonequiv \sum_{f:A\simeq B} f(a) = b.
  $$

  Finde für je zwei punktierte Typen $(A,a), (B,b) : \mU_*$ eine Äquivalenz
  $$
  \left((A,a) =_{\mU_*} (B,b)\right) \simeq \left((A,a) \simeq_* (B,b)\right).
  $$
\end{exercise}

\begin{exercise}[Transporte in Gleichheitstypen]
  Sei $A:\mU$ ein Typ.
  \begin{enumerate}
    \item Zeige für den abhängigen Typen $C(x)\colonequiv x =_A x$:
      $$
      \prod_{a,a':A}\prod_{p:a =_A a'} \prod_{q:C(a)} \transp_C (p,q) =_{C(a')} p^{-1} \cdot q \cdot p
      $$
    \item Sei $B:\mU$ ein Typ und $f,g : A \to B$ Funktionen.
      Zeige für den abhängigen Typen $C(x) \colonequiv f(x)=_{B} g(x)$:
      $$
      \prod_{a,a':A}\prod_{p:a =_A a'} \prod_{q:C(a)} \transp_C (p,q) =_{C(a')} \mathrm{ap}(f,p)^{-1} \cdot q \cdot \mathrm{ap}(g,p)
      $$
  \end{enumerate}
\end{exercise}

\begin{exercise}[Funktionen aus $S^1$ sind Schleifen]
  Sei $A$ ein Typ.
  \begin{enumerate}
    \item Verwende Funktionsextensionalität um das folgende Eindeutigkeitsprinzip für $S^1 \to A$ zu zeigen:
      $$
      \prod_{f:S^1 \to A} f = \rec{S^1}(A, f(\ast), f(l))
      $$
    \item Folgere, dass $(S^1\to A) \simeq \sum_{x:A} x=x$ gilt.
  \end{enumerate}
\end{exercise}

\end{document}
