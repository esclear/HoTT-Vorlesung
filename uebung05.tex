\documentclass{uebung}

\begin{document}
\maketitle{5}

\begin{exercise}[Kommutative Dreiecke]
  Seien $A,B,C$ Typen.
  Ein \emph{kommutatives Dreieck} besteht aus Funktionen
  \begin{center}
    \begin{tikzcd}[column sep=small]
      A
        \arrow[rr, "h"]
        \arrow[dr, "f"']
      && C
      \\
      & B
        \arrow[ur, "g"']
    \end{tikzcd}
  \end{center}
  sowie einer Homotopie $h\sim g\circ f$.
  Definiere den Typ der kommutativen Dreiecke anhand dieser Beschreibung und zeige für jedes kommutative Dreieck obiger Form:
  \begin{enumerate}
    \item Besitzt $h$ ein Rechtsinverses, so auch $g$.
      Die Umkehrung gilt, falls $f$ ein Rechtsinverses hat.
    \item Besitzt $h$ ein Linksinverses, so auch $f$.
      Die Umkehrung gilt, falls $g$ ein Linksinverses hat.
    \item Sind zwei der drei Funktionen $f,g,h$ Äquivalenzen, so auch die dritte.
  \end{enumerate}
\end{exercise}

\begin{exercise}[Äquivalenzen]
  Sei $A,B,C,D$ Typen.
  Finde Äquivalenzen:
  \begin{multicols}{2}
    \begin{enumerate}
      \item $(\einheit \to A)\simeq A$.
      \item $(\zwei \to A)\simeq A\times A$. 
      \item $x,y:A \yields (x=y) \simeq (y=x)$.
      \item $x,y,z:A,p:x=y \yields (y=z) \simeq (x=z)$.
      \item $x,y,z:A,q:y=z \yields (x=y) \simeq (x=z)$.
      \item $x,y:A,p:x=y \yields (y=y) \simeq (x=x)$.
      \item $\beta:B\simeq C\yields (A\to B) \simeq (A\to C)$
      \item $\alpha:A\simeq B\yields (B\to C) \simeq (A\to B)$
      \item $\alpha:A\simeq B\yields (B\to B) \simeq (A\to A)$
    \end{enumerate}
  \end{multicols}
\end{exercise}

\begin{bonus}[Quasi-Inversen kohärent machen]
  Seien $A,B$ Typen, $f:A\to B$, $g:B\to A$ Funktionen und $H:g\circ f \sim\id$, $K:f\circ g\sim\id$ Homotopien.
  Zeige dass für
  \begin{mathpar}
    K':\equiv (y:B)\mapsto K_{f(g(y))}^{-1}\kon f(H_{g(y)})\kon K_y : f\circ g\sim\id
  \end{mathpar}
  gilt:
  \begin{mathpar}
    \prod_{x:A}f(H_x)=K'_{f(x)}
  \end{mathpar}
  Damit ist $g$ eine kohärente Inverse von $f$.
\end{bonus}

\end{document}
