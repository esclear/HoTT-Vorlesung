\documentclass{uebung}

\begin{document}
\maketitle{1}

\begin{exercise}[Urteilsgleichheit ist eine Kongruenzrelation]
  Formuliere Regeln, welche die folgenden Aussagen kodieren:
  \begin{enumerate}
    \item Urteilsgleichheit von Typen ist eine Äquivalenzrelation zwischen Typen, welche mit der Formationsregel $\Pi\mathrm{F}$ verträglich ist.
    \item Urteilsgleichheit von Termen ist eine Äquivalenzrelation zwischen Termen desselben Typs, welche mit der Einführungsregel $\Pi\mathrm{I}$ und Eliminationsregel $\Pi\mathrm{E}$ verträglich ist.
  \end{enumerate}
\end{exercise}

\begin{exercise}[Extensionale Urteilsgleichheit abhängiger Funktionen]
  Seien $f:\Pi_{x:A}B(x)$ und $g:\Pi_{x:A}B(x)$ abhängige Funktionen, sodass $f(x)\equiv g(x):B(x)$ im Kontext einer Variable $x:A$ gilt.
  Gib eine Herleitung für $f\equiv g:\Pi_{x:A}B(x)$ an.
\end{exercise}

\begin{exercise}[Assoziativität der Komposition]
  Seien $f:A\to B$, $g:B\to C$ und $h:C\to D$ Funktionen.
  \begin{enumerate}
    \item Stelle einen Herleitungsbaum für $x\mapsto h(g(f(x))):A\to D$ auf.
    \item Leite aus $x\mapsto h(g(f(x))):A\to D$ folgendes ab:
      \begin{enumerate}
        \item $h\circ (g\circ f):A\to D$.
        \item $(h\circ g)\circ f:A\to D$.
      \end{enumerate}
    \item Kombiniere dies zu einer Herleitung von $h\circ(g\circ f)\equiv (h\circ g)\circ f:A\to D$.
  \end{enumerate}
\end{exercise}

\begin{exercise}[Konstante Funktionen]
  Sei $B$ ein Typ.
  Definiere eine konstante Funktion $\mathrm{const}_y:B\to C$ im Kontext $y:C$ und zeige
  \begin{enumerate}
    \item $f:C\to D,z:D\vdash\mathrm{const}_z\circ f \equiv \mathrm{const}_z:B\to D$,
    \item $g:A\to B, x:B\vdash g\circ\mathrm{const}_y \equiv \mathrm{const}_{g(x)}$.
  \end{enumerate}
\end{exercise}

\begin{exercise}[Zwei Einführungsregeln für Nachfolger]
  Zeige, dass die folgenden beiden Einführungsregeln für den Nachfolger einer natürlichen Zahl äquivalent sind:
  \begin{mathpar}
    \inferrule{
      \Gamma\vdash x:\mathbb{N}
    }{
      \Gamma\vdash\mathrm{succ}(x):\mathbb{N}
    }{
      \mathbb{N}\mathrm{I2}
    }
    \and
    \inferrule{
      \Gamma\text{ Kontext}
    }{
      \Gamma,x:\mathbb{N}\vdash\mathrm{succ}(x):\mathbb{N}
    }{
      \mathbb{N}\mathrm{I2'}
    }
  \end{mathpar}
  {\tiny Die linke Regel ist aus dem Skript, die rechte aus der Vorlesung.}
\end{exercise}

\end{document}
